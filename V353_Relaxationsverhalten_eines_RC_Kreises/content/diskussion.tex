\section{Diskussion}
\label{sec:diskussion}

Es war kein Wert für die Zeitkonstante auf der Apparatur angegeben.
Daher können lediglich die Ergebnisse aus den verschiedenen Messmethoden verglichen werden.


Die Werte für $RC$ aus \autoref{auswertung:1} und \autoref{auswertung:2} betrugen
$\SI{2.03(4)}{\milli\second}$ beziehungsweise $\SI{2.103(18)}{\milli\second}$.
Das entspricht einer relativen Abweichung zueinander von $\SI{3.47}{\percent}$.

Der in \autoref{auswertung:3} bestimmte Wert von $\SI{11.0(7)}{\milli\second}$
weicht hingegen um ein Vielfaches ab.
Es handelt sich augenscheinlich um einen systematischen Fehler.
Die Fehlerursache konnte nicht genau bestimmt werden,
aber es ist bemerkenswert,
dass der Fit selbst relativ gut funktioniert zu haben scheint.
% Daher lässt sich vermuten,
% dass kein gravierender Fehler in der Durchführung vorliegt.
Es wurde außerdem das abgelesene $b$ mit den Werten verglichen,
die direkt aus der Frequenz $\omega$ bestimmt werden können.
Dabei wurden keine nennenswerten Abweichungen festgestellt,
sodass gefolgert werden kann,
dass die für den großen Fehler verantwortlichen Abweichungen bei $a$ liegen.

Zur Durchführung ist des Weiteren zu erwähnen,
dass das Ablesen des analogen Oszilloskops für niedrige Frequenzen immer schwieriger wurde,
weil das Bild zu flackern begann.
Deshalb fehlen für die niedrigsten Frequenzwerte auch die Werte der Phasenverschiebung.

% Generatorinnenwiderstand von $\SI{600}{\ohm}$…
