\section{Diskussion}
\label{sec:diskussion}

\subsection{Abweichungen}

Die Messung der integralen \hyperref[sec:auswertung:energieverteilung]{Energieverteilung der Elektronen}
bei $\SI{148}{\celsius}$
lieferte einen sehr \enquote{glatten} Verlauf,
was das Bestimmen der Wendepunkte (der integralen Energieverteilung)
beziehungsweise der Minima der differenziellen Energieverteilung sehr ungenau machte.
Das zugehörige Kontaktpotential kann daher nur als eine grobe Schätzung angesehen werden.


Die Franck-Hertz-Kurven zeigen im Wesentlichen das erwartete Bild.
Das heißt auch, dass sich kein sich kein idealer Verlauf ergibt,
wie er in \autoref{fig:idealer_verlauf} dargestellt war.
Es soll kurz auf die Nebeneffekte aus \autoref{sec:einflüsse}
und ihre Auswirkung auf den Verlauf eingegangen werden:

\begin{description}
  \item[{\hyperref[sec:einflüsse:kontaktpotential]{Kontaktpotential}}]
  Da die effektive Potentialdifferenz geringer als die angelegte Spannung ist,
  ist die reale Franck-Hertz-Kurve relativ zur idealen Kurve
  in Richtung größerer Spannungen verschoben.

  \item[{\hyperref[sec:einflüsse:energieverteilung]{Energieverteilung der Elektronen}}]
  Die Maxima sind deutlich \enquote{stumpfer} als im idealen Verlauf.
  Dieser Effekt lässt sich unter anderem damit begründen,
  dass die aus der Glühkathode ausgelösten Elektronen nicht mit der gleichen Energie starten.
  Das Resultat ist bildlich gesprochen die Summe vieler zufällig in der X-Achse verschobener (idealer) Verläufe.

  \item[{\hyperref[sec:einflüsse:dampfdruck]{Temperaturabhängigkeit des Dampfdruckes}}]
  Wie zuvor besprochen,
  variiert die mittlere freie Weglänge der Elektronen
  mit dem Dampfdruck und somit auch mit der Temperatur.
  Daher nimmt die Anzahl der Stöße mit steigender Temperatur zu,
  was wiederum ausgeprägtere Extrema zur Folge hat.
\end{description}


Die aus der \hyperref[fig:franck_hertz_kurve_2]{zweiten Franck-Hertz-Kurve} bestimmte Anregungsenergie
lautete $\SI{5.24(12)}{\electronvolt}$.
Das entspricht einer Abweichung um $\SI{6.92}{\percent}$
vom Literaturwert $\SI{4.9}{\electronvolt}$ \cite{anregungsenergie}.

Das anhand der \hyperref[fig:franck_hertz_kurve_1]{ersten Franck-Hertz-Kurve} bestimmte Kontaktpotential
weicht merklich von dem zuvor in \autoref{sec:auswertung:energieverteilung} ermittelten ab.
In ersterem Teil wurde bei einer Temperatur von $\SI{148}{\celsius}$
ein Kontaktpotential von $\SI{8.149}{\volt}$ gemessen,
während es in letzterem
bei einer Temperatur von $\SI{166.6}{\celsius}$
$\SI{10.57(32)}{\volt}$ waren.
% 6.696 weicht ebenfalls von beiden anderen K ab…

Da kein Referenzwert zum Kontaktpotential bekannt ist,
kann über die tatäschliche Abweichung keine Aussage getroffen werden.


\subsection{Mögliche Fehlerquellen}

Eine offensichtliche Fehlerquelle ist der verwendete XY-Schreiber.
Indem die Messwerte erst auf Papier geplottet
und anschließend wieder abgelesen und digitalisiert wurden,
% konnte an verschiedensten Stellen…
konnten verschiedenste Abweichungen auftreten:

\begin{description}
  \item[Skala]
  Die Werte wurden mithilfe der vorgedruckten Millimeter-Skala abgelesen
  und anschließend umgerechnet.
  Ungenauigkeiten treten zwangsläufig beim Erstellen der Volt-Skala sowie beim Ablesen von Werten auf.

  \item[Messbereich]
  Der Messbereich des Picoamperemeter wurde während einer Probemessung überschritten.
  Dies äußert sich am XY-Schreiber dadurch,
  dass die Werte \enquote{abgeschnitten} werden.
  Wäre dieses Problem nicht erkannt worden,
  hätten sich unbrauchbare Daten ergeben.

  \item[Genauigkeit und Präzision des Schreibers]
  Hierzu sind keine Angaben bekannt.
  Es ist jedoch gut möglich,
  dass hier ein kleinerer systematischer Fehler auftritt.
  Abgesehen davon trat zwischenzeitlich ein Zittern auf,
  welches aber auch vom Picoamperemeter ausgegangen sein könnte.
\end{description}


Die Brems-/Beschleunigungsspannungen wurden über eine Minute hinweg abgefahren,
wenngleich die Apparatur langsamere Abtastungen zugelassen hätte.
Dies muss sich jedoch nicht nachteilig auswirken, wie der folgende Absatz verdeutlicht.


Der geregelte Heizgenerator erlaubte es nicht,
eine gewünschte Temperatur gleichzeitig einigermaßen schnell zu erreichen
und diese anschließend zu halten.
Die somit entstandenen Temperaturschwankungen können zu Abweichungen geführt haben,
welche für längere Messdauern größer zu vermuten sind.
