\section{Diskussion}
\label{sec:diskussion}
%NOTE: Keine Unterabschnitte hier! (wegen persönlicher Präferenz)

% ▒▒▒ Emissionsspektrum der Kupfer-Röntgenröhre
Die in \autoref{sec:auswertung:emissionsspektrum} bestimmten Energien zu $K_\alpha$ und $K_\beta$
konnten genau bestimmt werden;
die relative Abweichung von den Literaturwerten aus \autoref{tab:vorbereitung_1}
beträgt nur $\SI{0.06}{\kilo\electronvolt}$ bzw. $\SI{0.75}{\percent}$ für $K_\alpha$
und $\SI{0.01}{\kilo\electronvolt}$ bzw. $\SI{0.08}{\percent}$ für $K_\beta$.

% ▒▒▒ Bestimmung der Transmission als Funktion der Wellenlänge
Die Abhängigkeit der Transmission von der Wellenlänge konnte in \autoref{sec:auswertung:transmission}
genau bestimmt werden, wie an \autoref{fig:plt_transmission}
aber auch der geringen Unsicherheit der Parameter $a$ und $b$
erkennbar ist.
Die Unsicherheit der Messwerte ist im Plot kaum zu sehen;
dies ist unter anderem mit der guten Kondition
bei der Division der ähnlich großen und mit einem ähnlich großen relativen Fehler behafteten Impulszahlen
zu erklären:
\[
  \newcommand{\relerr}[1]{ \ensuremath{\frac{\symup{\Delta}{#1}}{#1}} }
  \relerr{T} = \relerr{I_\text{Al}} - \relerr{I_\text{0}} \ .
\]

Die Totzeitkorrektur hatte dabei
wegen der relativ großen Zählraten
einen signifikanten Einfluss.
Ein ungenauer Wert für die Totzeit $\tau$ könnte diese Messungen allerdings verfälscht haben.

% ▒▒▒ Bestimmung der Compton-Wellenlänge
Die Transmissionen $T_{1/2}$ sowie die zugehörigen Wellenlängen $\lambda_{1/2}$
konnten mit ausreichender Präzision bestimmt werden.
Über die Genauigkeit lässt sich mangels Referenz keine Aussage treffen.

Die Compton-Wellenlänge als Differenz $\lambda_2 - \lambda_1$
weicht vom Literaturwert ($\SI{2.43}{\pico\meter}$)
um $\SI{54.94}{\percent}$ ab,
wobei die Unsicherheit hier fast ebenso groß ist.
Damit liegt der bestimmte Wert zwar in der richtigen Größenordnung,
ist jedoch weder genau, noch präzise.

% ▒▒▒ Mögliche Fehlerquellen
Obgleich die Integrationszeit ausreichend groß angesetzt war,
würde sich für noch größere Integrationszeiten naturgemäß
die Messunsicherheit der Zählrate weiter verringern.

Bei geringen Impulszahlen haben mögliche Störeffekte
– beispielsweise äußere Strahlung oder Strahlung, welche nicht den vorgesehenen Weg durch die Apparatur nimmt –
einen größeren Einfluss,
wenn das Geiger-Müller-Zählrohr unzureichend vor diesen abgeschirmt ist.
