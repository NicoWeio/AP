\section{Vorbereitung}
\label{sec:vorbereitung}

    Zur Vorbereitung auf den Versuch sollten die Energien der $K_{\symup{\alpha}}$- und $K_{\symup{\beta}}$-Peaks von Kupfer recherchiert werden,
    da dies das Anodenmaterial der hier verwendeten Röntgenröhre ist.
    Zusätzlich sollten die zugehörigen Wellenlängen berechnet werden.
    Dazu wird die Gleichung
    \begin{equation*}
        E = h \cdot \frac{c}{\lambda} \iff \lambda = h \cdot \frac{c}{E}
    \end{equation*}
    verwendet.\\
    Es ergaben sich die folgenden Werte.

    \begin{table}
        \centering
        \caption{Energien und Wellenlängen der charakteristischen Peaks einer $\ce{Cu}$-Röntgenröhre. \cite{leifi} \cite{tu_dresden}}
        \label{tab:vorbereitung_1}
        \begin{tabular}{S S S S}
            \toprule
            {$E_{\symup{\alpha}} \mathbin{/} \si{\kilo\electronvolt}$} &
            {$\lambda_{\symup{\alpha}} \mathbin{/} \SI{e-10}{\meter}$} &
            {$E_{\symup{\beta}} \mathbin{/} \si{\kilo\electronvolt}$} &
            {$\lambda_{\symup{\beta}} \mathbin{/} \SI{e-10}{\meter}$} \\
            \midrule
            8.10 & 1.5318 & 8.905 & 1.3934 \\
            \bottomrule
        \end{tabular}
    \end{table}

    Mithilfe der Bragg-Bedingung \eqref{eqn:bragg_bedingung} können die zugehörigen Winkel berechnet werden.

    \begin{table}
        \centering
        \caption{Glanzwinkel $\alpha$ der $\ce{Cu}$-Röntgenröhre in erster Beugungsordnung ($n=1$).}
        \label{tab:vorbereitung_2}
        \begin{tabular}{S S}
           \toprule
            {$\alpha_{\symup{\alpha}} \mathbin{/} \si{\degree}$} &
            {$\alpha_{\symup{\beta}} \mathbin{/} \si{\degree}$} \\
            \midrule
            22.3515 & 20.2384 \\
            \bottomrule
        \end{tabular}
    \end{table}

    Zudem sollte die Compton-Wellenlänge berechnet werden.
    Es ergab sich der Wert
    \begin{equation*}
        \lambda_\text{c} = \frac{h}{m_\text{e} c} = \SI{2.424}{\pico\meter} \ .
    \end{equation*}
    mit dem Planck'schen Wirkungsquantum $h = \SI{6.626e-34}{\joule\second}$,
    der Elektronenmasse $m_\text{e} = \SI{9.11e-31}{\kilo\gram}$
    und der Lichtgeschwindigkeit $c = \SI{3e8}{\meter\per\second}$.
