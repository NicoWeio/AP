\section{Theorie} \label{sec:Theorie}

Im Folgenden sollen die theoretischen Grundlagen des Geiger-Müller-Zählrohrs aufgeführt werden.

\subsection{Aufbau des Zählrohrs}

    %TODO: Aufbau des GMZ einfügen (Abb1 in Anleitung)
    Das Geiger-Müller-Zählrohr besteht aus einem zylinderförmigen, negativ geladenen Rohr mit 
    Radius $r_\text{k}$, in dessen Mitte sich ein positiv geladener Draht 
    mit Radius $r_\text{a}$ befindet.
    Der Zylindermantel stellt eine Kathode dar und der Draht eine Anode.
    Zwischen Anode und Kathode befindet sich ein elektrisches Feld, welches aufgrund
    der Symmetrie des Zählrohrs radialsymmetrisch vom Draht nach außen gerichtet ist.
    Für die elektrische Feldstärke ergibt sich
    \begin{equation}
        E(r) = \frac{U}{r \symup{ln}(\frac{r_\text{k}}{r_\text{a}})}
    \end{equation}
    An den Zylindermantel und den Draht ist die Spannung $U$ angeschlossen.
    Durch den Deckel der Röhre können durch ein Eintrittsfenster Teilchen in das Innere des Zylinders 
    gelangen, welches mit einem Gasgemisch, in diesem Fall Halogen, gefüllt ist.

\subsection{Wirkungsweise}

    Das Geiger-Müller-Zählrohr wird zur Bestimmung der Intensität ionisierender Strahlung genutzt.\\
    Wenn ein geladenes Teilchen durch das Eintrittsfenster, welches aus einem sehr durchlässigen Material
    besteht, in das Innere des Zylinders gelangt, ionisiert es Gasatome. Dieser Vorgang
    wird Primärionisation genannt. So entstehen freie Elektronen, welche sich zum positiv geladenen Draht 
    in der Mitte bewegen und durch diesen abfließen. Es entsteht ein Ionisationsstrom. \\
    In Abhängigkeit der Spannung kommt es zu verschiedenen Reaktionen. In der 
    Abbildung sind diese in verschiedene Bereiche von (I) bis (V) unterteilt.\\
    %TODO: Abbildung zum Spannungs-Zahl-Verlauf einfügen (Anleitung Abb2)
    \\
    Wenn die Spannung klein ist, findet im Bereich (I) Rekombination statt und nur ein Teil der freien
    Elektronen gelangt zum Draht. In diesem Bereich steigt die Zahl der Elektronen-Ionen-Paare
    sehr stark an.\\
    \\
    Im Bereich (II) ist die Spannung etwas größer und damit auch die elektrische Feldstärke,
    sodass alle freien Elektronen aus der Primärionisation zum Draht gelangen. 
    Dieser Bereich wird auch als Ionisationskammer genutzt.\\
    \\
    Der Bereich (III) wird in Proportionalitätsbereich und Bereich begrenzter Proportionalität aufgeteilt.
    Die Spannung nimmt weiter zu, sodass die freien Elektronen genug Energie haben, um die Gasatome
    anzuregen. Es folgt die Stoßionisation, bei der weitere freie 
    Elektronen mit ausreichend hoher Energie freigesetzt werden, sodass eine Lawine 
    an freien Elektronen entsteht.
    Anhand der Ladungsimpulse, die beim Auftreffen der Elektronen auf dem Draht entstehen, kann
    mit der Proportionalitätsbeziehung zur Energie eine Abschätzung für die Teilchenzahl 
    getroffen werden. Je höher jedoch die Spannung steigt, desto mehr Teilchen treffen auf den Draht
    und die Proportionalität wird begrenzt. In diesem Bereich steigt die Zahl der Elektronen-Ionen-Paare
    stark an.\\
    \\
    Der eigentliche Arbeitsbereich des Geiger-Müller-Zählrohrs ist Bereich (IV). Die Spannung 
    ist jetzt so groß, dass bei der Ionisation zusätzlich zu Elektronen auch UV-Photonen
    entstehen können, welche sich unbeinflusst vom elekrischen Feld bewegen können. Es 
    entstehen Elektronenlawinen in der gesamten Zählröhre. 
    Der Ladungsimpuls, welcher am Anodendraht gemessen wird, ist nun unabhängig vom einfallenden Teilchen.
    Zu diesem Zeitpunkt bleibt die Zahl der Elektronen-Ionen-Paare nahezu konstant.
    Dieser Bereich wird Plateau genannt und wird mithilfe einer Charakteristik dargestellt,
    welche den Bereich (IV) vergrößert darstellt. \\ 
    %TODO: Abb der Charakteristik einfügen (Abb4 in Anleitung)
    \\
    Wenn die Spannung noch höhere Werte annimmt, kommt es in Bereich (V) zu einer Dauerentladung,
    welche eine hohe Stromdichte hervorruft. In diesem Bereich wird das Geiger-Müller-Zählrohr zerstört.\\
    \\
    Der mittlere Zählrohrstrom, also die Ladungsmenge, die beim Auftreffen der Teilchen auf 
    dem Draht freigesetzt wird, kann durch
    \begin{gather}
        \bar{I} \approx \frac{1}{\tau} \int_0^{\tau}{\frac{U(t)}{R}} \symup{d}t \\
        \bar{I} \approx \frac{\symup{\Delta}Q}{\symup{\Delta}t} Z
    \end{gather}
    berechnet werden.
    Dabei ist $\symup{\Delta}Q$ die Ladungsmenge, welche in einer Zeit $\symup{\Delta}t$ von Z
    Teilchen übertragen wurde.
    Für Z gilt 
    \begin{equation}
        Z = \frac{I}{\epsilon_0 N} . \label{eqn:Teilchenzahl}
    \end{equation}

\subsection{Totzeit und Nachentladung}

    Die positiven Ionen, die bei den Reaktionen des einfliegenden Teilchens mit den Gasatomen
    entstehen, fließen nicht durch den Draht ab.
    Es baut sich ein Ionenschlauch auf, also ein Gebiet positiver Raumladung, welche das
    elektrische Feld so beeinflusst, dass für eine Zeit T keine Stoßionisation stattfindet.
    Diese Zeit T wird Totzeit genannt. Während dieser Zeit wird ein eintreffendes Teilchen
    nicht regristriert. Nach der Totzeit folgt eine Erholungszeit $T_\text{E}$, in der 
    schwächere Ausgangsimpulse regristriert werden (vgl. Abb...).\\
    %TODO: Abb zur Totzeit einfügen (Abb3 in Anleitung)
    Die Totzeit kann mithilfe der Gleichung 
    \begin{equation}
        T \approx \frac{N_1 + N_2 - N_{1+2}}{2 N_1 N_2} \label{eqn:totzeit}
    \end{equation}
    berechnet werden, wobei die Näherung nur für $T^2 N_i ^2 << 1$ gilt.
    $N_1$ bezeichnet die Zählrate, die für ein erstes Präparat gemessen wird,
    $N_{1+2}$ die Zählrate, die gemessen wird, wenn ein zweites Präparat hinzugefügt wird und
    $N_2$ die Zählrate für das zweite Präparat, nachdem das erste entfernt wurde.\\
    \\
    Die positiven Ionen werden zur negativen Kathode gezogen, wodurch sich der Ionenschlauch
    auflöst und wieder Stoßionisation möglich ist.
    Wenn die Ionen auf der Kathode auftreffen, lösen sie Elektronen, sogenannte
    Sekundärelektronen, aus dem Material.
    Dieser Vorgang wird Nachentladung genannt und sorgt dafür, dass mehrere Ausgangsimpulse 
    entstehen, wenn ein Teilchen durch die Röhre geflogen ist, welche 
    im zeitlichen Abstand $T_\text{L}$ voneinander liegen. 
    Diese Zeit $T_\text{L}$ ist länger als die eigentliche Totzeit.
    Um die Nachentladung zu minimieren, werden zum Gasgemisch Alkoholdämpfe hinzugefügt, 
    welche bei Ionisation eine zu geringe Energie haben, um ihrerseits ionisieren zu können.\\
    \\
    Im Übergang von Bereich (IV) zu (V) nimmt die Zahl der Nachentladungen stark zu, 
    welche die Ursache für die Dauerentladung des Geiger-Müller-Zählrohrs sind.


    %Ansprechvermögen des Zählrohrs