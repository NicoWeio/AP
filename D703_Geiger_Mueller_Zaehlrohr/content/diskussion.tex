\section{Diskussion} \label{sec:Diskussion}

Die gemessene Kennlinie des Zählrohrs (siehe \autoref{fig:plot1}) ist deutlich unregelmäßiger als die Theoriekurve (siehe \autoref{fig:abb4}).
Entsprechend ungenau sind der abgelesene Plateaubereich und die Plateaulänge.
Der Plateauanstieg von $\SI{19.198(3727)}{{Imp}\per\second\per\kilo\volt}$ weist ebenfalls eine relativ große Unsicherheit auf.
Diese ermittelten Kennzahlen entsprechen jedoch in etwa den zu erwartenden Werten.
% https://docplayer.org/30428392-Versuch-703-das-geiger-mueller-zaehlrohr.html


Die nach \hyperref[sec:totzeit_zweiquellen]{Zwei-Quellen-Methode} und mit \hyperref[sec:totzeit_zweiquellen]{Oszilloskop} bestimmten Totzeiten
von $\SI{115(4)}{\micro\second}$ bzw. $\SI{110(20)}{\micro\second}$
stimmen (unter Berücksichtigung der ohnehin geringeren Genauigkeit bei der Bestimmung mit Oszilloskop)
sehr gut überein.


Die berechneten freigesetzten Ladungen pro einfallendem Teilchen
unterscheiden sich von der Regressionsgerade um wenig mehr als ihre Standardabweichung,
weshalb die Messung für relativ gut befunden wird.
% Die Abhängigkeit der freigesetzten Ladungen pro einfallendem Teilchen scheint im Messbereich weitestgehend proportional zur angelegten Spannung zu sein, was zu erwarten ist???
% → In welchem Bereich befinden wir uns überhaupt? Überwiegend doch im Geiger-Müller-Bereich, also im Plateau. Da soll's aber gar nicht so sehr steigen oder proportional sein…
