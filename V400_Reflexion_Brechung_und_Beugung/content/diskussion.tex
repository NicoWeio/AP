\section{Diskussion}
\label{sec:diskussion}
%NOTE: Keine Unterabschnitte hier! (wegen persönlicher Präferenz)

Das Reflexionsgesetz konnte in \autoref{sec:auswertung:reflexionsgesetz}
im Rahmen der Messgenauigkeit verifiziert werden.
Da jedoch keine Messungen mehrfach durchgeführt wurden,
kann zur Messgenauigkeit ohne die Annahme, dass das Reflexionsgesetz gilt,
keine empirische Angabe gemacht werden.
Diese Einschätzung stützt sich vielmehr auf die Beobachtung,
dass je nach Blickwinkel auf den Versuchsaufbau
die abgelesenen Werte um circa \SI{2}{\degree} variierten.
%
Die Steigung der Regressionsgeraden weicht um \SI{1.79}{\percent} vom Idealwert $a = 1$ ab.
Dass der Achsenabschnitt nicht genau bei $b = 0$ liegt,
ist auf einen systematischen Fehler zurückzuführen.
Die wahrscheinlichste Ursache ist,
dass Spiegel oder Skala nicht genau ausgerichtet waren.


In \autoref{sec:auswertung:brechungsgesetz}
wurde der Brechungsindex von Plexiglas zu \num{1.45(2)} bestimmt;
der Literaturwert lautet \num{1.489} \cite{n_acryl}.
Damit ergibt sich eine Abweichung von \num{0.04} beziehungsweise \SI{2.6}{\percent}.
% Gleiches gilt wegen der inversen Proportionalität für die Lichtgeschwindigkeit in Plexiglas.


Der \hyperref[sec:auswertung:strahlversatz]{Strahlversatz in einer planparallelen Platte}
wurde nur theoretisch bestimmt.
Die Werte in \autoref{tab:strahlversatz}
sowie der Graph in \autoref{fig:plt_strahlversatz}
erscheinen jedoch plausibel.


\autoref{sec:auswertung:prisma} widmete sich der Ablenkung von Laserlicht im Prisma.
Bei Betrachtung von \autoref{fig:prisma} wird klar,
dass $\delta < 2\gamma = \SI{120}{\degree}$ gelten muss.
Dies erfüllen die berechneten $\delta$ offensichtlich.
%
An der letzten Spalte von \autoref{tab:prisma} zeigt sich,
dass grünes Laserlicht stärker als rotes abgelenkt wird.
Wenngleich der Effekt hier noch in der Größenordnung der Messungenauigkeit liegt,
deutet sich zumindest an,
dass es sich um \hyperref[sec:theorie:dispersion]{Dispersion} handelt.


Anhand der Beugung am Gitter konnte in \autoref{sec:auswertung:beugung}
die Wellenlänge der beiden Laser genau bestimmt werden;
die relative Abweichung der Wellenlänge
des grünen Lasers beträgt \SI{1.51}{\percent},
die des roten Lasers beträgt \SI{1.23}{\percent}.


Die in mehreren Abschnitten verwendete
Näherung des Brechungsindex von Luft $n = 1.000292 \approx 1$
konnte problemlos angewandt werden,
weil die dadurch entstehende (relative) Abweichung
sehr klein im Vergleich zu den Messwerten und sonstigen Ungenauigkeiten ist –
sie wirkt sich auf die angegebenen Dezimalstellen nicht aus.
