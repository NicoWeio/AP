\section{Diskussion} \label{sec:diskussion}

\subsection{Abweichungen}

Die relative Abweichung der Messwerte wird als
\begin{equation*}
  \symup{\Delta}\chi = \frac{\chi_{R/U} - \chi_T}{\chi_T}
\end{equation*}
angegeben.
Daraus ergeben sich die Abweichungen,
die in \autoref{tab:abweichungen} aufgelistet sind.

\begin{table}
  \centering
  \caption{Abweichung der experimentell bestimmten Suszeptibilitäten von den Theoriewerten.}
  \label{tab:abweichungen}
  \begin{tabular}{c S S}
  \toprule
  Stoff &
  {$\symup{\Delta} \chi_R$ [\%]} &
  {$\symup{\Delta} \chi_U$ [\%]} \\
  \midrule
  $\symup{Dy_2O_3}$ & -11.94 & -67.25 \\
  $\symup{Nd_2O_3}$ & - 2.38 &  90.16 \\
  \bottomrule
  \end{tabular}
\end{table}

Die Abweichungen entsprechen in etwa dem,
was mit Blick auf die Messwerte zu erwarten war.
Die Spannungsdifferenzen $\symup{\Delta}U$ fielen beispielsweise in allen drei Messungen mit $\symup{Nd_2O_3}$ sehr unterschiedlich aus
und waren sehr klein gegenüber ihrem Fehler.
Entsprechend groß ist die Abweichung mit $\num{90.16}\%$.


\subsection{Mögliche Fehlerquellen}

Beim Einstellen des Abgleichwiderstands zeigte sich,
dass das verwendete Spannungsmessgerät nicht hinreichend genau abgelesen werden konnte:
Für einen gewissen Einstellbereich war auf dem Messgerät keine Änderung zu erkennen.
Es ist daher mit einer Abweichung des Widerstands von circa $\pm\SI{50}{\milli\ohm}$ zu rechnen.
Das entspricht rund $\num{50}\%$ der in \autoref{tab:messwerte_U_R} gelisteten $\symup{\Delta}R$ und kann zu entsprechend großen Fehlern führen.
% Spannung ging nicht auf 0 runter… ¯\_(ツ)_/¯


Der bestimmte reale Querschnitt $Q_\text{real}$ unterliegt einem systematischen Fehler,
da die Röhrchen nicht vollständig gefüllt waren,
aber die Länge des gesamten Teströhrchens gemessen wurde.
Daraus resultieren zu niedrige Werte.
% - Röhrchen nicht ganz gefüllt, nicht ganz in der Spule…


Die \hyperref[fig:plot_filterkurve]{Filterkurve} fiel mit einer \hyperref[misc:güte]{Güte von $8.7611 \pm 0.0164$} breit aus.
Dadurch konnten Störfrequenzen die gemessene Spannung stärker verfälschen.


Die Suszeptibilität der Proben ist bekanntlich auch von deren Temperatur abhängig.
Zwar wurde versucht, die Proben nicht durch längeren Hautkontakt zu erwärmen,
allerdings variierte möglicherweise auch die Umgebungstemperatur nicht unerheblich,
da ein nahegelegenes Fenster geöffnet war.


% - diverse Probleme im Vorhinein…
% - 100mV_eff
