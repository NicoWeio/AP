\subsection{Theoretische Bestimmung der Suszeptibilität}
\label{sec:auswertung:chi_T}

Aus den \hyperref[Hundsche_Regeln]{Hund'schen Regeln} gehen die Quantenzahlen,
also Spin $S$, Bahndrehimpuls $L$ und Gesamtdrehimpuls $J$,
hervor.
Aus diesen kann dann der jeweilige \hyperref[eqn:Lande_Faktor]{Landé-Faktor} $g_J$
und schließlich die Suszeptibilität $\chi_T$ berechnet werden.

Beispielhaft wird das Molekül $\symup{Dy_2O_3}$ betrachtet.
Dieses hat 9 4f-Elektronen \cite{versuchsanleitung}.
Die \hyperref[Hundsche_Regeln]{Hund'schen Regeln} liefern dann:
\begin{align*}
  L &= \lvert -3 -2 -1 +0 +1 \lvert = 5 \\
  S &= \frac{1}{2} \cdot 7 - \frac{1}{2} \cdot 2 = \num{2.5} \\
  J &= L - S = \frac{9}{2} = \num{4.5} \; .
\end{align*}

Die Zahl der Momente pro Volumeneinheit beschreibt
\begin{equation*}
  N = 2 \cdot \frac{\rho}{M} \cdot N_A
\end{equation*}
mit Avogadrokonstante $N_A$, Dichte $\rho$ und der molaren Masse $M$.

In \autoref{tab:Quantenzahlen_Lande} sind alle zuvor genannten Werte aufgelistet.

\begin{table}[H]
  \centering
  \caption{Quantenzahlen, Landé-Faktor und stoffliche Eigenschaften der untersuchten Seltenen Erden.}
  \label{tab:Quantenzahlen_Lande}
  \begin{tabular}{c S S S S c c S}
  \toprule
  Stoff &
  $S$ &
  $L$ &
  $J$ &
  $g_J$ &
  $\rho_w \mathbin{/} \si{\gram\per\cubic\centi\meter}$ &
  $M \mathbin{/} \si{\gram\per\mol}$ &
  $N \mathbin{/} \SI{e-28}{\per\cubic\meter}$ \\
  \midrule
  $\symup{Dy_2O_3}$ & 2.5 & 5 & 7.5 & 1.333 & \num{7.8}  \cite{versuchsanleitung} & \num{373} \cite{molmasse_Dy_2O_3} & 2.519 \\
  $\symup{Nd_2O_3}$ & 1.5 & 6 & 4.5 & 0.727 & \num{7.24} \cite{versuchsanleitung} & \num{336} \cite{molmasse_Nd_2O_3} & 2.595 \\
  \bottomrule
  \end{tabular}
\end{table}

Schließlich wird die Suszeptibilität gemäß \autoref{eqn:chi_T} bestimmt.
Die Ergebnisse finden sich in \autoref{tab:chi_T}.

\begin{table}[H]
  \centering
  \caption{Theoretisch berechnete Suszeptibilitäten der untersuchten Seltenen Erden.}
  \label{tab:chi_T}
  \begin{tabular}{c S}
  \toprule
  Stoff &
  $\chi_T$ \\
  \midrule
  $\symup{Dy_2O_3}$ & 0.0254 \\
  % 0.0126 rkallo
  % 0.025  aknierim
  % 0.026  Mampfzwerg
  % 0.256  mwindau (in sich inkonsistent)
  $\symup{Nd_2O_3}$ & 0.0030 \\
  % 0.003 aknierim
  % 0.003 Mampfzwerg
  % 0.003 mwindau
  % 0.0039 rkallo
  \bottomrule
  \end{tabular}
\end{table}
