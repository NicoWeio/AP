\section{Durchführung}

    In diesem Abschnitt soll die Durchführung der Messung der paramagnetischen Suszeptibilität
    von Proben Seltender Erde, sowie die Bestimmung der Filterkurve erläutert werden.

\subsection{Die Filterkurve des Selektivverstärkers}

    Zur Bestimmung der Filterkurve wird die Ausgansspannung $U_\text{A}$ gemessen, in Abhängigkeit der Frequenz.
    Dazu wird eine konstante Eingangsspannung $U_\text{E}$ von etwa $\SI{100}{\milli\volt_\text{eff}}$ 
    an den Selektivverstärker angelegt, welcher mit dem Spannungsmessgerät für $U_\text{A}$ verbunden ist.
    Anschließend wird ein Frequenzbereich von $\SIrange{20}{40}{\kilo\hertz}$ durchlaufen,
    und das Maximum von $U_\text{A}$ bestimmt.
    Die Frequenz, bei der $U_\text{A}$ maximal wird, 
    ist $\nu_0$.

\subsection{Messung der Proben}

    Der Aufbau der Apparatur zur Messung der Suszeptibilitäten wird in Abbildung %\ref{fig:Aufbau}
    dargestellt.\\
    %Abbildung der Messapparatur einfügen (Abb.4 in der Anleitung) \label{fig:Aufbau}
    
    Die Eingangsspannung $U_\text{E}$ (Wechselspannung) wird von einem Sinusgenerator erzeugt,
    mit einer Frequenz von $\SI{35}{\kilo\hertz}$,
    welche entsprechend als Durchlassfrequenz des Selektivverstärkers eingestellt wird.
    Der Sinusgenerator wird an die Brückenspannung %\ref{fig:Brückenschaltung}
    angeschlossen, 
    welche mit dem Selektivverstärker verbunden ist.
    Die in der Abbildung %\ref{fig:Aufbau}
    gezeigten Verstärker vor und hinter dem Selektivverstärker werden hier nicht verwendet.
    Die Brückenspannung wird allerdings am Selektivverstärker 10-fach verstärkt.
    Der Selektivverstärker ist wiederum mit einem Spannungsmessgerät verbunden, 
    welches die Brückenspannung $U_\text{Br}$ misst.\\
    \\
    Zu Beginn wird die Gesamtverstärkung gemessen,
    indem die maximale Brückenspannung mit und ohne Verstärkung gemessen wird.\\ 
    Bei der Ausmessung der Proben wird nun so vorgegangen,
    dass zu Anfang die Brücke ohne eingefügte Probe abgeglichen wird.
    Dazu wird der zu variierende Widerstand $R_3/R_4$ verstellt, 
    sodass $U_\text{Br}$ minimal wird.
    Es werden $U_\text{Br}$ und $R_3/R_4$ notiert.
    Anschließend wird die erste Probe,
    in diesem Fall $\ce{^{203}Dy}$,
    vorsichtig in eine der zylinderförmigen Spulen in der Brückenschaltung eingefügt.
    Die Brücke wird erneut abgeglichen und es werden $U_\text{Br}$ und $R_3/R_4$ notiert.
    Mithilfe der Differenzen zwischen den Spannungen und Widerständen mit und ohne Probe 
    kann mithilfe der Gleichungen \eqref{eqn:SusU} und \eqref{eqn:SusR} nun 
    die paramagnetische Suszeptibilität der $\ce{^{203}Dy}$-Probe berechnet werden.
    Die gesamte Messung für $\ce{^{203}Dy}$ wird dreimal ausgeführt.\\
    Das Verfahren für die zweite Probe $\ce{^{203}Nd}$ ist analog,
    auch diese Messung wird dreimal wiederholt.\\
    %\\
    Anschließend wird die Länge $L$ der beiden Proben jeweils dreimal ausgemessen.
    um mithilfe der Gleichung
    \begin{equation}
        Q_\text{Real} = \frac{M_\text{p}}{L \rho_\text{w}}
    \end{equation}
    den realen Querschnitt der Probe zu berechnen mit der Dichte $\rho_\text{w}$ und $M_\text{p}$.
    