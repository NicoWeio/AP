\section{Auswertung} \label{sec:auswertung}

\subsection{Die statische Methode}

    Für die Temperaturen T1 und T4 ergaben sich die Messwerte
    \begin{figure}
        \centering
        \includegraphics{Graph_T1T4.pdf}
        \caption{Die Temperaturen des breiten und schmalen Messingstabs.}
        \label{fig:grapht1t4}
    \end{figure}
    %TODO: Abbildung zu groß

    Die Temperaturen T1 und T4 steigen zu Beginn, nach einer kurzen Zeit, stark an.
    Ab etwa $\SI{250}{\second}$ wachsen die Temperaturen weniger schnell an.\\
    
    Für die Temperaturen T5 und T8 ergaben sich die Messwerte 
    \begin{figure}
        \centering
        \includegraphics{Graph_T5T8.pdf}
        \caption{Die Temperaturen des Aluminiumstabes und des Edelstahlstabes.}
        \label{fig:grapht5t8}
    \end{figure}\\
    %TODO: Abbildung ist nicht an der richtigen Stelle und zu groß

    Die Temperatur T5 des Aluminiumstabes steigt zu Beginn der Messung 
    sehr stark an und flacht ab etwa $\SI{100}{\second}$ ab.
    Im Gegensatz dazu steht die Temperatur T8 des Edelstahlstabes.
    Die Temperatur steigt langsamer und gleichmäßiger an und erreicht nur
    einen geringeren Wert von etwa $\SI{37}{\celsius}$.\\
    Im Vergleich der Metallstäbe erreicht der Aluminiumstab die höchste
    Temperatur von etwa $\SI{47}{\celsius}$. Zum selben Zeitpunkt haben
    die Messingstäbe eine Temperatur von etwa $\SI{42}{\celsius}$ bis $\SI{45}{\celsius}$,
    wobei der breitere Messingstab eine höhere Temperatur erreicht.
    Die Temperatur des Aluminiumstabes steigt zu Beginn auch schneller an als die
    der Messingstäbe.\\

    Um herauszufinden, welcher Stab die beste Wärmeleitung hat, wurden nach $\SI{700}{\second}$
    folgende Temperaturen gemessen.

    \begin{table}
        \centering
        \caption{Temperaturen T1, T4, T5, T8 in $\si{\celsius}$ nach $\SI{700}{\second}$.} %Einheit hier?
        \label{tab:temp700s}
        \begin{tabular}{c c c c}
            \toprule
            T1 & T4 & T5 & T8 \\
            \midrule
            42.36 & 40.81 & 45.72 & 32.73 \\ % 700s/5 = 140: Messwerte bei 140
            \bottomrule  
        \end{tabular}
    \end{table}
    Da die Temperatur T5 des Aluminiumstabes mit $\SI{45.72}{\celsius}$ am größten ist,
    kann daraus geschlossen werden, dass Aluminium die Wärme am besten leitet.

    %TODO: Wärmestrom für 5 Messzeiten berechnen.

    Für die Temperaturdifferenzen $\symup{\Delta}T_\text{St} = T7 - T8$ und 
    $T_\text{St} = T2 - T1$ ergeben sich folgende Graphen.
    \begin{figure}
        \centering
        \includegraphics{Graph_Tempdiff.pdf}
        \caption{Die Temperaturdifferenzen $\symup{\Delta}T_\text{St}$ und $T_\text{St}$ in $\si{\celsius}$.}
        \label{fig:graphtempdiff}
    \end{figure}
    Im Vergleich ist zu erkennen, dass die Temperaturdifferenz zwischen T7 und T8 während der gesamten
    Messung deutlich größer ist als die zwischen T1 und T2.
    In beiden Graphen ist die Differenz zu Beginn der Messung größer und hat nach etwa %25-30s?
    ihr Maximum erreicht, wobei dies bei $T_\text{St}$ etwas eher der Fall ist.
    Nach dem Maximum flachen beide Graphen ab und laufen gegen einen konstanten Wert.
    Die Abflachung bei $\symup{\Delta}T_\text{St}$ ist nicht so steil, während die Temperaturdifferenz
    $T_\text{St}$ scheinbar schnell gegen einen konstanten Wert strebt. %Ich bin mit der Formulierung noch nicht zufrieden.


    \subsection{Die dynamische Methode}