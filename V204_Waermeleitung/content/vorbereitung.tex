\section{Vorbereitung} \label{sec:vorbereitung}

    Zur Vorbereitung auf den Versuch sollte für die Metalle
    Aluminium, Edelstahl und Messing und für Wasser die Dichte $\rho$,
    die spezifische Wärmekapazität $c$ und die Wärmeleitfähigkeit $\kappa$ 
    recherchiert werden.

    %TODO Einheiten angeleichen und in die Tabelle einfügen!
    %TODO Quellen
   \begin{table}
        \centering
        \caption{Größen aus der Literatur.}
        \label{tab:daten_vorbereitung}
        \begin{tabular}{c c c c c}
         \toprule
         $Stoff$ & $\rho$ & $c$ & $\kappa$ \\
         \midrule
         %Wasser & 997 & 4.183 & 0.6 \\ %rho in kg/m³, c in kJ/kgK, kappa in W/mK ; ist nicht benötigt
         Aluminium & 2.7 & 0.888 & 237 \\ %rho in g/m³, c und kappa gleich
         Edelstahl & 7.9 & 0.477 & 15 \\ %rho in kg/dm³, c in J/gK, kappa gleich, kappa variiert zw. 15u21
         Messing & 8.73 & 377 & 120 \\ %rho in g/cm³, c in J/kgK, kappa gleich
         \bottomrule
        \end{tabular}
   \end{table}