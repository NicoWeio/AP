\section{Vorbereitung}
\label{sec:vorbereitung}

Zur Vorbereitung auf den Versuch sollte für Wasser und die Metalle
Aluminium, Edelstahl und Messing die Dichte $\rho$,
die spezifische Wärmekapazität $c$ und die Wärmeleitfähigkeit $\kappa$ recherchiert werden.

\begin{table}
    \centering
    \caption{Größen aus der Literatur.}
    \label{tab:daten_vorbereitung}
    \begin{tabular}{c c c c}
     \toprule
     Stoff &
     $\rho \mathbin{/} \si{\kilo\gram\per\cubic\meter}$ &
     $c \mathbin{/} \si{\joule\per\kilo\gram\per\kelvin}$ &
     $\kappa \mathbin{/} \si{\watt\per\meter\per\kelvin}$ \\
     \midrule
     Wasser    & 999  \cite{chemie_de_4} & 4183 \cite{chemie_de_4} & 0.6 \cite{chemie_de_2} \\
     Aluminium & 2700 \cite{chemie_de_1} & 888  \cite{chemie_de_1} & 237 \cite{chemie_de_1} \\
     Edelstahl & 7800 \cite{chemie_de_1} & 500  \cite{chemie_de_1} & 15  \cite{chemie_de_1} \\ % → Cr-Ni-Stahl
     Messing   & 8400 \cite{chemie_de_3} & 377  \cite{chemie_de_3} & 120 \cite{chemie_de_2} \\
     \bottomrule
    \end{tabular}
\end{table}
