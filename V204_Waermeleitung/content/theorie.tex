\section{Theorie} \label{sec:theorie}

    Der Wärmetransport in einem Körper kann durch Wärmestrahlung, Wärmeleitung oder Konvektion
    stattfinden.
    Im Folgenden sollen die theoretischen Grundlagen der Wärmeleitung erklärt werden.


    Die Wärmemenge $\symup{d}Q$, die in einer bestimmten Zeit $\symup{d}t$ durch einen Stab mit der
    Querschnittsfläche $A$ und der Dichte $\rho$ fließt, wird mithilfe der Gleichung
    \begin{equation}
      \label{eqn:waermemenge_pro_zeit}
      \symup{d}Q =  A j_\text{w} \symup{d}t
    \end{equation}
    beschrieben.
    Die Dichte $j_\text{w}$ des Wärmestromes, der beim Wärmetransport entsteht, ist durch
    \begin{equation}
      \label{eqn:waermestromdichte}
        j_\text{w} = - \kappa \frac{\partial T}{\partial x}
    \end{equation}
    gegeben.
    Das Vorzeichen entsteht durch die Richtung des Wärmeflusses.
    Die Wärme fließt vom wärmeren Teil des Stabes zum kälteren Teil.


    Für den Prozess der Wärmeleitung ergibt sich die eindimensionale Wärmeleitungsgleichung
    \begin{equation*}
        \frac{\partial T}{\partial t} = \frac{\kappa}{\rho c} \frac{\partial^2 T}{\partial x^2} \; .
    \end{equation*}
    Der Vorfaktor $\frac{\kappa}{\rho c}$ kann durch $\sigma_\text{T}$ ausgedrückt werden
    und bezeichnet die Temperaturleitfähigkeit.
    Der Faktor $\kappa$ gibt die spezifische Wärmeleitfähigkeit eines Stoffes an. \\


    Wenn ein hinreichend langer Stab $T$-periodisch erwärmt und gekühlt wird,
    entsteht eine Wärmewelle mit der Periode $T$.
    Die Welle kann durch
    \begin{equation*}
        T(x,t) = T_\text{max} \cdot \exp{\left( - \sqrt{\frac{\omega \rho c}{2\kappa}} x \right)}
        \cdot \cos \left( \omega t - \sqrt{\frac{\omega \rho c}{2 \kappa}} x \right)
    \end{equation*}
    beschrieben werden mit der Phasengeschwindigkeit
    \begin{equation}
      \label{eqn:phasengeschwindigkeit}
        v = \frac{\omega}{k} = \sqrt{\frac{2 \kappa \omega}{\rho c}} \; .
    \end{equation}
    Für die Wärmeleitfähigkeit des Stabes ergibt sich
    \begin{equation}
      \label{eqn:waermeleitfaehigkeit}
        \kappa = \frac{\rho c (\symup{\Delta}x)^2}{2 \symup{\Delta}t \ln \left( \frac{A_\text{nah}}{A_\text{fern}} \right)}
    \end{equation}
    mit dem Abstand $\symup{\Delta}x$ zwischen zwei Messpunkten und den Amplituden $A$.
    Der Phasenunterschied der Wärmewelle zwischen den Messpunkten wird durch $\symup{\Delta} t$ beschrieben.

    Die Frequenz der Welle lässt sich direkt als Kehrwert der Periodendauer, also
    \begin{equation}
      \label{eqn:frequenz}
      f = \frac{1}{T}
    \end{equation}
    schreiben.

    Aus Phasengeschwindigkeit und Periodendauer folgt sofort die Wellenlänge der Wärmewellen:
    \begin{equation}
      \label{eqn:wellenlaenge}
      \lambda = v T \; .
    \end{equation}
