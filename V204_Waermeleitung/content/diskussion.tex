\section{Diskussion} \label{sec:diskussion}

Insgesamt kann der Versuch als erfolgreich bewertet werden.

$\kappa_\text{Aluminium} = \SI{248.228 \pm 5.176}{\watt\per\meter\per\kelvin}$ liegt etwas über dem Literaturwert von $\SI{237}{\watt\per\meter\per\kelvin}$.
Obwohl dieser Wert die größte absolute und relative Messunsicherheit aufweist,
ist der Literaturwert nicht im Konfidenzintervall enthalten.

Ähnlich ist es bei $\kappa_\text{Messing, breit} = \SI{109.082 \pm 2.307}{\watt\per\meter\per\kelvin}$,
das vom Literaturwert $\SI{120}{\watt\per\meter\per\kelvin}$ so abweicht,
dass dieser nicht mehr im Konfidenzintervall enthalten ist.

Die experimentell bestimmte Wärmeleitfähigkeit von Edelstahl $\SI{15.109 \pm 0.199}{\watt\per\meter\per\kelvin}$
stimmt dagegen sehr gut mit dem Theoriewert von $\SI{15}{\watt\per\meter\per\kelvin}$ überein.

Relativ betrachtet liegen jedoch alle Unsicherheiten unter $3 \%$.


Es gibt eine Vielzahl von möglichen Gründen für die zuvor diskutierten Abweichungen.
Beispielsweise unterliegt die Periodendauer einer kleinen Unsicherheit,
da das Umschalten zwischen Heizen und Kühlen manuell durchgeführt wurde.
Wenngleich die Stäbe während der Messung mit einem Isolator abgedeckt wurden,
muss von einem Wärmeverlust an die Umgebung ausgegangen werden,
welcher insbesondere bei der statischen Methode zum Tragen käme.
Auch ist nicht bekannt, wie genau die Thermoelemente messen.

Auf der anderen Seite kann auch die tatsächliche Wärmeleitfähigkeit der Stäbe von den Literaturwerten abweichen.
Insbesondere Messing weist als Legierung unterschiedliche Zusammensetzungen auf,
die auch zu unterschiedlichen Wärmeleitfähigkeiten führen.
Aber auch die Materialien der anderen Stäbe könnten Verunreinigungen enthalten,
die sich auf die Wärmeleitfähigkeit auswirken.
