\section{Vorbereitung}
\label{sec:vorbereitung}

    In Aufgabe 1 sollte recherchiert werden,
    bei welchen Energien sich die $\ce{Cu} K_\alpha$- und $\ce{Cu} K_\beta \,$- Linien ergeben
    und bei welchen Glanzwinkeln $\theta$ diese bei einem LiF-Kristall mit $d = \SI{201.4}{\pico\meter}$ liegen.\\
    \\
    Mit den Wellenlängen der $\ce{Cu} K_\alpha$- und $\ce{Cu} K_\beta \,$- Linien können die Energien mithilfe der Gleichung
    \begin{equation}
        \label{eqn:lambda_to_E}
        E = h \cdot \frac{c}{\lambda}
    \end{equation}
    mit dem Planck'schen Wirkungsquantum ($h = \SI{6.626e-34}{\joule\second}$) und der Lichtgeschwindigkeit im Vakuum $c$ berechnet werden.
    Für die Berechnung der Glanzwinkel $\theta_n$
    mit der Beugungsordnung $n$ (hier $n=1$ und $n=2$)
    wird die \autoref{eqn:BraggBedingung} aus \autoref{sec:theorie} nach $\theta$ umgeformt.
    Es ergeben sich folgende Werte.

    \begin{table}
        \centering
        \caption{Wellenlängen, Energien und Glanzwinkel der $\ce{Cu} K_\alpha$- und $\ce{Cu} K_\beta \,$- Linie.}
        \label{tab:Vorbereitung1}
        \begin{tabular}{c S[table-format=1.3] S[table-format=1.5] S[table-format=2.3] S[table-format=2.3]}
            \toprule
            &
            {$\lambda [\si{\angstrom}]$} &
            {$E [\si{\kilo\electronvolt}]$} &
            {$\theta_1 [\si{\degree}]$} &
            {$\theta_2 [\si{\degree}]$} \\
            \midrule
            $\ce{Cu} K_\alpha$ & 1.530 & 8.10996 & 22.323 & 49.436 \\
            $\ce{Cu} K_\beta$  & 1.392 & 8.91396 & 20.217 & 43.722 \\
            \bottomrule
        \end{tabular}
    \end{table}


    In Aufgabe 2 sollten Ordnungszahl,
    Energie $E^\text{Lit}_\text{K}$ der $K$-Kante,
    der entsprechende Glanzwinkel $\theta^\text{Lit}_\text{K}$ der zu untersuchenden Materialen
    Zink, Germanium, Brom, Rubidium, Strontium und Zirkonium recherchiert werden.
    Mithilfe von \autoref{eqn:SigmaK} aus \autoref{sec:theorie} können die folgenden Werte für die Abschirmkonstante $\sigma_\text{K}$ berechnet werden.

    \begin{table}
        \centering
        \caption{Größen der zu untersuchenden Materialen.}
        \label{tab:Vorbereitung2}
        \begin{tabular}{c c S S S}
            \toprule
            &
            {Z} &
            {$E^\text{Lit}_\text{K} [\si{\kilo\electronvolt}]$} &
            {$\theta^\text{Lit}_\text{K} [\si{\degree}]$} &
            {$\sigma_\text{K}$} \\
            \midrule
            $\ce{Zn}$ & 30 &  9.65 & 18.60 & 3.650 \\
            $\ce{Ge}$ & 32 & 11.10 & 16.11 & 3.431 \\
            $\ce{Br}$ & 35 & 13.47 & 13.22 & 3.528 \\
            $\ce{Rb}$ & 37 & 15.20 & 11.69 & 3.569 \\
            $\ce{Sr}$ & 38 & 16.10 & 11.03 & 3.593 \\
            $\ce{Zr}$ & 40 & 17.99 & 09.85 & 3.630 \\
            \bottomrule
        \end{tabular}
    \end{table}
