\section{Vorbereitung}
\label{sec:vorbereitung}

    In Aufgabe 1 sollte recherchiert werden,
    bei welchen Energien sich die $\ce{Cu} K_\alpha$- und $\ce{Cu} K_\beta$- Linien ergeben und 
    bei welchen Glanzwinkeln $\theta$ diese bei einem LiF-Kristall mit $d = \SI{201.4}{\pico\meter}$ liegen.\\
    \\
    Mit den Wellenlängen der $\ce{Cu} K_\alpha$- und $\ce{Cu} K_\beta$- Linien können die Energien mithilfe der Gleichung
    \begin{equation}
        E = h \cdot \frac{c}{\lambda}
    \end{equation}
    mit dem Planck'schen Wirkungsquantum $\SI{6.626e-34}{\joule\second}$ und der Lichtgeschwindigkeit $c$ im Vakuum berechnet werden.
    Für die Berechnung der Glanzwinkel $\theta_n$,
    mit der Beugungsordnung $n$ (hier $n=1$ und $n=2$),
    wird die Gleichung \eqref{eqn:BraggBedingung} aus Kapitel \ref{sec:theorie} nach $\theta$ umgeformt.
    Es ergeben sich folgende Werte.

    \begin{table}
        \centering
        \caption{Wellenlängen, Energien und Glanzwinkel der $\ce{Cu} K_\alpha$- und $\ce{Cu} K_\beta$- Linie.}
        \label{tab:Vorbereitung1}
        \begin{tabular}{c c c c c}
            \toprule
            & {$\lambda$} & {$E$} & {$\theta_1$} & {$\theta_2$}\\
            \midrule
            $\ce{Cu} K_\alpha$ & $\SI{1.530}{\angstrom}$ & $\SI{8.10996}{\kilo\electronvolt}$ & $\SI{22.323}{\degree}$ & $\SI{49.436}{\degree}$\\
            $\ce{Cu} K_\beta$ & $\SI{1.392}{\angstrom}$ & $\SI{8.91396}{\kilo\electronvolt}$ & $\SI{20.217}{\degree}$ & $\SI{43.722}{\degree}$\\
            \bottomrule
        \end{tabular}
    \end{table}
    
    In Aufgabe 2 sollten Ordnungszahl, 
    Energie $E^\text{Lit}_\text{K}$ der $K$-Kante, 
    der entsprechende Glanzwinkel $\theta^\text{Lit}_\text{K}$ der zu untersuchenden Materialen Zink, 
    Germanium, Brom, Rubidium, Strontium und Zirkonium recherchiert werden.
    Mithilfe von Gleichung \eqref{eqn:SigmaK} aus Kapitel \ref{sec:theorie} können folgende Werte für die Abschirmkonstante $\sigma_\text{K}$ berechnet werden.

    \begin{table}
        \centering
        \caption{Größen der zu untersuchenden Materialen.}
        \begin{tabular}{c c c c c}
            \toprule
            & {Z} & {$E^\text{Lit}_\text{K}$} & {$\theta^\text{Lit}_\text{K}$} & {$\sigma_\text{K}$}\\
            \midrule
            $\ce{Zn}$ & 30 & $\SI{09.65}{\kilo\electronvolt}$ & $\SI{18.6}{\degree}$ & 3.65 \\
            $\ce{Ge}$ & 32 & $\SI{11.10}{\kilo\electronvolt}$ & $\SI{16.11}{\degree}$ & 3.431 \\
            $\ce{Br}$ & 35 & $\SI{13.47}{\kilo\electronvolt}$ & $\SI{13.22}{\degree}$ & 3.528 \\
            $\ce{Rb}$ & 37 & $\SI{15.20}{\kilo\electronvolt}$ & $\SI{11.69}{\degree}$ & 3.569 \\
            $\ce{Sr}$ & 38 & $\SI{16.10}{\kilo\electronvolt}$ & $\SI{11.03}{\degree}$ & 3.593 \\
            $\ce{Zr}$ & 40 & $\SI{17.99}{\kilo\electronvolt}$ & $\SI{09.85}{\degree}$ & 3.630 \\
            \bottomrule
        \end{tabular}
    \end{table}

    
    

    