\section{Diskussion}

\subsection{Abweichungen}

In \autoref{sec:auswertung:bragg} wurde bereits erwähnt,
dass die Abweichung des gemessenen Bragg-Winkels vom Erwartungswert
um $\SI{0.2}{\degree}$ abweicht.
% Das ist ein guter Wert…
% Wie ändert sich denn nun XY, wenn es 3° wären? → Frage aus der Versuchsanleitung


\autoref{sec:auswertung:emissionsspektrum} widmete sich der Analyse des Emissionsspektrums der Kupferanode.
Die dort bestimmten $\theta_{K_\alpha}$ und $\theta_{K_\beta}$ weichen nur um
$\SI{0.79}{\percent}$ bzw. $\SI{0.08}{\percent}$ von den in \autoref{tab:Vorbereitung1} angegebenen Werten ab.


\autoref{tab:diskussion:abschirmkonstanten} listet die in \autoref{sec:auswertung:absorptionsspektren} bestimmten Abschirmkonstanten
zusammen mit ihrem Literaturwert (siehe \autoref{tab:Vorbereitung2}) und der relativen Abweichung auf.

\begin{table}
    \centering
    \caption{Vergleich der berechneten Abschirmkonstanten mit den Literaturwerten.}
    \label{tab:diskussion:abschirmkonstanten}
    \begin{tabular}{l S[table-format=1.2] S[table-format=1.2] S[table-format=1.2]}
    \toprule
    Absorber &
    {$\sigma_K$} &
    {$\sigma_{K\text{, Lit}}$} &
    {Abweichung $[\si{\percent}]$} \\
    \midrule
    Zink      & 3.62 & 3.57 & 1.32 \\
    Gallium   & 3.68 & 3.62 & 1.64 \\
    Brom      & 3.84 & 3.85 & 0.29 \\
    Rubidium  & 4.08 & 3.95 & 3.16 \\
    Strontium & 4.12 & 4.01 & 2.78 \\
    Zirkonium & 4.31 & 4.11 & 4.79 \\
    \bottomrule
    \end{tabular}
\end{table}

Es ergeben sich kleine Abweichungen von $< \SI{5}{\percent}$,
welche aber bis auf Brom durchweg durch zu große $\sigma_K$ zustande kommen.


Die in \autoref{sec:auswertung:moseley} aus den verschiedenen Abschirmenergien berechnete
Rydbergfrequenz (bzw. Rydbergenergie)
weicht um $\SI{7.97}{\percent}$ vom Literaturwert
$\SI{13.61}{\electronvolt}$ \cite{rydhcev} (bzw. $\SI{3.29}{\peta\hertz}$ \cite{rydchz})
ab.


\subsection{Mögliche Fehlerquellen}

Zunächst sei darauf hingewiesen,
dass dieser Versuch nicht von den Autoren durchgeführt wurde,
sondern auf Basis vorgegebener Daten erfolgte.
Daher können zur tatsächlichen Durchführung nur Mutmaßungen angestellt werden.


In \autoref{sec:auswertung:absorptionsspektren} wurden die Absorptionskanten abgelesen.
Aufgrund der relativ geringen Winkelauflösung und der Schwankungen zwischen Messwerten
gab es dabei jedoch einen gewissen Spielraum.
Mit Blick auf die stets zu großen Messwerte kann vermutet werden,
dass die Winkelbereiche \enquote{zu weit rechts} geschätzt wurden.
Auch ein systematischer Fehler ist nicht auszuschließen.

Eine längere Integrationszeit könnte die Genauigkeit weiter erhöhen.
