\documentclass[
  bibliography=totoc,     % Literatur im Inhaltsverzeichnis
  captions=tableheading,  % Tabellenüberschriften
  titlepage=firstiscover, % Titelseite ist Deckblatt
]{scrartcl}

% Paket float verbessern
\usepackage{scrhack}

% Warnung, falls nochmal kompiliert werden muss
\usepackage[aux]{rerunfilecheck}

% unverzichtbare Mathe-Befehle
\usepackage{amsmath}
% viele Mathe-Symbole
\usepackage{amssymb}
% Erweiterungen für amsmath
\usepackage{mathtools}

% Fonteinstellungen
\usepackage{fontspec}
% Latin Modern Fonts werden automatisch geladen
% Alternativ zum Beispiel:
%\setromanfont{Libertinus Serif}
%\setsansfont{Libertinus Sans}
%\setmonofont{Libertinus Mono}

% Wenn man andere Schriftarten gesetzt hat,
% sollte man das Seiten-Layout neu berechnen lassen
\recalctypearea{}

% deutsche Spracheinstellungen
\usepackage[ngerman]{babel}


\usepackage[
  math-style=ISO,    % ┐
  bold-style=ISO,    % │
  sans-style=italic, % │ ISO-Standard folgen
  nabla=upright,     % │
  partial=upright,   % ┘
  warnings-off={           % ┐
    mathtools-colon,       % │ unnötige Warnungen ausschalten
    mathtools-overbracket, % │
  },                       % ┘
]{unicode-math}

% traditionelle Fonts für Mathematik
\setmathfont{Latin Modern Math}
% Alternativ zum Beispiel:
%\setmathfont{Libertinus Math}

\setmathfont{XITS Math}[range={scr, bfscr}]
\setmathfont{XITS Math}[range={cal, bfcal}, StylisticSet=1]

% Zahlen und Einheiten
\usepackage[
  locale=DE,                   % deutsche Einstellungen
  separate-uncertainty=true,   % immer Unsicherheit mit \pm
  per-mode=symbol-or-fraction, % / in inline math, fraction in display math
]{siunitx}

% chemische Formeln
\usepackage[
  version=4,
  math-greek=default, % ┐ mit unicode-math zusammenarbeiten
  text-greek=default, % ┘
]{mhchem}

% richtige Anführungszeichen
\usepackage[autostyle]{csquotes}

% schöne Brüche im Text
\usepackage{xfrac}

% Standardplatzierung für Floats einstellen
\usepackage{float}
\floatplacement{figure}{htbp}
\floatplacement{table}{htbp}

% Floats innerhalb einer Section halten
\usepackage[
  section, % Floats innerhalb der Section halten
  below,   % unterhalb der Section aber auf der selben Seite ist ok
]{placeins}

% Seite drehen für breite Tabellen: landscape Umgebung
\usepackage{pdflscape}

% Captions schöner machen.
\usepackage[
  labelfont=bf,        % Tabelle x: Abbildung y: ist jetzt fett
  font=small,          % Schrift etwas kleiner als Dokument
  width=0.9\textwidth, % maximale Breite einer Caption schmaler
]{caption}
% subfigure, subtable, subref
\usepackage{subcaption}

% Grafiken können eingebunden werden
\usepackage{graphicx}

% schöne Tabellen
\usepackage{booktabs}

% Verbesserungen am Schriftbild
\usepackage{microtype}

% Literaturverzeichnis
\usepackage[
  backend=biber,
]{biblatex}
% Quellendatenbank
\addbibresource{lit.bib}
\addbibresource{programme.bib}

% Hyperlinks im Dokument
\usepackage[
  german,
  unicode,        % Unicode in PDF-Attributen erlauben
  pdfusetitle,    % Titel, Autoren und Datum als PDF-Attribute
  pdfcreator={},  % ┐ PDF-Attribute säubern
  pdfproducer={}, % ┘
]{hyperref}
% erweiterte Bookmarks im PDF
\usepackage{bookmark}

% Trennung von Wörtern mit Strichen
\usepackage[shortcuts]{extdash}

\author{%
  Katharina Popp\\%
  \href{mailto:katharina.popp@tu-dortmund.de}{katharina.popp@tu-dortmund.de}%
  \and%
  Nicolai Weitkemper\\%
  \href{mailto:nicolai.weitkemper@tu-dortmund.de}{nicolai.weitkemper@tu-dortmund.de}%
}
\publishers{TU Dortmund – Fakultät Physik}


\subject{D206}
\title{Die Wärmepumpe}
\date{
    Durchführung: 28.10.2020
    \hspace{3em}
    Abgabe: 04.11.2020
}

\begin{document}

\maketitle
\thispagestyle{empty}
\tableofcontents
\newpage

\section{Theorie} \label{sec:Theorie}

\subsection{Grundlegendes Prinzip der Wärmepumpe}

    Die Wärmepumpe wird dazu verwendet, die Richtung des Wärmeflusses umzukehren.
    Der \enquote{natürliche} Wärmefluss findet in einem abgeschlossenen System von einem heißeren Körper
    zu einem kälteren Körper statt.

    Um die Richtung des Wärmeflusses umzukehren, muss (hier: mechanische) Arbeit aufgewendet werden.
    Nach dem ersten Hauptsatz der Thermodynamik ist dabei die abgegebene Temperatur $Q_1$ vom heißeren Körper
    gleich der aufgenommenen Wärme $Q_2$ des kälteren Körpers und der benötigten Arbeit A.
    Es gilt also:
    \begin{equation}
        Q_1 = Q_2 + A .
    \end{equation}
    Mit der Beziehung zwischen Wärme und Temperatur im realistischen, irreversiblen Fall
    \begin{equation}
        \frac{Q_1}{T_1} - \frac{Q_2}{T_2} > 0
    \end{equation}
    ergibt sich außerdem:
    \begin{equation}
        Q_1 = A + \frac{T_1}{T_2} Q_2
    \end{equation}
    Das Verhältnis zwischen der abgegeben Wärme $Q_1$ und der benötigten mechanischen Arbeit A wird durch die Güteziffer $\nu$
    beschrieben.
    \begin{equation}
        \nu = \frac{Q_1}{A}
    \end{equation}

\subsubsection{Bestimmung der realen Güteziffer}

    Die mechanische Arbeit kann auch durch ein Zeitintervall und die Leistungsaufnahme N des Kompressors ausgedrückt werden:
    \begin{equation}
        \nu = \frac{\Delta Q_1}{\Delta t N}
    \end{equation}
    $\frac{\Delta Q_1}{\Delta t}$ beschreibt hier die pro Zeiteinheit gewonnene Wärmemenge. Es gilt:
    \begin{equation}
        \frac{\Delta Q_1}{\Delta t} = (m_1 c_\text{w} + m_\text{k} c_\text{k}) \frac{\Delta T_1}{\Delta t}
    \end{equation}

\subsubsection{Bestimmung des Massenzusatzes}

    Die aus dem Reservoir 2 entnommene Wärmemenge wird durch
    \begin{equation}
        \frac{\Delta Q_2}{\Delta t} =(m_2 c_\text{w} + m_\text{k} c_\text{k}) \frac{\Delta T_2}{\Delta t}
    \end{equation}
    beschrieben.
    Wenn die Verdampfungswärme L bekannt ist, lässt sich der Massenzusatz aus
    \begin{equation}
        \frac{Q_2}{\Delta t} = L \frac{\Delta m}{\Delta t}
    \end{equation}
    bestimmen.

\subsubsection{Bestimmung der mechanischen Kompressorleistung $N_\text{mech}$}

    Die Arbeit $A_\text{m}$, die benötigt wird, um das Gasvolumen $V_\text{a}$ auf den Wert $V_\text{b}$ zu verringern, wird durch
    \begin{equation}
        A_\text{m} = - \displaystyle\int_{V_\text{a}}^{V_\text{b}} p dV
    \end{equation}
    bestimmt.
    Zudem ergibt sich die Poisson´sche Gleichung als Zusammenhang zwischen Druck und Voumen
    \begin{equation}
        p_\text{a} V_\text{a}^\text{\kappa} = p_\text{b} V_\text{b}^\text{\kappa} = p V^\text{\kappa} .
    \end{equation}
    Für $A_\text{m}$ ergibt sich damit
    \begin{equation}
       A_\text{m} = - p_\text{a} V_\text{a}^\text{\kappa} \displaystyle\int_{V_\text{a}}^{V_\text{b}} V^{-\kappa} dV
                  = \frac{1}{\kappa - 1} p_\text{a} V_\text{a}^\text{\kappa} (V_\text{b}^{-\kappa+1} - V_\text{a}^{-\kappa+1})
                  = \frac{1}{\kappa - 1} (p_\text{b} \sqrt[\kappa]{\frac{p_\text{a}}{p_\text{b}}} - p_\text{a}) V_\text{a} .
    \end{equation}
    Außerdem gilt für die mechanische Kompressorleistung mit der Beziehung $V_\text{a} = \frac{1}{\rho} m$
    \begin{equation}
        N_\text{mech} = \frac{\Delta A_\text{m}}{\Delta t} 
                      = \frac{1}{\kappa-1} (p_\text{b} \sqrt[\kappa]{\frac{p_\text{a}}{p_\text{b}}} - p_\text{a}) \frac{1}{\rho} \frac{\Delta m}{\Delta t} .
    \end{equation}
    $\rho$ bezeichnet hier die Dichte des Gases im gasförmigen Zustand.


\subsection{Aufbau einer Wärmepumpe}

% TODO Grafik

    Der grundsätzliche Aufbau der Wärmepumpe besteht aus zwei unterschiedlich warmen Reservoiren und einem Kompressor,
    der das Wärmetransportmedium erhitzt.
    Um die Wärme zwischen den Reservoiren 1 und 2 zu transportieren, wird ein Gas verwendet, welches beim Wechseln in den gasförmigen
    Aggregatzustand Wärme aufnimmt und sie wieder abgibt, sobald es wieder flüssig wird.
    Ein Kompressor $K$ stellt einen Mediumkreislauf in der Wärmepumpe her.
    Zwischen den Reservoiren herrscht ein hoher Druckunterschied, welcher durch ein Drosselventil $D$ erzeugt wird.
    Bei einem Druck [TODO] $p_b$ oder $p_1$ und einer Temperatur $T_1$ aus dem ersten Reservoir ist das Gas flüssig und bei einem Druck [TODO] $p_a$ oder $p_2$
    und einer Temperatur $T_2$ aus dem zweiten Reservoir ist das Gas gasförmig.
    Das Gas wird zu Beginn des Kreislaufs im Kompressor $K$ stark erhitzt und durchläuft anschließend das erste Reservoir.
    Hier wird dem Gas Wärme entzogen und es wird flüssig. Somit ist das erste Reservoir das Wärmenehmende.
    Danach läuft das Gas das Drosselventil $D$, wobei ein Reiniger $R$ das flüssige Medium von Blasenresten trennt.
    Die Durchlässigkeit wird hier durch den Temperaturunterschied zwischen $T_1$ und $T_2$ gesteuert.
    Im zweiten Reservoir nimmt das Gas wieder Wärme auf und wird gasförmig. Somit ist das zweite Reservoir das Wärmeabgebende.
    Das Gas gelangt zurück in den Kompressor und wird wieder erhitzt.

    %TODO Massendurchsatz hinzufügen
    %TODO Kompressorleistung hinzufügen

\section{Durchführung} \label{sec:Durchführung}
    Zu Beginn des Experiments werden die Reservoire aus Kapitel \ref{sec:Theorie} mit einer abgemessenen Wassermenge befüllt.
    Anschließend werden im Abstand $\Delta t$ von $\SI{1}{\minute}$ die Temperaturen $T_1$ und $T_2$ sowie die Drücke $p_a$ und $p_b$ gemessen.
    Auf den abgelesen Wert des Drucks muss zusätzlich ein Wert von 1bar addiert werden.
    Um die Zeitintervalle möglichst genau einzuhalten, sollten die Werte in einer festen Reihenfolge abgelesen werden.
    Bei einer Temeperatur von $\SI{50}{\celsius}$ soll die Messung abgebrochen werden.

\section{Auswertung} \label{sec:Auswertung}
    Bei der Messung haben sich folgende Messdaten ergeben:
    \begin{table}[H]
\centering
\caption{Aufgenommene Messdaten}
\label{tab:Messdaten}
\sisetup{table-format=2.1}
\begin{tabular}{c c c c c c}
\toprule
$t \mathbin{/} \si{\minute}$ &
$T_1 \mathbin{/} \si{\celsius}$ &
$p_1 \mathbin{/} \si{\bar}$ &
$T_2 \mathbin{/} \si{\celsius}$ &
$p_2 \mathbin{/} \si{\bar}$ &
$N \mathbin{/} \si{\watt}$ \\
\midrule
%  Messdaten zur Wärmepumpe
%  4l Wasser, Wärmekapazität der ´Kupferschlangen`: mkck = 750 J/K
%
%  t[min] T1 [C] p1[bar] T2[C] p2 [bar] N [W]
0	 & 21.7 & 4.0  & 21.7 & 4.1 & 120 \\
1  & 23.0 & 5.0  & 21.7 & 3.2	& 120 \\
2  & 24.3 & 5.5  & 21.6 & 3.4	& 120 \\
3  & 25.3 & 6.0  & 21.5 & 3.5	& 120 \\
4  & 26.4 & 6.0  & 20.8 & 3.5	& 120 \\
5  & 27.5 & 6.0  & 20.1 & 3.4	& 120 \\
6  & 28.8 & 6.5  & 19.2 & 3.3	& 120 \\
7  & 29.7 & 6.5  & 18.5 & 3.2	& 120 \\
8  & 30.9 & 7.0  & 17.7 & 3.2	& 120 \\
9  & 31.9 & 7.0  & 16.9 & 3.0	& 120 \\
10 & 32.9 & 7.0  & 16.2 & 3.0	& 120 \\
11 & 33.9 & 7.5  & 15.5 & 2.9	& 120 \\
12 & 34.8 & 7.5  & 14.9 & 2.8	& 120 \\
13 & 35.7 & 8.0  & 14.2 & 2.8	& 120 \\
14 & 36.7 & 8.0  & 13.6 & 2.7	& 120 \\
15 & 37.6 & 8.0  & 13.0 & 2.6	& 120 \\
16 & 38.4 & 8.5  & 12.4 & 2.6	& 120 \\
17 & 39.2 & 8.5  & 11.7 & 2.6	& 120 \\
18 & 40.0 & 9.0  & 11.3 & 2.5	& 120 \\
19 & 40.7 & 9.0  & 10.9 & 2.5	& 120 \\
20 & 41.4 & 9.0  & 10.4 & 2.4	& 120 \\
21 & 42.2 & 9.0  & 9.9  & 2.4	& 120 \\
22 & 42.9 & 9.5  & 9.5  & 2.4	& 120 \\
23 & 43.6 & 9.5  & 9.1  & 2.4	& 120 \\
24 & 44.3 & 0.0	 & 8.7  & 2.4	& 120 \\
25 & 44.9 & 0.0	 & 8.3  & 2.4	& 120 \\
26 & 45.5 & 0.0	 & 8.0  & 2.3	& 120 \\
27 & 46.1 & 0.0	 & 7.7  & 2.2	& 122 \\
28 & 46.7 & 0.5	 & 7.4  & 2.2	& 122 \\
29 & 47.3 & 0.5	 & 7.1  & 2.2	& 122 \\
30 & 47.8 & 0.75 & 6.8  & 2.2	& 122 \\
31 & 48.4 & 1.0	 & 5.6  & 2.2	& 122 \\
32 & 48.9 & 1.0	 & 4.3  & 2.2	& 122 \\
33 & 49.4 & 1.0	 & 3.4  & 2.2	& 122 \\
34 & 49.9 & 1.0	 & 3.0  & 2.2	& 122 \\
35 & 50.3 & 1.0	 & 2.9  & 2.2	& 122 \\
\bottomrule
\end{tabular}
\end{table}


\subsection{Diagramm der gemessenen Temperaturverläufe}

\begin{figure}
  \centering
  \includegraphics[scale=0.8]{build/wärmepumpe_plot.pdf}
  \caption{Temperaturverläufe}
  \label{fig:plot}
\end{figure}

\subsection{Foo}
b) Man versuche mit Hilfe einer nicht-linearen Ausgleichsrechnung die gemessenen
Temperaturverläufe durch einfache Gleichungen zu approximieren. Folgende Nähe-
rungslösungen sind denkbar […]

\subsection{Foo}
c) Aus 5b berechne man für 4 verschiedene Temperaturen die Differentialquotienten
dT 1 /dt und dT 2 /dt.

\subsection{Foo}
d) Für die 4 Temperaturen aus 5c bestimme man die Güteziffer der benutzten Wärme-
pumpe und vergleiche die erhaltenen Werte mit den Güteziffern einer idealen Wär-
mepumpe. Man verwende dazu die Ergebnisse aus 5c. Die Wärmekapazität der
Kupferschlange und des Eimers in den Reservoiren können an den Apparaturen ab-
gelesen werden.

\subsection{Foo}
e) Man errechne für die 4 Temperaturen aus 5c den Massendurchsatz des hier benutz-
ten Transportgases Dichlordifluormethan (Cl 2 F 2 C). Die Verdampfungswärme L die-
ses Stoffes kann man aus seiner Dampfdruck-Kurve gewinnen. (Näheres hierzu fin-
det man in V203.) Wertepaare (p,T) zur Darstellung dieser Kurve können an den
Manometern der Apparatur abgelesen werden. Mit den dort abgelesenen (p, T)-
Werten führe man eine Ausgleichsrechnung zur Bestimmung von L durch.

\subsection{Foo}
f) Man errechne die mechanische Leistung des Kompressors, die dieser abgibt, wenn
er zwischen den Drücken p a und p b arbeitet, für die 4 Temperaturen aus 5c. Daten
für Cl 2 F 2 C: ρ 0 = 5,51 g/l bei T = 0°C und p = 1 Bar, κ = 1,14

\section{Diskussion? [TODO]}

g) Man gebe Gründe für die relativ schlechte Güteziffer (im Vergleich zur idealen Wär-
mepumpe) bei dem hier verwendeten Gerät an.

\end{document}
