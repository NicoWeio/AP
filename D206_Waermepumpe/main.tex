\documentclass[
  bibliography=totoc,     % Literatur im Inhaltsverzeichnis
  captions=tableheading,  % Tabellenüberschriften
  titlepage=firstiscover, % Titelseite ist Deckblatt
]{scrartcl}

% Paket float verbessern
\usepackage{scrhack}

% Warnung, falls nochmal kompiliert werden muss
\usepackage[aux]{rerunfilecheck}

% unverzichtbare Mathe-Befehle
\usepackage{amsmath}
% viele Mathe-Symbole
\usepackage{amssymb}
% Erweiterungen für amsmath
\usepackage{mathtools}

% Fonteinstellungen
\usepackage{fontspec}
% Latin Modern Fonts werden automatisch geladen
% Alternativ zum Beispiel:
%\setromanfont{Libertinus Serif}
%\setsansfont{Libertinus Sans}
%\setmonofont{Libertinus Mono}

% Wenn man andere Schriftarten gesetzt hat,
% sollte man das Seiten-Layout neu berechnen lassen
\recalctypearea{}

% deutsche Spracheinstellungen
\usepackage[ngerman]{babel}


\usepackage[
  math-style=ISO,    % ┐
  bold-style=ISO,    % │
  sans-style=italic, % │ ISO-Standard folgen
  nabla=upright,     % │
  partial=upright,   % ┘
  warnings-off={           % ┐
    mathtools-colon,       % │ unnötige Warnungen ausschalten
    mathtools-overbracket, % │
  },                       % ┘
]{unicode-math}

% traditionelle Fonts für Mathematik
\setmathfont{Latin Modern Math}
% Alternativ zum Beispiel:
%\setmathfont{Libertinus Math}

\setmathfont{XITS Math}[range={scr, bfscr}]
\setmathfont{XITS Math}[range={cal, bfcal}, StylisticSet=1]

% Zahlen und Einheiten
\usepackage[
  locale=DE,                   % deutsche Einstellungen
  separate-uncertainty=true,   % immer Unsicherheit mit \pm
  per-mode=symbol-or-fraction, % / in inline math, fraction in display math
]{siunitx}

% chemische Formeln
\usepackage[
  version=4,
  math-greek=default, % ┐ mit unicode-math zusammenarbeiten
  text-greek=default, % ┘
]{mhchem}

% richtige Anführungszeichen
\usepackage[autostyle]{csquotes}

% schöne Brüche im Text
\usepackage{xfrac}

% Standardplatzierung für Floats einstellen
\usepackage{float}
\floatplacement{figure}{htbp}
\floatplacement{table}{htbp}

% Floats innerhalb einer Section halten
\usepackage[
  section, % Floats innerhalb der Section halten
  below,   % unterhalb der Section aber auf der selben Seite ist ok
]{placeins}

% Seite drehen für breite Tabellen: landscape Umgebung
\usepackage{pdflscape}

% Captions schöner machen.
\usepackage[
  labelfont=bf,        % Tabelle x: Abbildung y: ist jetzt fett
  font=small,          % Schrift etwas kleiner als Dokument
  width=0.9\textwidth, % maximale Breite einer Caption schmaler
]{caption}
% subfigure, subtable, subref
\usepackage{subcaption}

% Grafiken können eingebunden werden
\usepackage{graphicx}

% schöne Tabellen
\usepackage{booktabs}

% Verbesserungen am Schriftbild
\usepackage{microtype}

% Literaturverzeichnis
\usepackage[
  backend=biber,
]{biblatex}
% Quellendatenbank
\addbibresource{lit.bib}
\addbibresource{programme.bib}

% Hyperlinks im Dokument
\usepackage[
  german,
  unicode,        % Unicode in PDF-Attributen erlauben
  pdfusetitle,    % Titel, Autoren und Datum als PDF-Attribute
  pdfcreator={},  % ┐ PDF-Attribute säubern
  pdfproducer={}, % ┘
]{hyperref}
% erweiterte Bookmarks im PDF
\usepackage{bookmark}

% Trennung von Wörtern mit Strichen
\usepackage[shortcuts]{extdash}

\author{%
  Katharina Popp\\%
  \href{mailto:katharina.popp@tu-dortmund.de}{katharina.popp@tu-dortmund.de}%
  \and%
  Nicolai Weitkemper\\%
  \href{mailto:nicolai.weitkemper@tu-dortmund.de}{nicolai.weitkemper@tu-dortmund.de}%
}
\publishers{TU Dortmund – Fakultät Physik}


\subject{D206}
\title{Die Wärmepumpe}
\date{
    Durchführung: 28.10.2020
    \hspace{3em}
    Abgabe: 11.11.2020
}

\begin{document}

\maketitle
\thispagestyle{empty}
\tableofcontents
\newpage

\section{Theorie} \label{sec:Theorie}

\subsection{Grundlegendes Prinzip der Wärmepumpe}

    Die Wärmepumpe wird dazu verwendet, die Richtung des Wärmeflusses umzukehren.
    Der \enquote{natürliche} Wärmefluss findet in einem abgeschlossenen System von einem heißeren Körper
    zu einem kälteren Körper statt.

    Um die Richtung des Wärmeflusses umzukehren, muss (hier: mechanische) Arbeit aufgewendet werden.
    Nach dem ersten Hauptsatz der Thermodynamik ist dabei die abgegebene Temperatur $Q_1$ vom heißeren Körper
    gleich der aufgenommenen Wärme $Q_2$ des kälteren Körpers und der benötigten Arbeit $A$.
    Es gilt also:
    \begin{equation}
        Q_1 = Q_2 + A .
    \end{equation}
    Mit der Beziehung zwischen Wärme und Temperatur im realistischen, irreversiblen Fall
    \begin{equation}
        \frac{Q_1}{T_1} - \frac{Q_2}{T_2} > 0
    \end{equation}
    ergibt sich außerdem:
    \begin{equation}
        Q_1 = A + \frac{T_1}{T_2} Q_2
    \end{equation}

    Das Verhältnis zwischen der abgegebenen Wärme $Q_1$ und der benötigten mechanischen Arbeit $A$ wird durch die Güteziffer $\nu$
    beschrieben.
    \begin{equation}
        \nu = \frac{Q_1}{A}
    \end{equation}

\subsubsection{Bestimmung der realen Güteziffer}
% \label{sec:reale_gueteziffer}

    Die mechanische Arbeit kann auch durch ein Zeitintervall und die Leistungsaufnahme $N$ des Kompressors ausgedrückt werden:
    \begin{equation}
        \nu_\text{real} = \frac{\Delta Q_1}{\Delta t N}
    \end{equation}
    $\frac{\Delta Q_1}{\Delta t}$ beschreibt hier die pro Zeiteinheit gewonnene Wärmemenge. Es gilt
    \begin{equation}
        \frac{\Delta Q_1}{\Delta t} = (m_1 c_\text{w} + m_\text{k} c_\text{k}) \frac{\Delta T_1}{\Delta t}
    \end{equation}
    und somit auch
    \begin{equation}
      \label{eqn:reale_gueteziffer}
      \nu_\text{real} = \frac{1}{N} (m_1 c_\text{w} + m_\text{k} c_\text{k}) \frac{\Delta T_1}{\Delta t} \; .
    \end{equation}

\subsubsection{Bestimmung des Massendurchsatzes}

    Die aus dem Reservoir 2 entnommene Wärmemenge wird durch
    \begin{equation}
        \frac{\Delta Q_2}{\Delta t} =(m_2 c_\text{w} + m_\text{k} c_\text{k}) \frac{\Delta T_2}{\Delta t}
    \end{equation}
    beschrieben.
    Wenn die Verdampfungswärme L bekannt ist, lässt sich der Massendurchsatz aus
    \begin{equation}
        \frac{Q_2}{\Delta t} = L \frac{\Delta m}{\Delta t}
    \end{equation}
    bestimmen.

\subsubsection{Bestimmung der mechanischen Kompressorleistung $N_\text{mech}$}

    Die Arbeit $A_\text{m}$, die benötigt wird, um das Gasvolumen $V_\text{a}$ auf den Wert $V_\text{b}$ zu verringern, wird durch
    \begin{equation}
        A_\text{m} = - \displaystyle\int_{V_\text{a}}^{V_\text{b}} p dV
    \end{equation}
    bestimmt.
    Zudem ergibt sich die Poisson'sche Gleichung als Zusammenhang zwischen Druck und Volumen
    \begin{equation}
        p_\text{a} V_\text{a}^\text{\kappa} = p_\text{b} V_\text{b}^\text{\kappa} = p V^\text{\kappa} .
    \end{equation}
    Für $A_\text{m}$ ergibt sich damit
    \begin{equation}
       A_\text{m} = - p_\text{a} V_\text{a}^\text{\kappa} \displaystyle\int_{V_\text{a}}^{V_\text{b}} V^{-\kappa} dV
                  = \frac{1}{\kappa - 1} p_\text{a} V_\text{a}^\text{\kappa} (V_\text{b}^{-\kappa+1} - V_\text{a}^{-\kappa+1})
                  = \frac{1}{\kappa - 1} (p_\text{b} \sqrt[\kappa]{\frac{p_\text{a}}{p_\text{b}}} - p_\text{a}) V_\text{a} .
    \end{equation}
    Außerdem gilt für die mechanische Kompressorleistung mit der Beziehung $V_\text{a} = \frac{1}{\rho} m$
    \begin{equation}
      \label{eqn:N_mech}
        N_\text{mech} = \frac{\Delta A_\text{m}}{\Delta t}
                      = \frac{1}{\kappa-1} \left(p_\text{b} \sqrt[\kappa]{\frac{p_\text{a}}{p_\text{b}}} - p_\text{a}\right) \frac{1}{\rho} \frac{\Delta m}{\Delta t} .
    \end{equation}
    Hierbei bezeichnet $\rho$ die Dichte des Gases im gasförmigen Zustand.


\subsection{Aufbau einer Wärmepumpe}

% TODO Grafik

    Der grundsätzliche Aufbau der Wärmepumpe besteht aus zwei unterschiedlich warmen Reservoiren und einem Kompressor $K$,
    der das Wärmetransportmedium durch die Kompression erhitzt.
    Um die Wärme zwischen den Reservoiren 1 und 2 zu transportieren, wird ein Gas verwendet, welches beim Wechsel in den gasförmigen
    Aggregatzustand Wärme aufnimmt und sie wieder abgibt, sobald es wieder flüssig wird.

    Der Kompressor stellt einen Mediumkreislauf in der Wärmepumpe her.
    Zwischen den Reservoiren herrscht ein hoher Druckunterschied, welcher durch ein Drosselventil $D$ erzeugt wird.
    Bei einem Druck $p_1$ und einer Temperatur $T_1$ aus dem ersten Reservoir ist das Gas flüssig,
    bei einem Druck $p_2$ und einer Temperatur $T_2$ aus dem zweiten Reservoir ist das Gas gasförmig.

    Das Gas wird zu Beginn des Kreislaufs im Kompressor $K$ stark erhitzt und durchläuft anschließend das erste Reservoir.
    Hier wird dem Gas Wärme entzogen und es wird flüssig. Somit ist das erste Reservoir das Wärmenehmende.
    Danach läuft das Gas das Drosselventil $D$, wobei ein Reiniger $R$ das flüssige Medium von Blasenresten trennt, um den Kompressor nicht zu beschädigen.
    Die Durchlässigkeit wird hier durch den Temperaturunterschied zwischen $T_1$ und $T_2$ gesteuert.
    Im zweiten Reservoir nimmt das Gas wieder Wärme auf und wird gasförmig. Somit ist das zweite Reservoir das Wärmeabgebende.
    Das Gas gelangt zurück in den Kompressor und wird wieder erhitzt.

\section{Durchführung} \label{sec:Durchführung}
    Zu Beginn des Experiments werden die Reservoire aus Kapitel \ref{sec:Theorie} mit einer abgemessenen Wassermenge (hier: $\SI{4}{\kilogram}$) befüllt.
    Anschließend werden im Abstand von $\Delta t = \SI{1}{\minute}$ die Temperaturen $T_1$ und $T_2$, die Drücke $p_1$ und $p_2$, und die Leistungsaufnahme des Kompressors $N$ gemessen.
    Auf den abgelesen Wert des Drucks muss zusätzlich ein Wert von $\SI{1}{\bar}$ addiert werden.
    % Um die Zeitintervalle möglichst genau einzuhalten, sollten die Werte in einer festen Reihenfolge abgelesen werden.
    Bei einer Temperatur von $\SI{50}{\celsius}$ soll die Messung abgebrochen werden.

\section{Auswertung} \label{sec:Auswertung}
Die zur Verfügung gestellten Messdaten sind im Folgenden tabellarisch aufgelistet:
\begin{table}[H]
\centering
\caption{Aufgenommene Messdaten}
\label{tab:Messdaten}
\sisetup{table-format=2.1}
\begin{tabular}{c c c c c c}
\toprule
$t \mathbin{/} \si{\minute}$ &
$T_1 \mathbin{/} \si{\celsius}$ &
$p_1 \mathbin{/} \si{\bar}$ &
$T_2 \mathbin{/} \si{\celsius}$ &
$p_2 \mathbin{/} \si{\bar}$ &
$N \mathbin{/} \si{\watt}$ \\
\midrule
%  Messdaten zur Wärmepumpe
%  4l Wasser, Wärmekapazität der ´Kupferschlangen`: mkck = 750 J/K
%
%  t[min] T1 [C] p1[bar] T2[C] p2 [bar] N [W]
0	 & 21.7 & 4.0  & 21.7 & 4.1 & 120 \\
1  & 23.0 & 5.0  & 21.7 & 3.2	& 120 \\
2  & 24.3 & 5.5  & 21.6 & 3.4	& 120 \\
3  & 25.3 & 6.0  & 21.5 & 3.5	& 120 \\
4  & 26.4 & 6.0  & 20.8 & 3.5	& 120 \\
5  & 27.5 & 6.0  & 20.1 & 3.4	& 120 \\
6  & 28.8 & 6.5  & 19.2 & 3.3	& 120 \\
7  & 29.7 & 6.5  & 18.5 & 3.2	& 120 \\
8  & 30.9 & 7.0  & 17.7 & 3.2	& 120 \\
9  & 31.9 & 7.0  & 16.9 & 3.0	& 120 \\
10 & 32.9 & 7.0  & 16.2 & 3.0	& 120 \\
11 & 33.9 & 7.5  & 15.5 & 2.9	& 120 \\
12 & 34.8 & 7.5  & 14.9 & 2.8	& 120 \\
13 & 35.7 & 8.0  & 14.2 & 2.8	& 120 \\
14 & 36.7 & 8.0  & 13.6 & 2.7	& 120 \\
15 & 37.6 & 8.0  & 13.0 & 2.6	& 120 \\
16 & 38.4 & 8.5  & 12.4 & 2.6	& 120 \\
17 & 39.2 & 8.5  & 11.7 & 2.6	& 120 \\
18 & 40.0 & 9.0  & 11.3 & 2.5	& 120 \\
19 & 40.7 & 9.0  & 10.9 & 2.5	& 120 \\
20 & 41.4 & 9.0  & 10.4 & 2.4	& 120 \\
21 & 42.2 & 9.0  & 9.9  & 2.4	& 120 \\
22 & 42.9 & 9.5  & 9.5  & 2.4	& 120 \\
23 & 43.6 & 9.5  & 9.1  & 2.4	& 120 \\
24 & 44.3 & 0.0	 & 8.7  & 2.4	& 120 \\
25 & 44.9 & 0.0	 & 8.3  & 2.4	& 120 \\
26 & 45.5 & 0.0	 & 8.0  & 2.3	& 120 \\
27 & 46.1 & 0.0	 & 7.7  & 2.2	& 122 \\
28 & 46.7 & 0.5	 & 7.4  & 2.2	& 122 \\
29 & 47.3 & 0.5	 & 7.1  & 2.2	& 122 \\
30 & 47.8 & 0.75 & 6.8  & 2.2	& 122 \\
31 & 48.4 & 1.0	 & 5.6  & 2.2	& 122 \\
32 & 48.9 & 1.0	 & 4.3  & 2.2	& 122 \\
33 & 49.4 & 1.0	 & 3.4  & 2.2	& 122 \\
34 & 49.9 & 1.0	 & 3.0  & 2.2	& 122 \\
35 & 50.3 & 1.0	 & 2.9  & 2.2	& 122 \\
\bottomrule
\end{tabular}
\end{table}


\newpage
\subsection{Diagramm der gemessenen Temperaturverläufe} % a)
Im Diagramm sind die Approximationen aus \ref{sec:approx} bereits enthalten.

\begin{figure}
  \centering
  \includegraphics[scale=0.8]{build/wärmepumpe_plot.pdf}
  \caption{Temperaturverläufe}
  \label{fig:plot}
\end{figure}

\subsection{Approximation der Temperaturverläufe} \label{sec:approx} % b)
Zur Approximation des Temperaturverläufe bietet sich ein Polynom zweiten Grades an:
\[
T(t) = At^2 + Bt + C \; .
\]
Als Fit-Parameter wurden berechnet:

\begin{table}
\centering
\caption{Fit-Parameter}
\label{tab:fit_params}
% \sisetup{table-format=2.1}
\begin{tabular}{c c c c}
\toprule
 & A & B & C \\
\midrule
$T_1$ &
\num{-3.2249e-06} ± \num{4.1934e-08} &
\num{0.02028} ± \num{9.1105e-05} &
\num{294.97} ± \num{0.041345}\\
$T_2$ &
\num{9.5504e-07} ± \num{2.6711e-07} &
\num{-0.011209} ± \num{0.00058032} &
\num{295.87} ± \num{0.26336}\\
\bottomrule
\end{tabular}
\end{table}

\subsection{Berechnung von Differentialquotienten} % c)
Mithilfe der Approximation aus \ref{sec:approx} werden nun für vier verschiedene Temperaturen konkrete  Differentialquotienten $\frac{\mathrm{d}T_1}{\mathrm{d}t}$ und $\frac{\mathrm{d}T_2}{\mathrm{d}t}$ berechnet.
Die Ableitung des Polynoms lautet
\[
\frac{\mathrm{d}T}{\mathrm{d}t} = 2At + B \; .
\]
\\
Fehllerrechnung:
\begin{align*}
  \Delta \frac{\mathrm{d}T}{\mathrm{d}t}
  &= \sqrt{\left(\frac{\partial \frac{\mathrm{d}T}{\mathrm{d}t}}{\partial A} \cdot \Delta A\right)^2 + \left(\frac{\partial \frac{\mathrm{d}T}{\mathrm{d}t}}{\partial B} \cdot \Delta B\right)^2} \\
  &= \sqrt{(2t \cdot \Delta A)^2 + (\Delta B)^2}
\end{align*}

\begin{table}
\centering
\caption{Ableitungen der Approximationsfunktion}
\label{tab:derivatives}
\sisetup{table-format=2.1}
\begin{tabular}{c c c c c c}
\toprule
$t \,/\, \si{\minute}$ &
$\frac{\mathrm{d}T_1}{\mathrm{d}t} \,/\, \si{\celsius\per\minute}$ &
$\frac{\mathrm{d}T_2}{\mathrm{d}t} \,/\, \si{\celsius\per\minute}$ \\
\midrule
7  & \num{0.017571}  ± \num{0.000098}  & \num{-0.010406}  ± \num{0.000622} \\
14 & \num{0.014862}  ± \num{0.000115}  & \num{-0.0096042} ± \num{0.0007336} \\
21 & \num{0.012153}  ± \num{0.000140}  & \num{-0.0088020} ± \num{0.0008887} \\
28 & \num{0.0094441} ± \num{0.0001678} & \num{-0.0079998} ± \num{0.0010688} \\

% 7  & \num{1.757e-02} ± \num{0.010e-02} & \num{-1.041e-02} ± \num{0.062e-02} \\
% 14 & \num{1.486e-02} ± \num{0.012e-02} & \num{-9.604e-03} ± \num{0.734e-03} \\
% 21 & \num{1.215e-02} ± \num{0.014e-02} & \num{-8.802e-03} ± \num{0.889e-03} \\
% 28 & \num{9.444e-03} ± \num{0.168e-03} & \num{-8.000e-03} ± \num{1.069e-03} \\


\bottomrule
\end{tabular}
\end{table}


\subsection{Bestimmung der Güteziffern} % d)
\label{sec:auswertung_gueteziffern}
% In "Daten und Hinweise" ist lediglich die Rede vom *Fassungsvermögen*!
% „Fassungsvermögen der Wassereimer: m = 4kg“
% Als Füllmenge nehmen wir daher 4L an, wie es in "Daten.dat" steht

% „Wärmekapazität der Kupferschlangen“ meint *Wärmekapazität jeder Kupferschlange*, oder?
Jeder Wassereimer war mit $\SI{4}{\liter}$ Wasser befüllt, bei einem Fassungsvermögen von $\SI{4}{\kilogram}$.
Die Wärmekapazität der \enquote{Kupferschlangen} war gegeben als $m_k c_k = 750 \si{\joule\per\kelvin}$. Laut Versuchsanleitung soll $m_k c_k$ auch die Wärmekapazität der Eimer enthalten, dazu liegen uns jedoch keine weiteren Daten vor.
Die Wärmekapazität des Wassers in Reservoir 1 lässt sich mit
\begin{align*}
  \rho_\text{Wasser} &= \SI{0.998207}{\kilogram\per\liter}
  \tag*{(Wasserdichte bei 20°C \cite{wasserdichte})} \\
  \\
  c_\text{Wasser} &= \SI{4.1851}{\joule\per\gram\and\kelvin}
  \tag*{(spezifische Wärmekapazität von Wasser bei 20°C \cite{wasserwaermekapazitaet})} \\
  \\
  C_{\text{Wasser}, 1} &= m_1 \cdot c_\text{Wasser} \\
  &= (\rho_\text{Wasser} \cdot V_\text{Wasser}) \cdot c_\text{Wasser} \\
  &= \left(\SI{0.998207}{\kilogram\per\liter} \cdot \SI{4}{\liter}\right) \cdot \SI{4.181}{\joule\per\gram\and\kelvin} \\
  &= \SI{16736.0}{\joule\per\kelvin} \\
\end{align*}
bestimmen, sodass aus einfacher Summation die gesamte Wärmekapazität
\[
C_\text{ges} = 750 \si{\joule\per\kelvin} + \SI{16736.0}{\joule\per\kelvin} = \SI{17486.0}{\joule\per\kelvin}
\]
folgt.

\ \\
Unter Zuhilfenahme von Gleichung \ref{eqn:reale_gueteziffer} und der Differentialquotienten aus Tabelle \ref{tab:derivatives} kann nun die reale Güteziffer für verschiedene $t$ berechnet werden.

\begin{table}
\centering
\caption{Vergleich der idealen und realen Güteziffern}
% \sisetup{table-format=2.1}
\begin{tabular}{c c c c c c}
\toprule
$t \,/\, \si{\minute}$ &
$\nu_\text{ideal}$ &
$\nu_\text{real}$ \\
% TODO: Abweichung in % dazu?
\midrule
7  & 27.04 ± 0.34  & 2.560 ± 0.014 \\
14 & 13.41 ± 0.08  & 2.166 ± 0.017 \\
21 & 9.76  ± 0.04  & 1.771 ± 0.020 \\
28 & 8.139 ± 0.028 & 1.354 ± 0.024 \\
\bottomrule
\end{tabular}
\end{table}

\subsection{Bestimmung des Massendurchsatzes} % e)
% e) Man errechne für die 4 Temperaturen aus 5c den Massendurchsatz des hier benutzten Transportgases Dichlordifluormethan (Cl 2 F 2 C). Die Verdampfungswärme L dieses Stoffes kann man aus seiner Dampfdruck-Kurve gewinnen. (Näheres hierzu findet man in V203.) Wertepaare (p,T) zur Darstellung dieser Kurve können an den Manometern der Apparatur abgelesen werden. Mit den dort abgelesenen (p, T)-Werten führe man eine Ausgleichsrechnung zur Bestimmung von L durch.

% TODO: Verschieben zur Theorie → 1.1.2

universelle Gaskonstante $R$, TODO

\begin{align*}
  \frac{1}{p} \mathrm{d}p &= \frac{L}{R \cdot T^2} \mathrm{d} T \\
  \mathrm{ln}(p) &= \frac{L}{R} \cdot \frac{1}{T} + c \\
  \mathrm{ln}(p) &= \frac{L}{R \cdot T} + c \\
  \mathrm{ln}(p) \cdot (R \cdot T) &= L + c \\
\end{align*}

\begin{table}
\centering
\caption{Massendurchsatz zu verschiedenen Zeitpunkten}
% \sisetup{table-format=2.1}
\begin{tabular}{c c c c c c}
\toprule
$t \,/\, \si{\minute}$ &
$\nu_\text{ideal}$ &
$\nu_\text{real}$ \\
% TODO: Abweichung in % dazu?
\midrule
7  | deriv: (-1.041+/-0.062)e-02 & -1.54+/-0.09
14 | deriv: (-9.604+/-0.734)e-03 & -1.42+/-0.11
21 | deriv: (-8.802+/-0.889)e-03 & -1.30+/-0.13
28 | deriv: (-8.000+/-1.069)e-03 & -1.18+/-0.16
\bottomrule
\end{tabular}
\end{table}


\subsection{Bestimmung der Kompressorleistung} % f)
Die mechanische Leistung des Kompressors, die dieser abgibt, wenn
er zwischen den Drücken $p_1$ und $p_2$ arbeitet,
wird mithilfe von Gleichung \ref{eqn:N_mech}, der gegebenen Daten, und der Differentialquotienten aus Tabelle \ref{tab:derivatives} berechnet.
\ \\
Daten für $\mathrm{Cl}_2 \mathrm{F}_2 \mathrm{C}$: \\
% \[
$ρ_0 = \SI{5.51} g/l$ bei $T = \SI{0}{\celsius}$ und $p = \SI{1}{\bar}$, $κ = 1.14$
% \]


\section{Diskussion}
% TODO Verdampfungswärme Vgl. m. Literaturwert
Auffällig bei der Auswertung war die große Abweichung der realen Güteziffer vom Idealwert. Mögliche Gründe dafür sind vor allem folgende:
\begin{description}
  \item[nicht-ideale Isolierung]
  Obwohl Reservoire und Leitungen eine Isolierung aufweisen,
  lassen sich Wärmeverluste beziehungsweise Wärmeaufnahme in Reservoir 2 nicht vollständig vermeiden.

  \item[keine vollständig adiabatische Kompression]
  % → Mampfzwerg
  Zur Bestimmung der mechanischen Kompressorleistung $N_\text{mech}$ wurde näherungsweise angenommen,
  dass die Kompression adiabatisch erfolgt.

  \item[Ungenauigkeiten beim Ablesen]
  Da die Druckmessung mit analogen Manometern erfolgte,
  könnte an dieser Stelle eine Messunsicherheit aufgetreten sein.
  Zudem sind jede Minute fünf Messwerte aufzunehmen.
  Da dies manuell und nacheinander geschieht,
  entsteht auch dabei ein subjektiver Fehler.
  Es muss betont werden,
  dass wir den Versuch nicht selbst durchgeführt haben und daher in diesen Punkten nur mutmaßen können.

  \item[Wahl der vier Zeitindizes] Zur Bestimmung der vier Differentialquotienten, die im weiteren Verlauf genutzt wurden, sollten vier Zeitindizes $t_1, \ldots, t_4$ gewählt werden. Wir haben uns für die äquidistanten Zeiten \textbf{7, 14, 21, 28} (von 35 Minuten) entschieden. Eine andere Wahl hätte die Werte für die ideale und reale Güteziffer verändert.

  \item[Fehler der Ausgleichsfunktion]
  % → vsulaiman
  Da wir die Differentialquotienten der Temperaturverläufe mit einer Ausgleichsfunktion berechnet haben, spielt auch deren Fit eine Rolle. Allerdings war die Ungenauigkeit der Fit-Parameter so gering, dass daraus keine größeren Abweichungen resultieren sollten.
\end{description}

\printbibliography

\end{document}
