\input{header.tex}

\subject{D206}
\title{Die Wärmepumpe}
\author{Nicolai Weitkemper \and Katharina Popp}
\date{
    Durchführung: 28.10.2020
    \hspace{3em}
    Abgabe: 04.11.2020
}

\begin{document}

\maketitle
\thispagestyle{empty}
\tableofcontents
\newpage

\section{Theorie} \label{sec:Theorie}

    Die Wärmepumpe wird dazu verwendet, die Richtung des Wärmeflusses umzukehren.
    Der \enquote{natürliche} Wärmefluss findet in einem abgeschlossenen System von einem heißeren Körper
    zu einem kälteren Körper statt.

    Um die Richtung des Wärmeflusses umzukehren, muss (mechanische) Arbeit aufgewendet werden.
    Nach dem ersten Hauptsatz der Thermodynamik muss dabei die abgegebene Temperatur $Q_1$ vom heißeren Körper
    gleich der aufgenommenen Wärme $Q_2$ des kälteren Körpers und der benötigten Arbeit A sein.
    Es gilt also:
    \begin{equation}
        Q_1 = Q_2 + A .
    \end{equation}
    Mit der Beziehung zwischen Wärme und Temperatur im realistischen, irreversiblen Fall
    \begin{equation}
        \frac{Q_1}{T_1} - \frac{Q_2}{T_2} > 0
    \end{equation}
    ergibt sich außerdem:
    \begin{equation}
        Q_1 = A + \frac{T_1}{T_2} Q_2
    \end{equation}
    Das Verhältnis zwischen der abgegeben Wärme $Q_1$ und der benötigten mechanischen Arbeit A wird durch die Güteziffer $\nu$
    beschrieben.
    \begin{equation}
        \nu = \frac{Q_1}{A}
    \end{equation}
    Die mechanische Arbeit kann auch durch ein Zeitintervall und die Leistungsaufnahme N des Kompressors ausgedrückt werden:
    \begin{equation}
        \nu = \frac{\Delta Q_1}{\Delta t N}
    \end{equation}

    Der grundsätzliche Aufbau der Wärmepumpe besteht aus zwei unterschiedlich warmen Reservoiren und einem Kompressor,
    der das Wärmetransportmedium erhitzt.
    Um die Wärme zwischen den Reservoiren 1 und 2 zu transportieren, wird ein Gas verwendet, welches beim Wechseln in den gasförmigen
    Aggregatzustand Wärme aufnimmt und sie wieder abgibt, sobald es wieder flüssig wird.
    Ein Kompressor K stellt einen Mediumkreislauf in der Wärmepumpe her.
    Zwischen den Reservoiren herrscht ein hoher Druckunterschied, welcher durch ein Drosselventil D erzeugt wird.
    Bei einem Druck [TODO] $p_b$ oder $p_1$ und einer Temperatur $T_1$ aus dem ersten Reservoir ist das Gas flüssig und bei einem Druck [TODO] $p_a$ oder $p_2$
    und einer Temperatur $T_2$ aus dem zweiten Reservoir ist das Gas gasförmig.
    Das Gas wird zu Beginn des Kreislaufs im Kompressor K stark erhitzt und durchläuft anschließend das erste Reservoir.
    Hier wird dem Gas Wärme entzogen und es wird flüssig. Somit ist das erste Reservoir das Wärmenehmende.
    Danach läuft das Gas das Drosselventil D, wobei ein Reiniger R das flüssige Medium von Blasenresten trennt.
    Die Durchlässigkeit wird hier durch den Temperaturunterschied zwischen $T_1$ und $T_2$ gesteuert.
    Im zweiten Reservoir nimmt das Gas wieder Wärme auf und wird gasförmig. Somit ist das zweite Reservoir das Wärmeabgebende.
    Das Gas gelangt zurück in den Kompressor und wird wieder erhitzt.

    %TODO Massendurchsatz hinzufügen
    %TODO Kompressorleistung hinzufügen

\section{Durchführung} \label{sec:Durchführung}
    Zu Beginn des Experiments werden die Reservoire aus Kapitel \ref{sec:Theorie} [TODO] einer abgefüllten[!?] Wassermenge befüllt.
    Anschließend werden im Abstand $\Delta t$ von $\SI{1}{\minute}$ die Temperaturen $T_1$ und $T_2$ sowie die Drücke $p_a$ und $p_b$ gemessen.
    Auf den abgelesen Wert des Drucks muss zusätzlich ein Wert von 1bar addiert werden.
    Um die Zeitintervalle möglichst genau einzuhalten, sollten die Werte in einer festen Reihenfolge abgelesen werden.
    Bei einer Temeperatur von $\SI{50}{\celsius}$ soll die Messung abgebrochen werden.

\section{Auswertung} \label{sec:Auswertung}
    Bei der Messung haben sich folgende Messdaten ergeben:

    % \input{table1.tex}
    [TODO]

\end{document}
