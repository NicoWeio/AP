\section{Diskussion}
Insgesamt scheint der Versuch gut gelungen zu sein.
Die Messwerte weisen keine großen Abweichungen auf,
und die daraus berechneten Daten wie Verdampfungswärme und reale Güteziffer sind mit Literaturwerten vergleichbar.

Auffällig war jedoch die große Abweichung der realen Güteziffer vom Idealwert. Mögliche Gründe dafür sind vor allem folgende:
\begin{description}
  \item[Nicht-ideale Isolierung]
  Obwohl Reservoire und Leitungen eine Isolierung aufweisen,
  lassen sich Wärmeverluste beziehungsweise Wärmeaufnahme in Reservoir 2 nicht vollständig vermeiden.

  \item[Keine vollständig adiabatische Kompression]
  % → Mampfzwerg
  Zur Bestimmung der mechanischen Kompressorleistung $N_\text{mech}$ wurde näherungsweise angenommen,
  dass die Kompression adiabatisch erfolgt.
  Zudem wurde ein ideales Gas angenommen,
  um die Approximation der Clausius-Clapeyron-Gleichung verwenden zu können.

  \item[Ungenauigkeiten beim Ablesen]
  Da die Druckmessung mit analogen Manometern erfolgte,
  könnte an dieser Stelle eine Messunsicherheit aufgetreten sein.
  Zudem sind jede Minute fünf Messwerte aufzunehmen.
  Da dies manuell und nacheinander geschieht,
  entsteht auch dabei ein subjektiver Fehler.
  Es muss betont werden,
  dass der Versuch nicht von den Urhebern dieses Protokolls durchgeführt wurde
  und daher in diesen Punkten nur Mutmaßungen geäußert werden können.

  \item[Wahl der vier Zeitindizes] Zur Bestimmung der vier Differentialquotienten,
  die im weiteren Verlauf genutzt wurden,
  wurden die äquidistanten Zeiten \textbf{7, 14, 21, 28} (von 35 Minuten) als Zeitindizes $t_1, \ldots, t_4$ gewählt.
  Eine andere Wahl hätte die Werte für die ideale und reale Güteziffer verändert.

  \item[Fehler der Ausgleichsfunktion]
  % → vsulaiman
  Da die Differentialquotienten der Temperaturverläufe mit einer Ausgleichsfunktion berechnet wurden, spielt auch deren Fit eine Rolle. Allerdings war die Ungenauigkeit der Fit-Parameter so gering, dass daraus keine nennenswerten Abweichungen resultieren.
\end{description}
