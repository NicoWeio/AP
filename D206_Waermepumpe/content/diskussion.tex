\section{Diskussion}
% TODO Verdampfungswärme Vgl. m. Literaturwert
Auffällig bei der Auswertung war die große Abweichung der realen Güteziffer vom Idealwert. Mögliche Gründe dafür sind vor allem folgende:
\begin{description}
  \item[nicht-ideale Isolierung]
  Obwohl Reservoire und Leitungen eine Isolierung aufweisen,
  lassen sich Wärmeverluste beziehungsweise Wärmeaufnahme in Reservoir 2 nicht vollständig vermeiden.

  \item[keine vollständig adiabatische Kompression]
  % → Mampfzwerg
  Zur Bestimmung der mechanischen Kompressorleistung $N_\text{mech}$ wurde näherungsweise angenommen,
  dass die Kompression adiabatisch erfolgt.

  \item[Ungenauigkeiten beim Ablesen]
  Da die Druckmessung mit analogen Manometern erfolgte,
  könnte an dieser Stelle eine Messunsicherheit aufgetreten sein.
  Zudem sind jede Minute fünf Messwerte aufzunehmen.
  Da dies manuell und nacheinander geschieht,
  entsteht auch dabei ein subjektiver Fehler.
  Es muss betont werden,
  dass wir den Versuch nicht selbst durchgeführt haben und daher in diesen Punkten nur mutmaßen können.

  \item[Wahl der vier Zeitindizes] Zur Bestimmung der vier Differentialquotienten, die im weiteren Verlauf genutzt wurden, sollten vier Zeitindizes $t_1, \ldots, t_4$ gewählt werden. Wir haben uns für die äquidistanten Zeiten \textbf{7, 14, 21, 28} (von 35 Minuten) entschieden. Eine andere Wahl hätte die Werte für die ideale und reale Güteziffer verändert.

  \item[Fehler der Ausgleichsfunktion]
  % → vsulaiman
  Da wir die Differentialquotienten der Temperaturverläufe mit einer Ausgleichsfunktion berechnet haben, spielt auch deren Fit eine Rolle. Allerdings war die Ungenauigkeit der Fit-Parameter so gering, dass daraus keine größeren Abweichungen resultieren sollten.
\end{description}
