\section{Theorie} \label{sec:Theorie}

\subsection{Grundlegendes Prinzip der Wärmepumpe}

    Die Wärmepumpe wird dazu verwendet, die Richtung des Wärmeflusses umzukehren.
    Der \enquote{natürliche} Wärmefluss findet in einem abgeschlossenen System von einem heißeren Körper
    zu einem kälteren Körper statt.

    Um die Richtung des Wärmeflusses umzukehren, muss (hier: mechanische) Arbeit aufgewendet werden.
    Nach dem ersten Hauptsatz der Thermodynamik ist dabei die abgegebene Temperatur $Q_1$ vom heißeren Körper
    gleich der aufgenommenen Wärme $Q_2$ des kälteren Körpers und der benötigten Arbeit $A$.
    Es gilt also:
    \begin{equation}
        Q_1 = Q_2 + A .
    \end{equation}
    Mit der Beziehung zwischen Wärme und Temperatur im realistischen, irreversiblen Fall
    \begin{equation}
        \frac{Q_1}{T_1} - \frac{Q_2}{T_2} > 0
    \end{equation}
    ergibt sich außerdem:
    \begin{equation}
        Q_1 = A + \frac{T_1}{T_2} Q_2
    \end{equation}

    Das Verhältnis zwischen der abgegebenen Wärme $Q_1$ und der benötigten mechanischen Arbeit $A$ wird durch die Güteziffer $\nu$
    beschrieben.
    \begin{equation}
        \nu = \frac{Q_1}{A}
    \end{equation}

\subsubsection{Bestimmung der realen Güteziffer}
% \label{sec:reale_gueteziffer}

    Die mechanische Arbeit kann auch durch ein Zeitintervall und die Leistungsaufnahme $N$ des Kompressors ausgedrückt werden:
    \begin{equation}
        \nu_\text{real} = \frac{\Delta Q_1}{\Delta t N}
    \end{equation}
    $\frac{\Delta Q_1}{\Delta t}$ beschreibt hier die pro Zeiteinheit gewonnene Wärmemenge. Es gilt
    \begin{equation}
        \frac{\Delta Q_1}{\Delta t} = (m_1 c_\text{w} + m_\text{k} c_\text{k}) \frac{\Delta T_1}{\Delta t}
    \end{equation}
    und somit auch
    \begin{equation}
      \label{eqn:reale_gueteziffer}
      \nu_\text{real} = \frac{1}{N} (m_1 c_\text{w} + m_\text{k} c_\text{k}) \frac{\Delta T_1}{\Delta t} \; .
    \end{equation}

\subsubsection{Bestimmung des Massendurchsatzes}

    Die aus dem Reservoir 2 entnommene Wärmemenge wird durch
    \begin{equation}
        \frac{\Delta Q_2}{\Delta t} =(m_2 c_\text{w} + m_\text{k} c_\text{k}) \frac{\Delta T_2}{\Delta t}
    \end{equation}
    beschrieben.
    Wenn die Verdampfungswärme L bekannt ist, lässt sich der Massendurchsatz aus
    \begin{equation}
        \frac{Q_2}{\Delta t} = L \frac{\Delta m}{\Delta t}
    \end{equation}
    bestimmen.

\subsubsection{Bestimmung der mechanischen Kompressorleistung $N_\text{mech}$}

    Die Arbeit $A_\text{m}$, die benötigt wird, um das Gasvolumen $V_\text{a}$ auf den Wert $V_\text{b}$ zu verringern, wird durch
    \begin{equation}
        A_\text{m} = - \displaystyle\int_{V_\text{a}}^{V_\text{b}} p dV
    \end{equation}
    bestimmt.
    Zudem ergibt sich die Poisson'sche Gleichung als Zusammenhang zwischen Druck und Volumen
    \begin{equation}
        p_\text{a} V_\text{a}^\text{\kappa} = p_\text{b} V_\text{b}^\text{\kappa} = p V^\text{\kappa} .
    \end{equation}
    Für $A_\text{m}$ ergibt sich damit
    \begin{equation}
       A_\text{m} = - p_\text{a} V_\text{a}^\text{\kappa} \displaystyle\int_{V_\text{a}}^{V_\text{b}} V^{-\kappa} dV
                  = \frac{1}{\kappa - 1} p_\text{a} V_\text{a}^\text{\kappa} (V_\text{b}^{-\kappa+1} - V_\text{a}^{-\kappa+1})
                  = \frac{1}{\kappa - 1} (p_\text{b} \sqrt[\kappa]{\frac{p_\text{a}}{p_\text{b}}} - p_\text{a}) V_\text{a} .
    \end{equation}
    Außerdem gilt für die mechanische Kompressorleistung mit der Beziehung $V_\text{a} = \frac{1}{\rho} m$
    \begin{equation}
      \label{eqn:N_mech}
        N_\text{mech} = \frac{\Delta A_\text{m}}{\Delta t}
                      = \frac{1}{\kappa-1} \left(p_\text{b} \sqrt[\kappa]{\frac{p_\text{a}}{p_\text{b}}} - p_\text{a}\right) \frac{1}{\rho} \frac{\Delta m}{\Delta t} .
    \end{equation}
    Hierbei bezeichnet $\rho$ die Dichte des Gases im gasförmigen Zustand.


\subsection{Aufbau einer Wärmepumpe}

% TODO Grafik

    Der grundsätzliche Aufbau der Wärmepumpe besteht aus zwei unterschiedlich warmen Reservoiren und einem Kompressor $K$,
    der das Wärmetransportmedium durch die Kompression erhitzt.
    Um die Wärme zwischen den Reservoiren 1 und 2 zu transportieren, wird ein Gas verwendet, welches beim Wechsel in den gasförmigen
    Aggregatzustand Wärme aufnimmt und sie wieder abgibt, sobald es wieder flüssig wird.

    Der Kompressor stellt einen Mediumkreislauf in der Wärmepumpe her.
    Zwischen den Reservoiren herrscht ein hoher Druckunterschied, welcher durch ein Drosselventil $D$ erzeugt wird.
    Bei einem Druck $p_1$ und einer Temperatur $T_1$ aus dem ersten Reservoir ist das Gas flüssig,
    bei einem Druck $p_2$ und einer Temperatur $T_2$ aus dem zweiten Reservoir ist das Gas gasförmig.

    Das Gas wird zu Beginn des Kreislaufs im Kompressor $K$ stark erhitzt und durchläuft anschließend das erste Reservoir.
    Hier wird dem Gas Wärme entzogen und es wird flüssig. Somit ist das erste Reservoir das Wärmenehmende.
    Danach läuft das Gas das Drosselventil $D$, wobei ein Reiniger $R$ das flüssige Medium von Blasenresten trennt, um den Kompressor nicht zu beschädigen.
    Die Durchlässigkeit wird hier durch den Temperaturunterschied zwischen $T_1$ und $T_2$ gesteuert.
    Im zweiten Reservoir nimmt das Gas wieder Wärme auf und wird gasförmig. Somit ist das zweite Reservoir das Wärmeabgebende.
    Das Gas gelangt zurück in den Kompressor und wird wieder erhitzt.
