\section{Auswertung} \label{sec:Auswertung}
Die zur Verfügung gestellten Messdaten sind im Folgenden tabellarisch aufgelistet:
\begin{table}[H]
\centering
\caption{Aufgenommene Messdaten}
\label{tab:Messdaten}
\sisetup{table-format=2.1}
\begin{tabular}{c c c c c c}
\toprule
$t \mathbin{/} \si{\minute}$ &
$T_1 \mathbin{/} \si{\celsius}$ &
$p_1 \mathbin{/} \si{\bar}$ &
$T_2 \mathbin{/} \si{\celsius}$ &
$p_2 \mathbin{/} \si{\bar}$ &
$N \mathbin{/} \si{\watt}$ \\
\midrule
%  Messdaten zur Wärmepumpe
%  4l Wasser, Wärmekapazität der ´Kupferschlangen`: mkck = 750 J/K
%
%  t[min] T1 [C] p1[bar] T2[C] p2 [bar] N [W]
0	 & 21.7 & 4.0  & 21.7 & 4.1 & 120 \\
1  & 23.0 & 5.0  & 21.7 & 3.2	& 120 \\
2  & 24.3 & 5.5  & 21.6 & 3.4	& 120 \\
3  & 25.3 & 6.0  & 21.5 & 3.5	& 120 \\
4  & 26.4 & 6.0  & 20.8 & 3.5	& 120 \\
5  & 27.5 & 6.0  & 20.1 & 3.4	& 120 \\
6  & 28.8 & 6.5  & 19.2 & 3.3	& 120 \\
7  & 29.7 & 6.5  & 18.5 & 3.2	& 120 \\
8  & 30.9 & 7.0  & 17.7 & 3.2	& 120 \\
9  & 31.9 & 7.0  & 16.9 & 3.0	& 120 \\
10 & 32.9 & 7.0  & 16.2 & 3.0	& 120 \\
11 & 33.9 & 7.5  & 15.5 & 2.9	& 120 \\
12 & 34.8 & 7.5  & 14.9 & 2.8	& 120 \\
13 & 35.7 & 8.0  & 14.2 & 2.8	& 120 \\
14 & 36.7 & 8.0  & 13.6 & 2.7	& 120 \\
15 & 37.6 & 8.0  & 13.0 & 2.6	& 120 \\
16 & 38.4 & 8.5  & 12.4 & 2.6	& 120 \\
17 & 39.2 & 8.5  & 11.7 & 2.6	& 120 \\
18 & 40.0 & 9.0  & 11.3 & 2.5	& 120 \\
19 & 40.7 & 9.0  & 10.9 & 2.5	& 120 \\
20 & 41.4 & 9.0  & 10.4 & 2.4	& 120 \\
21 & 42.2 & 9.0  & 9.9  & 2.4	& 120 \\
22 & 42.9 & 9.5  & 9.5  & 2.4	& 120 \\
23 & 43.6 & 9.5  & 9.1  & 2.4	& 120 \\
24 & 44.3 & 0.0	 & 8.7  & 2.4	& 120 \\
25 & 44.9 & 0.0	 & 8.3  & 2.4	& 120 \\
26 & 45.5 & 0.0	 & 8.0  & 2.3	& 120 \\
27 & 46.1 & 0.0	 & 7.7  & 2.2	& 122 \\
28 & 46.7 & 0.5	 & 7.4  & 2.2	& 122 \\
29 & 47.3 & 0.5	 & 7.1  & 2.2	& 122 \\
30 & 47.8 & 0.75 & 6.8  & 2.2	& 122 \\
31 & 48.4 & 1.0	 & 5.6  & 2.2	& 122 \\
32 & 48.9 & 1.0	 & 4.3  & 2.2	& 122 \\
33 & 49.4 & 1.0	 & 3.4  & 2.2	& 122 \\
34 & 49.9 & 1.0	 & 3.0  & 2.2	& 122 \\
35 & 50.3 & 1.0	 & 2.9  & 2.2	& 122 \\
\bottomrule
\end{tabular}
\end{table}


\newpage
\subsection{Diagramm der gemessenen Temperaturverläufe} % a)
Im Diagramm sind die Approximationen aus Gleichung in Kapitel \ref{sec:approx} bereits enthalten.
Der Graph zeigt die Verläufe der Temperaturen $T_1$ und $T_2$ bei zunehmender Zeit an.

\begin{figure}
  \centering
  \includegraphics[scale=0.8]{build/wärmepumpe_plot.pdf}
  \caption{Temperaturverläufe in Abhängigkeit der Zeit}
  \label{fig:plot}
\end{figure}

\subsection{Approximation der Temperaturverläufe} \label{sec:approx} % b)
Zur Approximation des Temperaturverläufe bietet sich ein Polynom zweiten Grades an:
\[
T(t) = At^2 + Bt + C \; .
\]
Als Fit-Parameter wurden berechnet:

\begin{table}
\centering
\caption{Fit-Parameter}
\label{tab:fit_params}
% \sisetup{table-format=2.1}
\begin{tabular}{c c c c}
\toprule
 & A & B & C \\
\midrule
% $T_1$ &
% \num{-3.2249e-06} ± \num{4.1934e-08} &
% \num{0.02028} ± \num{9.1105e-05} &
% \num{294.97} ± \num{0.041345}\\
% $T_2$ &
% \num{9.5504e-07} ± \num{2.6711e-07} &
% \num{-0.011209} ± \num{0.00058032} &
% \num{295.87} ± \num{0.26336}\\
\input{../build/table_polyfit.tex}
\bottomrule
\end{tabular}
\end{table}

\subsection{Berechnung von Differentialquotienten} % c)
Mithilfe der Approximation aus der Gleichung in Kapitel \ref{sec:approx} werden nun für vier verschiedene Temperaturen konkrete  Differentialquotienten $\frac{\mathrm{d}T_1}{\mathrm{d}t}$ und $\frac{\mathrm{d}T_2}{\mathrm{d}t}$ berechnet.
Die Ableitung des Polynoms lautet
\[
\frac{\mathrm{d}T}{\mathrm{d}t} = 2At + B \; .
\]
\\
Fehlerrechnung:
\begin{align*}
  \Delta \frac{\mathrm{d}T}{\mathrm{d}t}
  &= \sqrt{\left(\frac{\partial \frac{\mathrm{d}T}{\mathrm{d}t}}{\partial A} \cdot \Delta A\right)^2 + \left(\frac{\partial \frac{\mathrm{d}T}{\mathrm{d}t}}{\partial B} \cdot \Delta B\right)^2} \\
  &= \sqrt{(2t \cdot \Delta A)^2 + (\Delta B)^2} .
\end{align*}

\begin{table}
\centering
\caption{Ableitungen der Approximationsfunktion}
\label{tab:derivatives}
\sisetup{table-format=2.1}
\begin{tabular}{c c c c c c}
\toprule
$t \,/\, \si{\minute}$ &
$\frac{\mathrm{d}T_1}{\mathrm{d}t} \,/\, \si{\celsius\per\minute}$ &
$\frac{\mathrm{d}T_2}{\mathrm{d}t} \,/\, \si{\celsius\per\minute}$ \\
\midrule
7  & \num{0.017571}  ± \num{0.000098}  & \num{-0.010406}  ± \num{0.000622} \\
14 & \num{0.014862}  ± \num{0.000115}  & \num{-0.0096042} ± \num{0.0007336} \\
21 & \num{0.012153}  ± \num{0.000140}  & \num{-0.0088020} ± \num{0.0008887} \\
28 & \num{0.0094441} ± \num{0.0001678} & \num{-0.0079998} ± \num{0.0010688} \\

% 7  & \num{1.757e-02} ± \num{0.010e-02} & \num{-1.041e-02} ± \num{0.062e-02} \\
% 14 & \num{1.486e-02} ± \num{0.012e-02} & \num{-9.604e-03} ± \num{0.734e-03} \\
% 21 & \num{1.215e-02} ± \num{0.014e-02} & \num{-8.802e-03} ± \num{0.889e-03} \\
% 28 & \num{9.444e-03} ± \num{0.168e-03} & \num{-8.000e-03} ± \num{1.069e-03} \\


\bottomrule
\end{tabular}
\end{table}


\subsection{Bestimmung der Güteziffern} % d)
\label{sec:auswertung_gueteziffern}
% In "Daten und Hinweise" ist lediglich die Rede vom *Fassungsvermögen*!
% „Fassungsvermögen der Wassereimer: m = 4kg“
% Als Füllmenge nehmen wir daher 4L an, wie es in "Daten.dat" steht

% „Wärmekapazität der Kupferschlangen“ meint *Wärmekapazität jeder Kupferschlange*, oder?
Jeder Wassereimer war mit $\SI{4}{\liter}$ Wasser befüllt, bei einem Fassungsvermögen von $\SI{4}{\kilo\gram}$.
Die Wärmekapazität der \enquote{Kupferschlangen} war gegeben als $m_k c_k = 750 \si{\joule\per\kelvin}$. Laut Versuchsanleitung soll $m_k c_k$ auch die Wärmekapazität der Eimer enthalten, dazu liegen jedoch keine weiteren Daten vor.
Die Wärmekapazität des Wassers in Reservoir 1 lässt sich mit
\begin{align*}
  \rho_\text{Wasser} &= \SI{0.998207}{\kilo\gram\per\liter}
  \tag*{(Wasserdichte bei 20°C \cite{wasserdichte})} \\
  \\
  c_\text{Wasser} &= \SI{4.1851}{\joule\per\gram\and\kelvin}
  \tag*{(spezifische Wärmekapazität von Wasser bei 20°C \cite{wasserwaermekapazitaet})} \\
  \\
  C_{\text{Wasser}, 1} &= m_1 \cdot c_\text{Wasser} \\
  &= (\rho_\text{Wasser} \cdot V_\text{Wasser}) \cdot c_\text{Wasser} \\
  &= \left(\SI{0.998207}{\kilo\gram\per\liter} \cdot \SI{4}{\liter}\right) \cdot \SI{4.181}{\joule\per\gram\and\kelvin} \\
  &= \SI{16736.0}{\joule\per\kelvin} \\
\end{align*}
bestimmen, sodass aus einfacher Summation die gesamte Wärmekapazität
\[
C_\text{ges} = 750 \si{\joule\per\kelvin} + \SI{16736.0}{\joule\per\kelvin} = \SI{17486.0}{\joule\per\kelvin}
\]
folgt.

\ \\
Unter Zuhilfenahme von Gleichung \ref{eqn:reale_gueteziffer} und der Differentialquotienten aus Tabelle \ref{tab:derivatives} kann nun die reale Güteziffer für verschiedene $t$ berechnet werden.

\begin{table}
\centering
\caption{Vergleich der idealen und realen Güteziffern}
% \sisetup{table-format=2.1}
\begin{tabular}{c c c c c c}
\toprule
$t \,/\, \si{\minute}$ &
$\nu_\text{ideal}$ &
$\nu_\text{real}$ \\
% TODO: Abweichung in % dazu?
\midrule
7  & 27.04 ± 0.34  & 2.560 ± 0.014 \\
14 & 13.41 ± 0.08  & 2.166 ± 0.017 \\
21 & 9.76  ± 0.04  & 1.771 ± 0.020 \\
28 & 8.139 ± 0.028 & 1.354 ± 0.024 \\
\bottomrule
\end{tabular}
\end{table}

\subsection{Bestimmung des Massendurchsatzes} % e)
% e) Man errechne für die 4 Temperaturen aus 5c den Massendurchsatz des hier benutzten Transportgases Dichlordifluormethan (Cl 2 F 2 C). Die Verdampfungswärme L dieses Stoffes kann man aus seiner Dampfdruck-Kurve gewinnen. (Näheres hierzu findet man in V203.) Wertepaare (p,T) zur Darstellung dieser Kurve können an den Manometern der Apparatur abgelesen werden. Mit den dort abgelesenen (p, T)-Werten führe man eine Ausgleichsrechnung zur Bestimmung von L durch.

% TODO: Verschieben zur Theorie → 1.1.2

! Dieser Abschnitt ist \textbf{work-in-progress} !

universelle Gaskonstante $R$

\begin{align*}
  \frac{1}{p} \mathrm{d}p &= \frac{L}{R \cdot T^2} \mathrm{d} T \\
  \mathrm{ln}(p) &= \frac{L}{R} \cdot \frac{1}{T} + c \\
  \mathrm{ln}(p) &= \frac{L}{R \cdot T} + c \\
  \mathrm{ln}(p) \cdot (R \cdot T) &= L + c \\
\end{align*}

\begin{table}
\centering
\caption{Massendurchsatz}
% \sisetup{table-format=2.1}
\begin{tabular}{c c c c c c}
\toprule
$t \,/\, \si{\second}$ &
$\frac{\mathrm{d}T_2}{\mathrm{d}t} \,/\, \SI{e-3}{\kelvin\per\second}$ &
$\frac{\mathrm{d}m}{\mathrm{d}t} \,/\, \si{\gram\per\second}$ \\
% TODO: Abweichung in % dazu?
\midrule
% 7  | deriv: (-1.041+/-0.062)e-02 & -1.54+/-0.09 TODO \\
% 14 | deriv: (-9.604+/-0.734)e-03 & -1.42+/-0.11 TODO \\
% 21 | deriv: (-8.802+/-0.889)e-03 & -1.30+/-0.13 TODO \\
% 28 | deriv: (-8.000+/-1.069)e-03 & -1.18+/-0.16 TODO \\
\input{../build/table_massendurchsatz.tex}
\bottomrule
\end{tabular}
\end{table}


\subsection{Bestimmung der Kompressorleistung} % f)
Die mechanische Leistung des Kompressors, die dieser abgibt, wenn
er zwischen den Drücken $p_1$ und $p_2$ arbeitet,
wird mithilfe von Gleichung \ref{eqn:N_mech}, der gegebenen Daten, und der Differentialquotienten aus Tabelle \ref{tab:derivatives} berechnet.
\ \\
Daten für $\mathrm{Cl}_2 \mathrm{F}_2 \mathrm{C}$: \\
% \[
$ρ_0 = \SI{5.51} g/l$ bei $T = \SI{0}{\celsius}$ und $p = \SI{1}{\bar}$, $κ = 1.14$
% \]

\begin{table}
\centering
\caption{Kompressorleistung}
% \sisetup{table-format=2.1}
\begin{tabular}{c c c c c c}
\toprule
% $t \,/\, \si{\second}$ &
% $\frac{\mathrm{d}T_2}{\mathrm{d}t} \,/\, \SI{e-3}{\kelvin\per\second}$ &
% $\frac{\mathrm{d}m}{\mathrm{d}t} \,/\, \si{\gram\per\second}$ \\

$t \,/\, \si{\second}$ &
% $\rho \,/\, \si{\kilo\gram\per\meter³}$ &
$ N_\text{mech} \,/\, \si{\watt}$ \\
\midrule
\input{../build/table_kompressorleistung.tex}
\bottomrule
\end{tabular}
\end{table}
