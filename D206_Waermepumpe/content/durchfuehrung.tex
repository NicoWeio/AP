\section{Durchführung} \label{sec:Durchführung}
    Zu Beginn des Experiments werden die Reservoire aus Kapitel \ref{sec:Theorie} mit einer abgemessenen Wassermenge (hier: $\SI{4}{\kilogram}$) befüllt.
    Anschließend werden im Abstand von $\Delta t = \SI{1}{\minute}$ die Temperaturen $T_1$ und $T_2$, die Drücke $p_1$ und $p_2$, und die Leistungsaufnahme des Kompressors $N$ gemessen.
    Auf den abgelesen Wert des Drucks muss zusätzlich ein Wert von $\SI{1}{\bar}$ addiert werden.
    % Um die Zeitintervalle möglichst genau einzuhalten, sollten die Werte in einer festen Reihenfolge abgelesen werden.
    Bei einer Temperatur von $\SI{50}{\celsius}$ soll die Messung abgebrochen werden.
