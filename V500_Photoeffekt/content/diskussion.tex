\section{Diskussion} \label{sec:diskussion}

\subsection{Abweichungen}

$\frac{h}{e}$ wurde experimentell zu $\SI{3.841(14)e-15}{\volt\second}$ bestimmt.
Der Theoriewert beträgt $\SI{4.136e-15}{\volt\second}$ \cite{e}\cite{h},
was einer Abweichung von $\SI{7.11(34)}{\percent}$ entspricht.
Die sehr gute Genauigkeit der in \autoref{fig:plot_nu_ug} dargestellten Regressionsgerade,
beziehungsweise die geringe Standardabweichung,
deuten darauf hin, dass ein unbekannter, geringer systematischer Fehler vorliegen könnte.

\subsection{Mögliche Fehlerquellen}

Es ist nochmals zu betonen,
wie instabil die Verbindung der Photozelle mit dem Picoamperemeter war.
Selbst Berührungen des Tischs, auf dem sich der Versuchsaufbau befand,
konnten zur Folge haben, dass kein Strom mehr gemessen wurde.
Auch hätte die Photozelle versehentlich verschoben werden können,
da keine Möglichkeit bestand, diese auf ihrer Schiene zu fixieren.

Die Höhe und Breite der untersuchten Spektrallinien entsprachen nicht genau
der Größe des Spalts der Photozelle
und das Bild war nicht perfekt fokussiert.
Daraus ergibt sich eine geringere Intensität an der Photokathode,
welche zu kleineren Strömen und letztlich zu größeren Fehlern geführt haben dürfte.

Wegen der hohen Empfindlichkeit des Picoamperemeters waren teilweise Schwankungen sichtbar,
die das Ablesen erschwerten.
In diesen Fällen wurde ein Mittelwert nach Augenmaß notiert.

Möglicherweise mitverantwortlich für diese Ausschläge war die Spannungsquelle,
welche laut der integrierten Digitalanzeige um ca. $\pm\SI{0.01}{\volt}$ fluktuierte.

Es soll nicht unerwähnt bleiben,
dass auch der \hyperref[sec:durchfuehrung:dunkelstrom]{Dunkelstrom} gemessen wurde,
welcher jedoch mit $\SI{0.001}{\nano\ampere}$ kaum messbar war
und so gering ausfiel,
dass ihm keine weitere Beachtung geschenkt wurde.
