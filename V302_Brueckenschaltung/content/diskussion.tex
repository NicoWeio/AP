\section{Diskussion} \label{sec:Diskussion}

Wie anhand der relativ geringen Unsicherheiten bei den berechneten Werten zu sehen ist, sind Brückenschaltungen eine gute Möglichkeit, um unbekannte Bauteile auszumessen.
Es zeigte sich jedoch auch, dass die bekannten Bauteile einen entscheidenen Einfluss auf die Genauigkeit der Ergebnisse haben.
Beispielsweise wurden für die \hyperref[sec:AufgabeA]{Wheatstonesche Brückenschaltung} zwei verschiedene Potentiometer verwendet, die zwar in sich stimmige Werte lieferten, jedoch die vom Hersteller angegebene Toleranz von $\pm 3\%$ scheinbar voll ausnutzten.
Dennoch: Die mit dem simplen Aufbau verbundene geringe Anzahl von Komponenten bedeutet auch eine geringe Anzahl an (potentiellen) Fehlerquellen.

Erwähnenswert ist noch der menschliche Fehler bei \hyperref[sec:AufgabeE]{Aufgabenteil e)}, da das Justieren zweier voneinander abhängigen Potentiometer, die zudem eher ungeeignet für eine Feinabstimmung waren, mehr Versuch und Irrtum glich. Dennoch zeugt die gute Übereinstimmung mit den Theorie-Werten von einem erfolgreichen Experiment.
