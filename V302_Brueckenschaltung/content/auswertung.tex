\section{Auswertung} \label{sec:Auswertung}
\subsection{a)}
% Zunächst sollten zwei unbekannte Ohm'sche Widerstände mithilfe der Wheatstoneschen Brückenschaltung ausgemessen werden.
% Aber wir nur einen vermessen – was wir besser nicht erwähnen ;)
Zunächst sollte ein unbekannter Ohm'sche Widerstand (\enquote{Wert 13}) mithilfe der \hyperref[sec:Wheatstone]{Wheatstoneschen Brückenschaltung} ausgemessen werden.
Für die Linearität $\frac{R_3}{R_4}$ gilt eine unsystematische Abweichung von $\pm 0.5 \%$.
Für den festen Widerstand $R_2$ war dagegen keine Abweichung angegeben, sodass die berechnete Abweichung für $R_x$ kaum aussagekräftig ist.

Die nachstehende Tabelle ist in je zwei Zeilen untergliedert, weil der Versuch mit einem anderen Potentiometer wiederholt wurde.

\begin{table}
\centering
\caption{TODO}
\label{tab:todo1}
% \sisetup{table-format=2.1}
\begin{tabular}{c c c c}
\toprule
$R_2$ \,/\, \si{\ohm} &
$R_3$ \,/\, \si{\ohm} &
$R_4$ \,/\, \si{\ohm} &
$R_x$ \,/\, \si{\ohm} \\
\midrule
664	& 317 &	683	&	308.182 \pm 1.541 \\
332	& 483 &	517	&	310.166 \pm 1.551 \\
\midrule
332	& 488 &	512	&	316.438 \pm 1.582 \\
664	& 323 &	677	&	316.798 \pm 1.584 \\
\bottomrule
\end{tabular}
\end{table}

Damit ist $R_x$ zu $312.896 \pm 0.782$ bestimmt.

\subsection{b)}
\label{sec:AufgabeB}
Mithilfe der \hyperref[sec:Kapazität]{Kapazitätsmessbrücke} sollte eine unbekannte Kapazität (\enquote{Wert 9}) bestimmt werden.
Die Toleranz (Eichgenauigkeit) des Verlustwiderstands beträgt hier $\pm 3\%$.
Wie zuvor wird die Abweichung für $\frac{R_3}{R_4}$ berücksichtigt.

\begin{table}
  \centering
  \caption{TODO}
  \label{tab:todo2}
  % \sisetup{table-format=2.1}
  \begin{tabular}{c c c c c c}
    \toprule
    $C_2$ \,/\, \si{\nano\farad} &
    $R_2$ \,/\, \si{\ohm} &
    $R_3$ \,/\, \si{\ohm} &
    $R_4$ \,/\, \si{\ohm} &
    $C_x$ \,/\, \si{\nano\farad} &
    $R_x$ \,/\, \si{\ohm} \\
    \midrule
    750 &	267   & 630 & 370 & 440.476 \pm 2.202 & 454.622 \pm 13.827 \\
    450 & 438.5 & 508 & 492 & 435.827 \pm 2.179 & 452.760 \pm 13.770 \\
    \bottomrule
  \end{tabular}
\end{table}

\ \\
Die gemittelten Werte sind $C_x = \SI{438.151 \pm 1.549}{\nano\farad}$ und $R_x = \SI{453.691 \pm 9.757}{\ohm}$.

\subsection{c)}
Mit der \hyperref[sec:Induktivität]{Induktivitätsbrücke} soll eine unbekannte Induktivität (\enquote{Wert 17}) berechnet werden.
Die Toleranz des Verlustwiderstands entspricht der aus \hyperref[sec:AufgabeB]{b)}.

\begin{table}
  \centering
  \caption{TODO}
  \label{tab:todo3}
  % \sisetup{table-format=2.1}
  \begin{tabular}{c c c c c c}
    \toprule
    $L_2$ \,/\, \si{\milli\henry} &
    $R_2$ \,/\, \si{\ohm} &
    $R_3$ \,/\, \si{\ohm} &
    $R_4$ \,/\, \si{\ohm} &
    $L_x$ \,/\, \si{\milli\henry} &
    $R_x$ \,/\, \si{\ohm} \\
    \midrule
    27.5 & 57 & 605.5 & 394.5 & 87.49 & 42.21 \\
    14.6 & 33 & 740.5 & 259.5 & 94.17 & 41.66 \\
    \bottomrule
  \end{tabular}
\end{table}

\subsection{d)}
Hier soll der \enquote{Wert 17} erneut berechnet werden, allerdings mit einer \hyperref[sec:Maxwell]{Maxwell-Brücke}.
Die Toleranz für $R_3$ und $R_4$ beträgt $\pm 3\%$.

Wert 17

\begin{table}
  \centering
  \caption{TODO}
  \label{tab:todo4}
  % \sisetup{table-format=2.1}
  \begin{tabular}{c c c c c c}
    \toprule
    $R_2$ \,/\, \si{\ohm} &
    $R_3$ \,/\, \si{\ohm} &
    $R_4$ \,/\, \si{\ohm} &
    $C_4$ \,/\, \si{\nano\farad} &
    $R_x$ \,/\, \si{\ohm} &
    $L_x$ \,/\, \si{\milli\henry} \\
    \midrule
    664	&  81.5	& 608  & 750 & 89.006 & 40.587  \\
    664	& 137.5	& 1003 & 450 & 91.026 & 41.085  \\
    332	& 277.5	& 1003 & 450 & 91.854 & 41.459  \\
    \bottomrule
  \end{tabular}
\end{table}

\subsection{e)}
In dieser Messreihe wurde die Abhängigkeit der Brückenspannung von der Frequenz untersucht.
Das folgende Diagramm zeigt den Verlauf der Spannung.
Auf der x-Achse ist $\Omega = \frac{v}{v_0}$ aufgetragen, wobei $v_0$ die Frequnz dargestellt, bei
der die Brückenspannung $U_\text{Br}$ miminal wird.
Auf der y-Achse ist der Quotient $\frac{U_\text{Br}}{U_\text{S}}$ dargestellt.


\subsection{f)}
In diesem Aufgabenteil soll der sogenannte Klirrfaktor $k$ nach der Formel
\begin{equation}
     k := \frac{\sqrt{U_2^2 + U_3^2 + ...}}{U_1}
\end{equation}
berechnet werden.
Dazu gilt
\begin{equation}
    U_2 = \frac{U_\text{Br}}{f(2)} \; .
\end{equation}
