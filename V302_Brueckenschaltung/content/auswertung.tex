\section{Auswertung} \label{sec:Auswertung}
\subsection{a)}
Zunächst sollten zwei unbekannte Ohm'sche Widerstände mithilfe der Wheatstoneschen Brückenschaltung ausgemessen werden.

Wert 13

\begin{table}
\centering
\caption{TODO}
\label{tab:todo1}
% \sisetup{table-format=2.1}
\begin{tabular}{c c c c}
\toprule
$R_2$ \,/\, \si{\ohm} &
$R_3$ \,/\, \si{\ohm} &
% TODO: Trennung…
$R_4$ \,/\, \si{\ohm} &
$R_x$ \,/\, \si{\ohm} \\
\midrule
664	& 317 &	683	&	308,18 \\
332	& 483 &	517	&	310,17 \\
332	& 488 &	512	&	316,44 \\
664	& 323 &	677	&	316,80 \\
\bottomrule
\end{tabular}
\end{table}

\subsection{b)}

Wert 9

\begin{table}
  \centering
  \caption{TODO}
  \label{tab:todo2}
  % \sisetup{table-format=2.1}
  \begin{tabular}{c c c c c c}
    \toprule
    $C_2$ \,/\, \si{\nano\farad} &
    $R_3$ \,/\, \si{\ohm} &
    $R_4$ \,/\, \si{\ohm} &
    $R_2$ \,/\, \si{\ohm} &
    $C_x$ \,/\, \si{\nano\farad} &
    $R_x$ \,/\, \si{\ohm} \\
    \midrule
    750 &	630 &	370 &	267		& 440.48 & 454.62 \\
    450 &	508 &	492 &	438.5	&	435.83 & 452.76 \\
    \bottomrule
  \end{tabular}
\end{table}


\subsection{c)}

Wert 17


\subsection{d)}

Wert 17

\subsection{e)}


\subsection{f)}
