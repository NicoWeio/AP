\section{Auswertung} \label{sec:Auswertung}
\subsection{a)} %Ungenauigkeit des Potentiometers?
Zunächst sollten zwei unbekannte Ohm'sche Widerstände mithilfe der Wheatstoneschen Brückenschaltung ausgemessen werden.
Für die Linearität $R_3/R_4$ gilt eine unsystematische Abweichung von $\pm 0.5 \%$.

Wert 13

\begin{table}
\centering
\caption{TODO}
\label{tab:todo1}
% \sisetup{table-format=2.1}
\begin{tabular}{c c c c}
\toprule
$R_2$ \,/\, \si{\ohm} &
$R_3$ \,/\, \si{\ohm} &
% TODO: Trennung…
$R_4$ \,/\, \si{\ohm} &
$R_x$ \,/\, \si{\ohm} \\
\midrule
664	& 317 &	683	&	308,18 \\
332	& 483 &	517	&	310,17 \\
332	& 488 &	512	&	316,44 \\
664	& 323 &	677	&	316,80 \\
\bottomrule
\end{tabular}
\end{table}

\subsection{b)}
Mithilfe der Kapazitätsmessbrücke sollte eine unbekannte Kapazität mit dem Wert 9 bestimmt werden.
Die Toleranz oder auch Eichgenauigkeit des Verlustwiderstands beträgt hier $\pm 3\%$.

Wert 9

\begin{table}
  \centering
  \caption{TODO}
  \label{tab:todo2}
  % \sisetup{table-format=2.1}
  \begin{tabular}{c c c c c c}
    \toprule
    $C_2$ \,/\, \si{\nano\farad} &
    $R_3$ \,/\, \si{\ohm} &
    $R_4$ \,/\, \si{\ohm} &
    $R_2$ \,/\, \si{\ohm} &
    $C_x$ \,/\, \si{\nano\farad} &
    $R_x$ \,/\, \si{\ohm} \\
    \midrule
    750 &	630 &	370 &	267		& 440.48 & 454.62 \\
    450 &	508 &	492 &	438.5	&	435.83 & 452.76 \\
    \bottomrule
  \end{tabular}
\end{table}


\subsection{c)}
Mit der Induktivitätsbrücke soll eine unbekannte Induktivität mit dem Wert 17 berechnet werden.
Die Toleranz des Verlustwiderstands entspricht der aus b).
Wert 17


\subsection{d)}
Hier soll der Wert 17 erneut berechnet werden, allerdings mit einer Maxwell-Brücke.
Die Toleranz für $R_3$ und $R_4$ beträgt $\pm 3\%$.

Wert 17

\subsection{e)}
In dieser Messreihe wurde die Abhängigkeit der Brückenspannung von der Frequenz untersucht.
Das folgende Diagramm zeigt den Verlauf der Spannung.
Auf der x-Achse ist $\Omega = \frac{v}{v_0}$ aufgetragen, wobei $v_0$ die Frequnz dargestellt, bei 
der die Brückenspannung $U_\text{Br}$ miminal wird.
Auf der y-Achse ist der Quotient $\frac{U_\text{Br}}{U_\text{S}}$ dargestellt.


\subsection{f)}
In diesem Aufgabenteil soll der sogenannte Klirrfaktor k nach der Formel
\begin{equation}
     k := \frac{\wurzel{U_2^2 + U_3^2 + ...}}{U_1}
\end{equation}
berechnet werden.
Dazu gilt
\begin{equation}
    U_2 = \frac{U_\text{Br}}{f(2)} .
\end{equation}