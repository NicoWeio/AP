\section{Durchführung} \label{sec:Durchführung}

    Im Folgenden soll die Durchführung der Messungen von verschiedenen Größen mithilfe der oben beschriebenen
    Brückenschaltungen beschrieben werden.
    Dabei werden zu jeder Schaltung mehrere Messungen unter Variation der Bauteile
    $R_2$, $C_2$ und $L_2$ aufgenommen.
    Zu Beginn des Experiments sind eine Speisespannung $U_\text{S}$, ein digitales Oszilloskop, welches die
    Brückenspannung $U_\text{Br}$ anzeigt, verschiedene ohmsche Widerstände, Kapazitäten und Induktivitäten, sowie
    zwei Potentiometer gegeben.

\subsection{Wheatstone'sche Brückenschaltung}

    Zunächst wird die Wheatstone'sche Brückenschaltung anhand der Abbildung in Kapitel \ref{sec:Wheatstone} gebaut.
    Es wird eine konstante Frequenz von $f = \SI{1000}{\hertz}$ eingestellt.

    Die Brücke wird mit Wechselstrom betrieben.
    Anschließend wird mithilfe des Potentiometers, welches einen Gesamtwiderstand von $\SI{1}{\kilo\ohm}$ besitzt,
    die Brückenschaltung abgeglichen, indem
    die Widerstände $R_3$ und $R_4$ variiert werden und $U_\text{Br}$ auf Null geregelt wird.
    Nachdem die Messwerte für $R_2$, $R_3$ und $R_4$ abgelesen wurden, wird mithilfe von Gleichung \eqref{eqn: Rx}
    der unbekannte Widerstand $R_x$ mit dem Wert 13 berechnet.
    Der Widerstand $R_4$ wird durch $\SI{1000}{\kilo\ohm}-R_3$ berechnet, da zwischen den Eingängen des
    Potentiometers ein Unterschied von $\SI{1000}{\kilo\ohm}$ besteht.
    Die Messung wird wiederholt unter der Variation von $R_2$.
    Wir haben die Messung mit einem anderen Potentiometer noch ein weiteres Mal durchgeführt.

\subsection{Kapazitätsmessbrücke}

    Die Schaltung der Kapazitätsmessbrücke wird nach der Abbildung in Kapitel \ref{sec:Kapazität} aufgebaut.
    Die Brücke wird mit Wechselspannung betrieben.
    Die ohmschen Widerstände $R_2$ und $R_3/R_4$ werden bei dieser Schaltung durch Potentiometer eingestellt.
    Um beide Widerstände möglichst gleichmäßig einzustellen, sollten die Potentiometer abwechselnd geregelt werden
    für das Abgleichen der Brücke.
    Wenn die Brückenspannung gleich Null ist, können die Werte der Widerstände abgelesen werden und Wert 9,
    welcher den unbekannten Verlustwiderstand und die unbekannte Kapazität beschreibt, kann mit den Gleichungen 
    \eqref{eqn: Rx} und \eqref{eqn: Cx} identifiziert werden.
    Anschließend wiederholen wir die Messung unter Variation von $C_2$ 

\subsection{Induktivitätsmessbrücke}

    Die Schaltung der Induktivitätsmessbrücke wird nach dem Schaltplan in Kapitel \ref{sec:Induktivität} 
    geschaltet.
    In dieser Schaltung werden die Widerstände $R_2$, $R_3/R_4$ durch die Potentiometer bestimmt, die wieder 
    alternierend zum Abgleichen eingestellt werden.
    Der zu bestimmende Wert ist hier 17. 
    Mit den Gleichungen \eqref{eqn: Rx} und \eqref{eqn: Lx} kann der Verlustwiderstand und die Induktivität
    berechnet werden.
    Wir wiederholen die Messung unter Variation von $L_2$.

\subsection{Maxwell-Brücke}

    Der Wert 17 soll hier noch einmal mithilfe einer Maxwell-Brücke bestimmt werden.
    Wir bauen die Schaltung anhand von Abbildung \ref{sec:Maxwell} auf.
    Der Widerstand $R_2$ ist konstant, während $R_3$ und $R_4$ dieses Mal von zwei getrennten
    Potentiometern bestimmt werden.
    Im Gegensatz zur Kapazitätsmessbrücke variieren wir die Kapazität $C_2$ und haben keine weitere 
    Induktivität in der Schaltung.
    Die unbekannte Induktivität wird mit Gleichung \eqref{sec:LxMax} bestimmt und der unbekannte
    Verlustwiderstand mit Gleichung \eqref{eqn: Rx}.

\subsection{Wien-Robinson-Brücke}

    Zunächst wird die Schaltung nach der Abbildung in Kapitel \ref{sec:WR} aufgebaut.
    Wir benutzen drei konstante Widerstände, wobei $2R'$ der doppelte Wert von $R'$ sein sollte.
    Zusätzlich werden zwei gleiche Kapazitäten benötigt, wobei in unserem Fall nur zwei ähnliche Kapazitäten zur
    Verfügung standen.
    Im ersten Teil der Messung wird eine konstante Speisespannung von $U_\text{S} = \SI{20}{\kilo\ohm}$ eingestellt.
    Ziel ist es, in einem Frequenzspektrum von $\SI{20}{\hertz}$ bis $\SI{30000}{\hertz}$ ein 
    Minimum der Brückenspannung $U_\text{Br}$ zu finden.
    Dazu beginnen wir bei $\SI{20}{\hertz}$ und verdoppeln den Wert, um einen groben Überblick
    über den Verlauf von $U_\text{Br}$ zu bekommen. 
    Im Bereich des Minimums werden in kleineren Abständen der Frequenzen nochmal mehr Messwerte aufgenommen.
    Anschließend wird der grobe Verlauf weiter bis zur oberen Grenze des Spektrums von $\SI{30000}{\hertz}$ gemessen.
    Im zweiten Teil des Messung wird die Speisespannung $U_\text{S}$ auf dem Oszilloskop dargestellt 
    und der Verlauf dieser Spannung in Abhängigkeit der Frequenz bestimmt.
