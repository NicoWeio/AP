\section{Durchführung} \label{sec:Durchführung}

    Im Folgenden soll die Durchführung der Messungen von verschiedenen Größen mithilfe der oben beschriebenen
    Brückenschaltungen beschrieben werden.
    Dabei werden zu jeder Schaltung mehrere Messungen unter Variation der Bauteile
    $R_2$, $C_2$ und $L_2$ aufgenommen.
    Zu Beginn des Experiments sind eine Speisespannung $U_\text{S}$, ein digitales Oszilloskop, welches die
    Brückenspannung $U_\text{Br}$ anzeigt, verschiedene ohmsche Widerstände, Kapazitäten und Induktivitäten, sowie
    zwei Potentiometer gegeben.

\subsection{Wheatstone'sche Brückenschaltung}

  % TODO: Direkt auf die Abbildung verlinken
    Zunächst wird die Wheatstone'sche Brückenschaltung anhand der Abbildung in Kapitel \ref{sec:Theorie} gebaut.
    Es wird eine konstante Speisespannung von $U_\text{S} = $ eingestellt, sowie eine Frequenz von $f = \SI{1000}{\hertz}$.
    Die Brücke wird mit Wechselstrom betrieben.
    Anschließend wird mithilfe des Potentiometers, welcher einen Gesamtwiderstand von $\SI{1}{\kilo\ohm}$ besitzt,
    die Brückenschaltung abgeglichen, indem
    die Widerstände $R_3$ und $R_4$ variiert werden und $U_\text{Br}$ auf Null geregelt wird.
    Nachdem die Messwerte für $R_2$, $R_3$ und $R_4$ abgelesen wurden, wird mithilfe von Gleichung \eqref{eqn: Rx}
    der unbekannte Widerstand $R_x$ berechnet.
    Die Messung wird unter der Variation von $R_2$ wiederholt.


\subsection{Kapazitätsmessbrücke}
\subsection{Induktivitätsmessbrücke}
\subsection{Maxwell-Brücke}
\subsection{Wien-Robinson-Brücke}
