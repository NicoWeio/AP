\section{Diskussion} \label{sec:diskussion}

Die Halbwertszeit von
$\ce{^{52}_{23}V}$
hat einen Literaturwert von
$\SI{224.6}{\second}$ \cite{hwz_vanadium},
wohingegen in \autoref{sec:auswertung:vanadium}
eine Halbwertszeit von $\SI{194 \pm 11}{\second}$ ermittelt wurde.
Das entspricht einer Abweichung von \SI{14}{\percent},
womit der Literaturwert als bestätigt angesehen werden kann,
wenngleich er außerhalb des Konfidenzintervalls liegt.

Der Literaturwert für die Halbwertszeit von $\ce{^{104i}_{45}Rh}$
beträgt \SI{260}{\second} \cite{hwz_rhodium_langsam}.
Der \hyperref[sec:auswertung:rhodium:langsam]{hier ermittelte Wert}
ist hingegen $T_\text{½, langsam} = \SI{212 \pm 26}{\second}$.
Die prozentuale Abweichung beträgt also \SI{14}{\percent}.

Die Halbwertszeit von $\ce{^{104}_{45}Rh}$
wird in der Literatur mit
$\SI{42.3}{\second}$ angegeben \cite{hwz_rhodium_schnell}.
Der \hyperref[sec:auswertung:rhodium:schnell]{im Versuch bestimmte Wert} von
$T_\text{½, schnell} = \SI{36.5 \pm 1.4}{\second}$
stimmt damit bei einer Abweichung von \SI{14}{\percent} wieder ausreichend gut überein.

% Vergleichswerte zu Rhodium:
%
% dormail:
% T_hw1 = 230 ± 50 s
% T_hw2 = 48 ± 1.2 s
%
% chris-topher6:
% λ1 = -0.0032735519265965395
% T_hw1 = 211.7416176992187
% λ2 = -0.004992268343961808
% T_hw2 = 138.8441351311399
%
% dlmsr:
% T_hw1 = 896 ± 1026 s
% T_hw2 = 49.64 ± 2.00 s
