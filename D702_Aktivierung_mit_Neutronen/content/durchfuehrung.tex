\section{Durchführung} \label{sec:durchfuehrung}

Zu Beginn des Versuches wird der Wert der Untergrundrate des Nulleffekts $N_\text{U}$ mehrfach gemessen,
in einem Zeitintervall von $\symup{\Delta}t = \SI{300}{\second}$.
Die empfohlene \enquote{Mindest-Gesamt-Messzeit} der Untergrundrate von $\SI{600}{\second}$
wurde mit umgerechnet $7 \cdot \SI{300}{\second} = \SI{35}{\minute}$ eingehalten.\\
Die Isotopen-Proben sollten kurz vor Beginn der Messung hergestellt werden,
damit möglichst wenige Kerne schon vorher zerfallen.\\
Zur Bestimmung der Halbwertszeit von Vanadium wird die Probe
in den Aktivierungsschacht der Neutronenquelle in \autoref{fig:neutronenquelle} gebracht
und anschließend auf das Geiger-Müller-Zählrohr in \autoref{fig:gmz} gesteckt.
Es wird etwa $\SI{20}{\minute}$ lang gemessen. %??? max. Wert von t bei 1230s
Für die Messung wird ein Zeitintervall von $\symup{\Delta}t = \SI{30}{\second}$ eingestellt.
Die Zählrate wird mithilfe der \autoref{eqn:deltazählrate} bestimmt.\\
Für die Messung der Rhodium-Probe wird das Vorgehen wiederholt,
diesmal mit einem Zeitintervall von $\symup{\Delta}t = \SI{15}{\second}$.
Es wird etwa $\SI{10}{\minute}$ lang gemessen. %??? max. Wert von t bei 630s
Die Zählrate wird wieder mithilfe der \autoref{eqn:deltazählrate} bestimmt.\\
Die Halbwertszeit der beiden Isotope kann mit \autoref{eqn:Halbwertszeit} berechnet werden.
