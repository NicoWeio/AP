\section{Durchführung} \label{sec:durchfuehrung}

Zu Beginn des Versuches wird der Wert der Untergrundrate des Nulleffekts $N_\text{U}$ mehrfach gemessen,
in einem Zeitintervall von $\symup{\Delta}t = \SI{300}{\second}$.
Insgesamt ist es sinnvoll, 
über einen Zeitraum von mindestens $\SI{500}{\second}$ bis $\SI{600}{\second}$ zu messen.\\
Die Isotopen-Proben sollten kurz vor Beginn der Messung hergestellt werden,
damit möglichst wenige Kerne schon vorher zerfallen.\\
Zur Bestimmung der Halbwertszeit von Vanadium wird die Probe in den Aktivierungsschacht der Neutronenquelle in Abbildung 
%\ref{fig:neutronenquelle}
gebracht und anschließend auf das Geiger-Müller-Zählrohr in Abbildung %ref{fig:gmz}
gesteckt.
Es wird etwa $\SI{20}{\minute}$ lang gemessen. %??? max. Wert von t bei 1230s
Für die Messung wird ein Zeitintervall von $\symup{\Delta}t = \SI{30}{\second}$ eingestellt.
Die Zählrate wird mithilfe der Gleichung \eqref{eqn:deltazählrate} bestimmt.\\
Für die Messung der Rhodium-Probe wird das Vorgehen wiederholt,
diesmal mit einem Zeitintervall von $\symup{\Delta}t = \SI{15}{\second}$.
Es wird etwa $\SI{10}{\minute}$ lang gemessen. %??? max. Wert von t bei 630s
Die Zählrate wird mithilfe der Gleichung \eqref{eqn:deltazählrate} bestimmt.\\
Die Halbwertszeit der beiden Isotope kann mit Gleichung \eqref{eqn:Halbwertszeit} berechnet werden.