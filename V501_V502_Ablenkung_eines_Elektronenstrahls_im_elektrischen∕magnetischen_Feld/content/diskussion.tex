\section{Diskussion}
\label{sec:diskussion}

Zuletzt soll eine Diskussion der durchgeführten Experimente erfolgen.

\subsection{Abweichungen}

\newcommand{\dings}[2]{\hyperref[#1]{\textbf{#2}}\\}

\subsubsection*{Ablenkung im elektrischen Feld}
\dings{sec:auswertung:501:empfindlichkeit}{Linearer Zusammenhang zwischen Ablenkspannung und Leuchtpunktverschiebung}
Die Regressionsrechnungen zu Empfindlichkeit und Apparaturkonstante des Kathodenstrahl-Oszillographen
lieferten gute Werte,
wie an den Plots (\ref{fig:a_501}, \ref{fig:c_501})
und den relativ geringen Unsicherheiten zu erkennen ist.

\dings{sec:auswertung:501:apparaturkonstante}{Bestimmung der Apparaturkonstante}
Die Apparaturkonstante weicht um $\SI{-11.53}{\percent}$ vom theoretisch bestimmten Wert ab.

Dabei ist zu beachten,
dass auch der Theoriewert nur eine Näherung darstellt,
weil beispielsweise das elektrische Feld zwischen den Ablenkplatten nicht homogen ist.
Tatsächlich sind diese auch nicht über ihre gesamte Länge parallel zueinander.

Die prozentuale Abweichung sei hier und im Folgenden so definiert:
\[ \frac{x_\text{exp} - x_\text{theo}}{x_\text{theo}} \ . \]

\dings{sec:auswertung:501:frequenz}{Wechselstromfrequenz des Sinusgenerators}
Die Messung der Frequenz des Sinusgenerators
konnte aufgrund der technischen Gegebenheiten zwar nicht genau wie geplant durchgeführt werden,
die aufgenommenen Messwerte sind jedoch optimal.

Für die Spannungsamplitude liegt kein Vergleichswert vor.

\subsubsection*{Ablenkung im magnetischen Feld}
\dings{sec:auswertung:502:spezifische_elektronenladung}{spezifische Ladung der Elektronen}
Die Unsicherheiten der experimentell bestimmten spezifischen Ladung wurden bereits in \autoref{tab:a_502} angegeben.
Erwähnenswert ist jedoch,
dass die jeweils bestimmte spezifische Ladung stets kleiner als der Theoriewert ist.
Es ist daher von einem systematischen Fehler auszugehen.
Der Mittelwert $\SI{1.613(24) e11}{\coulomb\per\kilo\gram}$ weicht entsprechend um $\SI{-8.31}{\percent}$ vom Theoriewert
$\SI{1.759 e11}{\coulomb\per\kilo\gram}$ ab.

\dings{sec:auswertung:502:erdmagnetfeld}{Bestimmung des lokalen Erdmagnetfelds}
Der Literaturwert für die magnetische Flussdichte in Deutschland
liegt bei etwa $\SI{49}{\micro\tesla}$. \cite{erdmagnetfeld}
Experimentell bestimmt wurde ein Wert von $\SI{36.79(95)}{\micro\tesla}$.
Der Inklinationswinkel $\varphi$ müsste $\SI{37.40}{\degree}$ statt $\SI{54}{\degree}$ betragen,
damit diese Abweichung vom Literaturwert genau ausgeglichen wird.
Tatsächlich sollte er aber bei etwa $\SI{60}{\degree}$ liegen.

\subsection{Mögliche Fehlerquellen}

Zunächst soll auf diejenigen Fehlerquellen,
welche in beiden Versuchen relevant sind,
eingegangen werden.

So gab es an verschiedenen Stellen Probleme beim Ablesen:
Insbesondere bei geringen Beschleunigungsspannungen konnte der Elektronenstrahl
nicht präzise fokussiert werden,
was einen gewissen Interpretationsspielraum beim Zentrieren auf eine Skalenlinie ließ.


\subsubsection*{Ablenkung im elektrischen Feld}

Auch der für die \hyperref[sec:auswertung:501:frequenz]{Bestimmung einer Sinusfrequenz}
benötigte Frequenzzähler ließ sich nicht genau ablesen,
weil sich die angezeigte Frequenz sehr häufig aktualisierte.


\subsubsection*{Ablenkung im magnetischen Feld}

Es war vorgesehen,
dass der Versuchsaufbau für die Messung der spezifischen Elektronenladung
parallel zur Horizontalkomponente des Erdmagnetfelds ausgerichtet wird.
Dies ließ sich jedoch nur in grober Näherung realisieren,
da das verwendete Deklinatorium-Inklinatorium
zu einer genauen Messung nicht geeignet war.
Entsprechend unpräzise war auch die Messung des Inklinationswinkels.

Der Inkrementgeber der an das Helmholtzspulenpaar angeschlossenen Stromquelle
ließ sich nicht präzise einstellen;
die angezeigten Werte fluktuierten etwas.
