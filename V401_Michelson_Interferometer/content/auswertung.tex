\section{Auswertung}

\subsection{Bestimmung der Wellenlänge des Lasers}
\label{sec:auswertung:wellenlaenge}

Zunächst soll berechnet werden,
welche Änderung der optischen Weglänge $\Delta$
aus dem Verstellen der Mikrometerschraube um $\SI{5}{\milli\meter}$ resultiert.
Dazu muss beachtet werden,
dass diese über einen Untersetzungsgebel mit Übersetzung $\num{5.017}$
mit dem zu verschiebenden Spiegel verbunden ist.
Außerdem muss berücksichtigt werden,
dass das Licht die Strecke zweimal (hin und zurück) passiert.
Daher ergibt sich
\begin{equation*}
  \Delta = 2 \cdot \frac{\SI{5}{\milli\meter}}{\num{5.017}} = \SI{1.993}{\milli\meter} \; .
\end{equation*}
% TODO: Abgleichen mit Theorie-Kapitel

\begin{table}[H]
  \centering
  \caption{Gezählte Intensitätsmaxima für jeden Messdurchlauf.}
  \label{tab:messwerte1}
  \begin{tabular}{c}
  \toprule
  $z$ \\
  \midrule
  3184 \\
  3028 \\
  3177 \\
  3137 \\
  3179 \\
  3154 \\
  3183 \\
  3158 \\
  3186 \\
  3150 \\
  \bottomrule
  \end{tabular}
\end{table}

Wird über die in \autoref{tab:messwerte1} angegebenen Zählwerte gemittelt,
ergibt sich $\bar{z} = \num{3153.6(449)}$.
Mithilfe von \autoref{eqn:Wellenlänge} berechnet sich dann die Wellenlänge des benutzten Lasers zu
% ↑ ($\lambda = \frac{\Delta}{z}$)
$\SI{632(9)}{\nano\meter}$.
% TODO: mehr Stellen?

\subsection{Bestimmung des Brechungsindex von Luft}
\label{sec:auswertung:brechungsindex}

Die Normalbedingungen sind durch
\begin{align*}
  p_0 &= \SI{1.0132}{\bar} \\
  T_0 &= \SI{273.15}{\kelvin} \\
\end{align*}
gegeben \cite{versuchsanleitung};
die Umgebungstemperatur wird auf $T = \SI{293.15}{\kelvin}$ festgelegt.

\begin{table}[H]
  \centering
  \caption{Gezählte Intensitätsmaxima mit Angabe der Richtung der Druckänderung.}
  \label{tab:messwerte2}
  \def\arraystretch{1.2}
  \begin{tabular}{r c}
  \toprule
  Druckänderung ($\SI{0.6}{\bar}$) &
  $z$ \\
  \midrule
  $\downarrow$ & 27 \\
  $\uparrow$   & 33 \\
  $\downarrow$ & 16 \\
  $\uparrow$   & 34 \\
  $\downarrow$ & 24 \\
  $\uparrow$   & 33 \\
  $\downarrow$ & 23 \\
  $\uparrow$   & 34 \\
  $\downarrow$ & 26 \\
  $\uparrow$   & 33 \\
  \bottomrule
  \end{tabular}
\end{table}

Der Mittelwert für die in \autoref{tab:messwerte2} angegebenen Zählwerte beträgt $\num{28.3(58)}$.
Mithilfe von \autoref{eqn:brechungsindexunterschied} wird die Brechungsindexänderung damit
zu $\symup{\Delta}n = \num{0.00018(4)}$ bestimmt.
Für diese Berechnung wurde die angegebene Wellenlänge von $\SI{635}{\nano\meter}$,
nicht die zuvor experimentell bestimmte,
verwendet.
Aus \autoref{eqn:brechungsindex} kann schließlich der Brechungsindex berechnet werden.
Es ergibt sich $n = \num{1.00033(7)}$.

Auffällig ist die Abweichung zwischen den Messwerten bei Erhöhung und Verminderung des Drucks.
Es liegt daher nahe,
dass ein systematischer Fehler vorliegt.
Darauf soll in \autoref{sec:diskussion:fehlerquellen} näher eingegangen werden.
Werden die Werte mit negativer Druckänderung ($\downarrow$) weggelassen,
ergibt sich
die Brechungsindexänderung zu $\symup{\Delta}n = \num{0.000147(25)}$,
der Brechungsindex zu $n = \num{1.00027(4)}$,
und eine deutlich geringere (relative) Abweichung vom Literaturwert.
