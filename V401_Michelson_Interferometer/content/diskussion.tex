\section{Diskussion}

\subsection{Abweichungen}
Insgesamt lieferte die Auswertung gute bis sehr gute Werte mit minimalen Abweichungen.

Die in \autoref{sec:auswertung:wellenlaenge} bestimmte Wellenlänge des Lasers
$\SI{632(9)}{\nano\meter}$
unterscheidet sich um $\SI{2.953(8989)}{\nano\meter}$
von der Herstellerangabe von $\SI{635}{\nano\meter}$,
welche auf diesem angebracht ist.
Das ist ein gutes Ergebnis und beweist die hohe Präzision und Genauigkeit des genutzten Versuchsaufbaus.
Dies ist darauf zurückzuführen,
dass es keine bedeutsamen Quellen systematischer Fehler gibt;
das mehrfache Zählen vieler Intensitätsmaxima
(hier rund $\num{30000}$ in $\num{10}$ Durchgängen)
minimiert den statistischen Fehler dabei nahezu beliebig.
% …ist idealerweise ohne Fehler möglich.
% Bin ich zu optimistisch? :P

Um die Fehler in \autoref{sec:auswertung:brechungsindex} besser vergleichen und einordnen zu können,
wird die Änderung des Brechungsindex $\symup{\Delta}n$
statt des absoluten Wertes $n$ betrachtet.
Für den Literaturwert $n_\text{lit}$ \cite{brechzahl} wird diese durch Umstellen von \autoref{eqn:brechungsindex} ermittelt:
\begin{equation*}
  \symup{\Delta}n_\text{lit} = \frac{n_\text{lit} - 1}{\frac{T}{T_0} \cdot \frac{p_0}{Δp}} = \num{0.000150} \; .
\end{equation*}

Dann ergibt sich eine Abweichung von $\SI{19.74(2453)}{\percent}$,
wenn alle Messwerte berücksichtigt werden,
beziehungsweise $\SI{1.837(16365)}{\percent}$,
wenn nur Werte zu positiver Druckänderung ($\uparrow$) betrachtet werden.
Offenbar verringert letzteres Vorgehen den systematischen Fehler,
während die schon in den Ausgangsdaten beobachtbaren statistischen Fehler weiterhin durchschlagen.

Es sei erwähnt,
dass die prozentuale Abweichung von $n$ mit
$\SI{0.0054(67)}{\percent}$ beziehungsweise $\SI{0.00050(445)}{\percent}$
deutlich geringer ist.


\subsection{Mögliche Fehlerquellen}
\label{sec:diskussion:fehlerquellen}

Aufgrund der hohen Empfindlichkeit des Interferometers
konnten auch kleine Erschütterungen zu einer ungewollten Auslösung des Zählers führen.

Insbesondere bei der Erzeugung des Grobvakuums
mithilfe einer Handpumpe
ließen sich solche Fehler beobachten,
da die verwendeten Schläuche einen Teil der Bewegungen übertrugen.

Daher könnten einerseits die Werte mit negativer Druckänderung ($\downarrow$),
während derer also die handbetriebene Vakuumpumpe zur Erzeugung eines Grobvakuums benutzt wurde,
zu groß sein,
da bei diesem Vorgang die Übertragung von Erschütterungen über die Schläuche nicht zu vermeiden war.

Andererseits könnten diese Werte auch zu niedrig sein,
da sich mit der Handpumpe nur sprunghafte Druckänderungen erzeugen lassen,
was schnelle Phasendurchgänge % ist das ein Wort?
zur Folge hatte,
welche vom Zähler möglicherweise nicht registriert wurden.

Dieses Problem bezieht sich in erster Linie auf die positiven,
aber auch auf die negativen Druckänderungen,
da das Belüftungsventil ebenfalls schwer zu kontrollieren war.

% - Ablesegenauigkeit der Mikrometerschraube
