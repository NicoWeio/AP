\section{Theorie} \label{sec:theorie}

\subsection{Problemstellung} \label{sec:problem}

    Die Kraft $F$, die auf eine Fläche wirkt, wird als Spannung bezeichnet.
    Man unterscheidet zwischen der senkrecht wirkenden Komponente, der Normalspannung $\sigma$,
    und der parallel zur Fläche wirkenden Komponente, der Tangential- oder Schubspannung.
    Die Normalspannung ist proportional zur Längenänderung $\symup{\Delta}L$ und wird
    durch das Hookesche Gesetz beschrieben:
    \begin{equation}
        \sigma = E \cdot \frac{\symup{\Delta}L}{L} .
    \end{equation}
    Das Elastizitätsmodul $E$ stellt eine wichtige Materialkonstante eines Stoffes dar und 
    ist hier ein Proportionalitätsfaktor.
    Diese Größe kann mithilfe der Biegung eines Körpers bestimmt werden, 
    wobei die Durchbiegung durch eine Funktion $D(x)$ beschrieben wird.
    $D(x)$ setzt sich aus der Auslenkung $D_0(x)$ bei einer sogenannten Nullmessung, 
    einer Messung ohne angehängtes Gewicht, und einer Auslenkung $D_\text{M}(x)$  bei einer
    Messung mit angehängtem Gewicht zusammen.
    Es gilt
    \begin{equation}
        D(x) = D_\text{M}(x) - D_0(x) .
    \end{equation}

    Im Folgenden sollen zwei Möglichkeiten zur Berechnung der Biegung vorgestellt werden.

\subsection{Durchbiegung eines homogenen Stabes bei einseitiger Einspannung} \label{sec:einseitig}

    Die erste Möglichkeit besteht darin, einen Stab an einer Seite einzuspannen.
    Die Funktion $D(x)$ kann durch ein Drehmoment $M_\text{F}$ berechnet werden, welches die anliegende Kraft F erzeugt.
    Das Drehmoment bewirkt eine Änderung des Querschnitts des Stabes. Die oberen Schichten werden gedehnt und die unteren
    gestaucht. Dazwischen liegt die neutrale Faser, eine Fläche, die ihre ursprüngliche Länge beibehält.
    Im Inneren treten Normalspannungen auf, die ein inneres Drehmoment $M_{\symup{\sigma}}$ erzeugen. 
    Der Stab wird soweit gebogen, bis das äußere und innere Drehmoment gleich sind: 
    \begin{equation}
        M_{\symup{\sigma}} = M_\text{F}
    \end{equation}
    mit 
    \begin{equation}
        M_{\symup{\sigma}} = \int_Q y\sigma(y) dq 
    \end{equation}
    und
    \begin{equation}
        M_\text{F} = F (L-x) .
    \end{equation}
    Die Variable $x$ stellt eine beliebige Stelle auf dem Stab dar, $Q$ ist der Querschnitt des Stabes und $y$ beschreibt den 
    Abstand zwischen Flächenelement dq und der neutralen Faser.
    Für $D(x)$ ergibt sich 
    \begin{equation}
        D(x) = \frac{F}{2EI} \left(Lx^2 - \frac{x^3}{3}\right) \label{eqn:einsD}
    \end{equation}
    für $0 \leq x \leq L$. Die Variable $I$ stellt das Flächenträgheitsmoment dar.

\subsection{Durchbiegung eines homogenen Stabes bei beidseitiger Einspannung} \label{sec:beidseitig}

    Die zweite Möglichkeit besteht darin, den Stab an beiden Enden einzuspannen und die Kraft in der 
    Mitte des Stabes angreifen zu lassen.
    Für das äußere Drehmoment $M_\text{F}$ ergibt sich 
    \begin{equation}
        M_\text{F} = - \frac{F}{2} x 
    \end{equation}
    für $0 \leq x \leq \frac{L}{2}$ und
    \begin{equation}
        M_\text{F} = \frac{F}{2} (L-x)
    \end{equation}
    für $\frac{L}{2} \leq x \leq L$.
    Für $D(x)$ ergibt sich 
    \begin{equation}
        D(x) = \frac{F}{48EI} (3L^2 x - 4x^3) \label{eqn:beidD1}
    \end{equation}
    für $0 \leq x \leq \frac{L}{2}$ und 
    \begin{equation}
        D(x) = \frac{F}{48EI} (4x^3 - 12Lx^2 + 9L^2 x - L^3) \label{eqn:beidD2}
    \end{equation}
    für $\frac{L}{2} \leq x \leq L$. \\ \\ \\

Das Elastizitätsmodul kann durch Umformen der Gleichungen \eqref{eqn:einsD} , \eqref{eqn:beidD1} oder \eqref{eqn:beidD2}
für den jeweiligen Fall berechnet werden.