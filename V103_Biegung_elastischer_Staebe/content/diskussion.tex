\section{Diskussion}

Im Folgenden sollen die Ergebnisse der Messung diskutiert werden.
Ein möglicher Grund für Unregelmäßigkeiten kann das ungenaue Ablesen der Messuhren sein.
Der genaue Messwert ließ sich auf den verwendeten Uhren teilweise nur durch Umdenken in der Skalierung 
ablesen, sodass die abgelesenen Werte noch einmal umgerechnet werden mussten, um in der richtigen 
Größenordnung zu liegen. Die Messung müsste mit einer anderen Messuhr wiederholt werden, um diese 
Messunsicherheiten zu minimieren.
Teilweise war auf der Unterschied des Auslenkung zwischen zwei Messpunkten sehr gering, beziehungsweise 
kaum wirklich zu erkennen, was zu einer konstanten Verteilung der Messwerte führt, die eventuell von
Literaturwerten abweichen können.
Bei der Messung des beidseitig eingespannten Stabes können Abweichungen entstanden sein, da 
die Kraft nicht bei jeder Messung an derselben Stelle angegriffen hat, aufgrund des Wechsels von Nullmessung
und Messung mit Gewicht. Die Schwierigkeit bestand auch darin, das Gewicht genau in die Mitte des Stabes 
zu hängen, da die Enden aufgrund der Messapparatur nicht zu sehen waren.
Auch bei der Messung des einseitig eingespannten Stabes können an dieser Stelle Abweichungen entstanden sein.

