\section{Durchführung} \label{sec:durchführung}

\subsection{Messung des Bauteile} \label{sec:bauteile}

    Zu Beginn des Versuchs werden ein runder und ein eckiger Stab ausgemessen.
    Es wird die Länge, der Radius und die Seitenlängen der Stäbe jeweils fünfmal mit 
    einer Schieblehre gemessen.

\subsection{Schematischer Aufbau} \label{sec:aufbau} %Umbenennen?
%TODO: Abbildung einfügen

\subsection{Einseitige Einspannung}

    Zuerst wird der runde Stab wie in Abbildung in Kapitel \ref{sec:aufbau} in die Messapparatur eingespannt. %TODO: Referenz zur Abbildung
    Eine der verschiebbaren Messuhren wird an den Anfang der Messskala geschoben, während die andere
    nicht verwendet wird.
    Anschließend wird ein Gewicht gewählt, sodass die maximale Auslenkung des Stabes zwischen 3 mm und 7 mm liegt.
    Das Gewicht wird mit einer elektronischen Waage fünfmal gemessen.
    Um nun die Durchbiegung des Stabes zu messen, wird die Messuhr die Skala entlanggeschoben und in 
    regelmäßigen Abständen wird zuerst die Auslenkung $D_0(x)$ des runden Stabes ohne angehängtes Gewicht 
    und dann mit angehängtem Gewicht, $D_\text{M}$, abgelesen.
    Es werden 20 Messungen gemacht.

    Die Messung für den eckigen Stab ist analog.

\subsection{Beidseitige Einspannung}

    Diese Messung wird nur für den runden Stab durchgeführt.
    Der Stab wird in der Messapparatur an den Enden an den Punkten A und B (siehe Abbildung) %TODO: Referenz zur Abbildung
    befestigt.
    Es wird ein Gewicht bestimmt, mit dem die maximale Auslenkung des Stabes zwischen 3 mm und 7 mm liegt 
    und mit einer elektronischen Waage fünfmal ausgemessen.
    Eine Messuhr wird an den Anfang der Messskala geschoben, die andere an das Ende.
    Um die Durchbiegung zu messen, wird die Messuhr die Skala entlanggeschoben und in regelmäßigen Abständen
    wird eine Nullmessung und eine Messung mit angehängtem Gewicht durchgeführt.
    Es werden 16 Messungen gemacht.



