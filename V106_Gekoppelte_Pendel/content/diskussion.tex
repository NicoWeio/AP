\section{Diskussion}
\label{sec:diskussion}

\subsection{Abweichungen}

Die Abweichungen wurden bereits in den jeweiligen Unterkapiteln der Auswertung aufgeführt.
Zusammenfassend lässt sich sagen,
dass die relativen Abweichungen
für gleichsinnige und gegensinnige Schwingungen stets unter $\SI{5}{\percent}$ lagen.
% Im Rahmen der Möglichkeiten des Versuchsaufbaus sind die Ergebnisse also als \enquote{gut} zu bewerten.
Im Rahmen der Möglichkeiten des Versuchsaufbaus können diese Ergebnisse also als Erfolg verbucht werden.

Größere Abweichungen finden sich bei den Schwebungsfrequenzen in \autoref{sec:auswertung:gleichsinnig};
die experimentell bestimmte Schwebungsfrequenz des $\SI{1}{\meter}$-Pendels
weicht um über ein Viertel vom \enquote{theoretisch} berechneten Wert ab.

% Es darf jedoch nicht unerwähnt bleiben,
Es muss betont werden,
dass die experimentell bestimmte Kopplungskonstante $K$ benutzt wurde,
um besagte „Theoriewerte“ zu berechnen (siehe z.B. \autoref{eqn:dauer_gegensinnig}).
Ein unabhängiger Referenzwert steht nicht zur Verfügung.


\subsection{Mögliche Fehlerquellen}

Eine wesentliche Fehlerquelle ist das manuelle Erfassen der Periodendauern.
Um bessere Werte zu erzielen,
wurde zwar meist über mehrere Periodendauern hinweg
und von zwei Personen gleichzeitig
gemessen,
dennoch unterliegen die Werte schon hierbei einer gewissen Unsicherheit.

Ebenfalls manuell erfolgte das anfängliche Auslenken der Pendel.
Diese könnten beim Loslassen eine zusätzliche Beschleunigung erfahren haben.
Auch der anfängliche Auslenkwinkel konnte mangels Skala o.ä. nur geschätzt werden
und war bei den Messungen somit nicht konsistent.

Für die Berechnungen wurde angenommen,
dass das Trägheitsmoment (abhängig von Länge und Masse)
beider Pendel gleich ist.
Wie bereits in \autoref{sec:auswertung:gleichsinnig} erwähnt wurde,
stimmen die Schwingungsdauern $T_1$ und $T_2$ in beiden Durchläufen
im Rahmen der Messgenauigkeit überein,
sodass die Annahme weitestgehend zutrifft.

Die Kleinwinkelnäherung ist – offensichtlich – nur eine Näherung.
Wenngleich mit Winkeln $< \SI{10}{\degree}$ gearbeitet wurde,
wird durch diese Näherung ein Fehler induziert.

Durch Reibung an der Aufhängung (Spitzenlagerung),
welche zwar optimiert, aber nicht ideal ist,
aber auch durch Luftwiderstand,
wird das Pendel gebremst.
Da nicht davon auszugehen ist,
dass diese Bremseffekte genau linear von der Geschwindigkeit abhängen,
kann dies die Periodendauer beeinflussen.
% https://de.wikipedia.org/wiki/Schwingung#Schwingfall
