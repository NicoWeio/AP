\section{Auswertung}
\label{sec:auswertung}

Im Folgenden sind beide Durchgänge stets
unter Angabe der jeweiligen Pendellänge
nebeneinander aufgeführt.

\subsection{Gleichsinnige Schwingungen}
\label{sec:auswertung:gleichsinnig}

Tatsächlich wurde nicht wirklich eine gleichsinnige Schwingung gemessen;
stattdessen werden die einzeln bestimmten Schwingungsdauern $T_1$ und $T_2$ als Datengrundlage genommen.
(Das ist vertretbar, weil aus \autoref{eqn:dauer_gleichsinnig} hervorgeht,
dass die Schwingungsdauern nicht von der Kopplungskonstante abhängen,
sodass auch $K=0$ gewählt und somit die koppelnde Feder entfernt werden kann.)
Entsprechend wird in \autoref{tab:gleichsinnig} zwischen dem linken und rechten Pendel unterschieden,
aber nachher über alle Werte gemittelt.

Es lässt sich aber erkennen,
dass die Schwingungsdauern links ($T_1$) und rechts ($T_2$) in beiden Durchläufen
im Rahmen der Messgenauigkeit übereinstimmen.

Die aufgelisteten Werte wurden bereits durch die Anzahl der pro Messung gezählten Perioden dividiert,
sodass es sich tatsächlich um Periodendauern handelt.

\begin{table}
    \centering
    \caption{Periodendauern der gleichsinnigen Schwingung.}
    \label{tab:gleichsinnig}
    \begin{tabular}{r S S S S}
        \toprule
        & \multicolumn{4}{c}{$T_+ \mathbin{/} \si{\second}$} \\
        \cmidrule(lr){2-5}
        Pendellänge $l \mathbin{/} \si{\meter}$ &
        \multicolumn{2}{S}{0.5} &
        \multicolumn{2}{S}{1} \\
        \cmidrule(lr){2-3}
        \cmidrule(lr){4-5}
        & {links ($T_1$)}
        & {rechts ($T_2$)}
        & {links ($T_1$)}
        & {rechts ($T_2$)} \\
        \midrule
        & 1.38 & 1.40 & 1.94 & 1.98 \\
        & 1.44 & 1.41 & 1.94 & 1.92 \\
        & 1.30 & 1.41 & 1.96 & 1.93 \\
        & 1.40 & 1.44 & 1.96 & 1.97 \\
        & 1.40 & 1.41 & 1.94 & 1.95 \\
        & 1.41 & 1.45 & 1.97 & 1.96 \\
        & 1.42 & 1.43 & 1.97 & 1.96 \\
        & 1.43 & 1.43 & 1.98 & 1.95 \\
        & 1.41 & 1.43 & 1.99 & 1.96 \\
        & 1.43 & 1.42 & 1.95 & 1.95 \\
        & 1.44 & 1.39 & 1.95 & 1.90 \\
        & 1.41 & 1.42 & 1.91 & 1.91 \\
        & 1.39 & 1.42 & 1.91 & 1.89 \\
        & 1.41 & 1.41 & 1.93 & 1.93 \\
        & 1.41 & 1.43 & 1.93 & 1.95 \\
        & 1.40 & 1.40 & 1.94 & 1.93 \\
        & 1.41 & 1.42 & 1.94 & 1.93 \\
        & 1.36 & 1.40 & 1.91 & 1.91 \\
        & 1.41 & 1.41 & 1.93 & 1.92 \\
        & 1.42 & 1.44 & 1.92 & 1.95 \\
        \midrule
        % \hyperref[eqn:mittelwert]{Mittelwerte} $\bar{T_1}$, $\bar{T_2}$ &
        % 1.403 & 1.418 & 1.943 & 1.938 \\
        % Standardabweichung &
        % 0.0304 & 0.0141 & 0.0217 & 0.0237 \\
        % \midrule
        \hyperref[eqn:mittelwert]{Mittelwert} $\bar{T_+}$ &
        \multicolumn{2}{S}{1.410} &
        \multicolumn{2}{S}{1.941} \\
        \hyperref[eqn:standardabweichung]{Standardabweichung des Mittelwerts} &
        \multicolumn{2}{S}{0.00398} &
        \multicolumn{2}{S}{0.00365} \\
        \bottomrule
    \end{tabular}
\end{table}

Der in den folgenden Tabellen verwendete Mittelwert berechnet sich mittels folgender Formel:
\begin{equation}
    \bar{T}=\frac{1}{n}\sum_{\textrm{i=1}}^n T_\textrm{i} \ .
    \label{eqn:mittelwert}
\end{equation}
Die Standardabweichung des Mittelwerts wird mit
\begin{equation}
    \sigma = \sqrt{\frac{\sum_{i=1}^{n}(x_i-\bar{x})^2}{n}}
    \label{eqn:standardabweichung}
\end{equation}
berechnet.

Die prozentuale Abweichung wird stets als
\begin{equation}
  \frac{\omega_\text{exp} - \omega_\text{theo}}{\omega_\text{theo}} \cdot \SI{100}{\percent}
  \label{eqn:abweichung_prozent}
\end{equation}
angegeben.

\begin{table}
    \centering
    \caption{Vergleich der experimentell bestimmten mit der berechneten Schwingungsfrequenz für gleichsinnige Schwingungen.}
    \label{tab:omega_gleichsinnig}
    \begin{tabular}{r S S}
        \toprule
        & \multicolumn{2}{c}{$\omega_+ \mathbin{/} \si{\radian\per\second}$} \\
        \cmidrule(lr){2-3}
        Pendellänge $l \mathbin{/} \si{\meter}$ &
        0.5 &
        1 \\
        \midrule
        gemessen (nach \autoref{eqn:omega_T}) & 4.456 & 3.238 \\
        berechnet (nach \autoref{eqn:frequenz_gleichsinnig}) & 4.429 & 3.132 \\
        \midrule
        \hyperref[eqn:abweichung_prozent]{Abweichung in \%} & 0.61 & 3.39 \\
        \bottomrule
    \end{tabular}
\end{table}


\subsection{Gegensinnige Schwingungen}
\label{sec:auswertung:gegensinnig}

\autoref{tab:gegensinnig} zeigt die Periodendauern der gegensinnigen Schwingungen
sowie die zugehörigen Mittelwerte und Standardabweichungen.

\begin{table}
    \centering
    \caption{Periodendauern der gegensinnigen Schwingung.}
    \label{tab:gegensinnig}
    \begin{tabular}{r S S}
        \toprule
        & \multicolumn{2}{c}{$T_- \mathbin{/} \si{\second}$} \\
        \cmidrule(lr){2-3}
        Pendellänge $l \mathbin{/} \si{\meter}$ &
        0.5 &
        1 \\
        \midrule
        & 1.37 & 1.89 \\
        & 1.36 & 1.93 \\
        & 1.38 & 1.91 \\
        & 1.37 & 1.92 \\
        & 1.35 & 1.89 \\
        & 1.39 & 1.92 \\
        & 1.40 & 1.94 \\
        & 1.38 & 1.93 \\
        & 1.37 & 1.93 \\
        & 1.37 & 1.93 \\
        & 1.35 & 1.89 \\
        & 1.34 & 1.87 \\
        & 1.36 & 1.91 \\
        & 1.38 & 1.88 \\
        & 1.36 & 1.90 \\
        & 1.36 & 1.87 \\
        & 1.35 & 1.88 \\
        & 1.33 & 1.91 \\
        & 1.37 & 1.96 \\
        & 1.39 & 1.92 \\
        \midrule
        \hyperref[eqn:mittelwert]{Mittelwert} $\bar{T}$                       & 1.366   & 1.910 \\
        \hyperref[eqn:standardabweichung]{Standardabweichung des Mittelwerts} & 0.00380 & 0.00548 \\
        \bottomrule
    \end{tabular}
\end{table}

Die in \autoref{tab:omega_gegensinnig} angegebenen berechneten Werte
greifen bereits auf die nach \autoref{eqn:kopplungskonstante} berechnete Kopplungskonstante $K$ zurück.

\begin{table}
    \centering
    \caption{Vergleich der experimentell bestimmten mit der berechneten Schwingungsfrequenz für gegensinnige Schwingungen.}
    \label{tab:omega_gegensinnig}
    \begin{tabular}{r S S}
        \toprule
        & \multicolumn{2}{c}{$\omega_- \mathbin{/} \si{\radian\per\second}$} \\
        \cmidrule(lr){2-3}
        Pendellänge $l \mathbin{/} \si{\meter}$ &
        0.5 &
        1 \\
        \midrule
        gemessen (nach \autoref{eqn:omega_T}) & 4.601 & 3.290 \\
        berechnet (nach \autoref{eqn:frequenz_gegensinnig}) & 4.443 & 3.137 \\
        \midrule
        \hyperref[eqn:abweichung_prozent]{Abweichung in \%} & 3.55 & 4.90 \\
        \bottomrule
    \end{tabular}
\end{table}


\subsection{Gekoppelte Schwingungen}
\label{sec:auswertung:gekoppelt}

In \autoref{tab:gekoppelt} sind Schwingungsdauern $T$ und Schwebungsdauern $T_S$ sowie Mittelwerte und Standardabweichungen der gekoppelten Schwingung aufgeführt.
Für das $\SI{1}{\meter}$-Pendel wurde $T_S$ nur in ganzen Sekunden gemessen.

\begin{table}
    \centering
    \caption{Schwingungsdauern $T$ und Schwebungsdauern $T_S$ der gekoppelten Schwingung.}
    \label{tab:gekoppelt}
    \begin{tabular}{r S S S S}
        \toprule
        Pendellänge $l \mathbin{/} \si{\meter}$ &
        \multicolumn{2}{S}{0.5} &
        \multicolumn{2}{S}{1} \\
        \cmidrule(lr){2-3}
        \cmidrule(lr){4-5}
        & {$T   \mathbin{/} \si{\second}$}
        & {$T_S \mathbin{/} \si{\second}$}
        & {$T   \mathbin{/} \si{\second}$}
        & {$T_S \mathbin{/} \si{\second}$} \\
        \midrule
        & 1.40 & 37.23 & 1.97 & 92.00 \\
        & 1.38 & 39.80 & 1.92 & 95.00 \\
        & 1.42 & 36.91 & 1.92 & 96.00 \\
        & 1.37 & 37.75 & 1.93 & 95.00 \\
        & 1.40 & 37.73 & 1.95 & 96.00 \\
        & 1.37 & 36.85 & 1.86 & 92.00 \\
        & 1.37 & 38.63 & 1.91 & 93.00 \\
        & 1.37 & 36.78 & 1.91 & 96.00 \\
        & 1.41 & 36.44 & 1.89 & 95.00 \\
        & 1.45 & 37.97 & 1.92 & 97.00 \\
        \midrule
        \hyperref[eqn:mittelwert]{Mittelwert} &
        1.395 & 37.609 & 1.920 & 94.7 \\
        \hyperref[eqn:standardabweichung]{Standardabweichung des Mittelwerts} &
        0.00857 & 0.321 & 0.00993 & 0.559 \\
        \bottomrule
    \end{tabular}
\end{table}

\begin{table}
    \centering
    \caption{Vergleich der experimentell bestimmten mit der berechneten Schwebungsfrequenz für gekoppelte Schwingungen.}
    \label{tab:omega_gekoppelt}
    \begin{tabular}{r S S}
        \toprule
        & \multicolumn{2}{c}{$\omega_S \mathbin{/} \si{\radian\per\second}$} \\
        \cmidrule(lr){2-3}
        Pendellänge $l \mathbin{/} \si{\meter}$ &
        0.5 &
        1 \\
        \midrule
        gemessen (nach \autoref{eqn:omega_T})             & 0.167 & 0.0663 \\
        berechnet (nach \autoref{eqn:frequenz_schwebung}) & 0.145 & 0.0530 \\
        \midrule
        \hyperref[eqn:abweichung_prozent]{Abweichung in \%} & 15.08 & 26.03 \\
        \bottomrule
    \end{tabular}
\end{table}


\subsection{Kopplungskonstante}

Aus den experimentellen Werten zu $\omega_+$ und $\omega_-$
kann mithilfe von \autoref{eqn:kopplungskonstante} die Kopplungskonstante $K$ bestimmt werden.
Sie ergibt sich zu
$\SI{0.032(4)}{\meter\per\square\second}$ für das $\SI{0.5}{\meter}$-Pendel
beziehungsweise
$\SI{0.0161(34)}{\meter\per\square\second}$ für das $\SI{1}{\meter}$-Pendel.
% Es ist sinnvoll, dass sich unterschiedliche Werte ergeben,
% weil wir auch die Position der Feder geändert haben – glaube ich.
