\section{Durchführung}

    Im Folgenden sollen der Versuchsaufbau
    und die Durchführung der verschiedenen Messungen erläutert werden.


    Für den Versuch sind zwei Pendel sowie ein Stoppuhr gegeben.
    Die Pendel bestehen aus einem dünnen Stahlstab,
    welcher so befestigt ist,
    dass das Pendel möglichst reibungsfrei schwingen kann.
    Auf dem Stab ist eine Masse $m = \SI{1}{\kilo\gram}$ befestigt,
    welche sich auf dem Stab verschieben lässt,
    sodass verschiedene Pendellängen $l$,
    welche vom Punkt der Aufhängung zum Mittelpunkt der Masse gemessen werden,
    eingestellt werden können.
    Bei der Messung darf das Pendel nicht zu weit ausgelenkt werden,
    um die Gültigkeit der \hyperref[eqn:kleinwinkelnaeherung]{Kleinwinkelnäherung} zu gewährleisten.
    Die maximale Auslenkung $x$,
    unter der Annahme,
    dass der Auslenkwinkel $\alpha < \SI{10}{\degree}$ beträgt,
    lässt sich mithilfe der Gleichung
    \begin{equation}
        x = \sin{(\alpha)} \cdot l
    \end{equation}
    berechnen.


    Es werden in verschiedenen Messungen die Schwingungsdauern der Pendel gemessen.\\
    Zu Beginn der Messung wird die Pendellänge wie beschrieben mit einem Maßband bestimmt,
    wobei diese bei beiden Pendeln ungefähr gleich sein sollte.
    Anschließend wird die Schwingungsdauer $T_+$ der einzelnen Pendel beobachtet,
    was gleichzeitig auch der Messung der gleichsinnigen Schwingung in \autoref{sec:gleichsinnige_schwingung} entspricht,
    bei der keine Feder zur Kopplung benötigt wird.
    Es werden für beide Pendel jeweils 20 Messwerte aufgenommen,
    indem die Zeit für fünf Schwingungsdauern gemessen wird.\\
    Anschließend wird dieselbe Messung für die gegensinnige Schwingung in \autoref{sec:gegensinnige_schwingung} durchgeführt,
    wobei hier die Feder wieder in den Versuchsaufbau integriert wird.
    Außerdem muss darauf geachtet werden,
    dass sich die gegeneinander schwingenden Massen nicht berühren,
    weshalb beide Pendel anfangs \textit{zueinander} ausgelenkt werden.
    Es werden wieder 20 Messwerte für je fünf Schwingungsdauern $T_-$ aufgenommen.\\
    Danach wird die gekoppelte Schwingung aus \autoref{sec:gekoppelte_schwingung} untersucht.
    Dazu wird zuerst die Schwingungsdauer $T$ \textit{eines} Pendels betrachtet,
    indem dieses ausgelenkt
    und das andere festgehalten wird.
    Beim Loslassen werden zehnmal die Zeiten für je fünf Schwingungsdauern gemessen.
    Anschließend wird die Schwebungsdauer $T_\text{S}$ gemessen;
    dazu werden die Pendel wie vorher ausgelenkt und es wird die Zeit gemessen,
    bis das ursprünglich ruhende Pendel vom schwingenden Pendel angeregt wurde und wieder zur Ruhe gekommen ist.
    Hier wird zehnmal eine Schwebungsdauer gemessen.


    Sämtliche obenstehende Messungen werden für eine weitere Pendellänge wiederholt.


    Die Kopplungskonstante $K$ kann anschließend mit der \autoref{eqn:kopplungskonstante} berechnet werden.
